\documentclass{article}
\usepackage[utf8]{inputenc}
\usepackage[dvips]{graphicx}
\usepackage{a4wide}
\usepackage{amsmath}
\usepackage{euscript}
\usepackage{amssymb}
\usepackage{amsthm}
\usepackage{amsopn}
\usepackage{mathtools}

\theoremstyle{definition}
\newtheorem*{definition}{Definition}
\newtheorem{theorem}{Theorem}
\newcommand{\cis}{\mbox{cis}}
\newcommand{\vv}{\ensuremath{\vec{v}}}
\newcommand{\vu}{\ensuremath{\vec{u}}}
\newcommand{\vw}{\ensuremath{\vec{w}}}
\newcommand{\vx}{\ensuremath{\vec{x}}}
\newcommand{\vy}{\ensuremath{\vec{y}}}
\newcommand{\vb}{\ensuremath{\vec{b}}}
\newcommand{\vo}{\ensuremath{\vec{0}}}
\newcommand{\va}{\ensuremath{\vec{a}}}
\newcommand{\ve}{\ensuremath{\vec{e}}}
\newcommand{\deriv}{\frac{d}{dz}}

\usepackage{halloweenmath, tikzsymbols}

\newcommand{\R}{\mathbb{R}}
\newcommand{\Z}{\mathbb{Z}}
\newcommand{\C}{\mathbb{C}}
\newcommand{\N}{\mathbb{N}}
\newcommand{\Q}{\mathbb{Q}}
\newcommand{\Arg}{\mbox{Arg}}
\newcommand{\Log}{\mbox{Log}}

\newcommand{\cs}[1]{\color{blue}{#1}\normalcolor}
\newcommand{\ab}[1]{\color{red}{#1}\normalcolor}

\title{Complex Analysis}
\author{ajbergquist }
\date{August 2021}

\begin{document}
\fbox{maximum modulus principle}\\ If $f$ is non-constant and analytic on a domain $D$, then $f(z)$ has no max in $D$. \\

\fbox{202}\\

\fbox{corollary} Suppose $f$ is continuous on a closed, bounded region $R$ and $f$ is nonconstant and analytic in $\int R$. Then $\max|f(z)|$ occurs in $\partial R$ and not in $\int R$. \\

\fbox{corollary} If $u$ is harmonic on the closed and bounded region $R\subseteq \C$, and non-constant interior to $R$, then its maximum occurs on $\partial R$. \\

\fbox{proof} Let $v$ be the harmonic conjugate of $u$ on $\R$. Then $z = u(x,y) + iv(x,y)$ is analytic and non-constant on $\int R$. Therefore $e^z$ is also analytic interior to $R$, and non-constant. Therefore $|e^z|$ has its maximum on $\partial R$ by MMP. Since $e^x$ must be increasing, the max of $e^u$ occurs at the same points as the max of $u$. Therefore the max of $u$ is on $\partial R$. \\


\end{document}