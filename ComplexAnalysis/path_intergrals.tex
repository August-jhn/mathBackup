\documentclass{article}
\usepackage[utf8]{inputenc}
\usepackage[dvips]{graphicx}
\usepackage{a4wide}
\usepackage{amsmath}
\usepackage{euscript}
\usepackage{amssymb}
\usepackage{amsthm}
\usepackage{amsopn}

\theoremstyle{definition}
\newtheorem*{definition}{Definition}
\newtheorem{theorem}{Theorem}
\newcommand{\cis}{\mbox{cis}}
\newcommand{\vv}{\ensuremath{\vec{v}}}
\newcommand{\vu}{\ensuremath{\vec{u}}}
\newcommand{\vw}{\ensuremath{\vec{w}}}
\newcommand{\vx}{\ensuremath{\vec{x}}}
\newcommand{\vy}{\ensuremath{\vec{y}}}
\newcommand{\vb}{\ensuremath{\vec{b}}}
\newcommand{\vo}{\ensuremath{\vec{0}}}
\newcommand{\va}{\ensuremath{\vec{a}}}
\newcommand{\ve}{\ensuremath{\vec{e}}}
\newcommand{\deriv}{\frac{d}{dz}}


\usepackage{halloweenmath, tikzsymbols}

\newcommand{\R}{\mathbb{R}}
\newcommand{\Z}{\mathbb{Z}}
\newcommand{\C}{\mathbb{C}}
\newcommand{\N}{\mathbb{N}}
\newcommand{\Q}{\mathbb{Q}}
\newcommand{\Arg}{\mbox{Arg}}
\newcommand{\Log}{\mbox{Log}}

\newcommand{\cs}[1]{\color{blue}{#1}\normalcolor}

\title{Complex Analysis}
\author{ajbergquist }
\date{August 2021}

\begin{document}
\fbox{thm} Let $w:\R\rightarrow \C$ and $t\mapsto u(t) + iv(t).$ Then $w(t) = u(t) + iv(t)$, $w = <u,v>$. ALso $u,v:\R\rightarrow \R$ This parametrizes a curve in $\C$. Also, to find the derivative we just do $w'(t) = u'(t) + iv'(t).$ Also $ d/dt(z_0w(t)) = d/dt((x_0 + iy_0)(u(t)+iv(t))) = x_0u'(t) - y_0 v'(t) + i(y_0u'(t) + x_0v'(t)) = z_0w'(t).$\\

\fbox{remark} so we have a constant multiple rule for parametrized curves on the complex plain!\\

\fbox{thm} How about $d/dt e^{z_0 t} = d/dt(\exp(x_0 t + iy_0t)) = d/dte^{x_0 t}(\cos y_0t + i\sin y_0t) = d/dt(\mbox{them we just apply the product rule})$.\\

A whole bunch of stuff happens, and then we get $z_0 e^{z_0 t}.$ We could just do this ourself, and its very tedious to write in Latex.\\

\fbox{thm} The mean value theorem fails.\\

\fbox{proof} Consider $e^{it}$ for $t\in [0,2\pi].$ The mean value here would be $0$. But the derivative of $e^{it}$ can never be zero.\\

\fbox{now for integrals}\\

\fbox{definition} Let $x = x(t)$, $y  = y(t)$ be cont on $[a,b]$. The set of points $C : z = x(t) + iy(t)$ is an arc in $\C$. C is a Jordan arc if $z : [a,b] \rightarrow \C$ is 1-1. If $z(a) = z(b)$ but otherwise $z$ is 1-1. C is a jordan curve (simplified curve.).\\

\fbox{defn} Suppose $C:z(t) = x(t) = iy(t)$ has a derivative on $[a,b]$, and $x'$ and $y'$ are continuous on that interval. Then $C$ is a differentiable arc. Its length is defined $\int_a^b |z'(t)| dt$. The unit tangent $T$ is just the derivative divided by the magnitude of the derivative. \\

\fbox{defn} A contour is a piecewise smooth curve. A simicircle is a piecewise smooth curve, a non-example would be a fractal.\\

\fbox{thm} If $C$ is a simple closed contour, then $C$ partitions $\C$ into three sets: $C$, $D_1$ and $D_2$ such that 
\begin{itemize}
    \item $D_1$ and $D_2$ are domains with boundaries $C$
    \item One of $D_1$ and $D_2$ is boudned and the other is not. 
\end{itemize}

Let $C:z=z(t)$ on $[a,b]$ be a contour from $z_1 = z(a)$ to $z_2= z(b)$. Assume $f$ is a function that is piecewise continuous on $C$. The contour integral of $f$ along $C$ is 

$$\int_Cf(z)dz = \int_a^bf(z(t))z'(t)dt$$

Suppose $t = \phi(\tau)$ on $[\alpha,\beta]$. Assume $\phi$ and $\phi'$ are continuous. Suppose $\phi'(\tau) > 0$.\\


\fbox{theorem} $|\int_a^bw(t)dt| \le \int_a^b|w(t)|dt$.\\

\fbox{theorem} Suppose that on a contour $C$ of length $L$, $|f(z)|\le M$ and $f$ is piecewise continuous on $C$. Then $|\int_Cf(z)dz|\le ML$.\\

\fbox{proof} Assume $C:z = z(t)$. Then $|\int_Cf(z)dz| = |\int_Cf(z(t))z'(t)dt| \le \int_C|f(z(t))||z'(t)|dt \le \int_CM|z'(t)|dt = ML$.\\


\fbox{theorem} Recall: $F$ is an antideriative of a continuous function $f$ if $F' = f$ throughout some domain $D$, where $F$ is analytic on $D$. TFAE
\begin{enumerate}
    \item $f$ has an antiderivative on $D$
    \item For all contours $C_1$, $C_2\subseteq D$ joining fized points $z_1$ and $z_2$ in $D$, then $\int_{C_1}f(z)dz = \int_{C_2}f(z)dz$.
    \item If $C\subseteq D$ is a closed contour, then $\int_Cf(z)dz = 0$
\end{enumerate}
\fbox{proof} Let $C\subseteq D$ be a contour. Then there exist arcs $C_1,C_2,...,C_n$ that are smooth.\\


\end{document}


