\documentclass{article}
\usepackage[utf8]{inputenc}
\usepackage[dvips]{graphicx}
\usepackage{a4wide}
\usepackage{amsmath}
\usepackage{euscript}
\usepackage{amssymb}
\usepackage{amsthm}
\usepackage{amsopn}
\usepackage{mathtools}

\theoremstyle{definition}
\newtheorem*{definition}{Definition}
\newtheorem{theorem}{Theorem}
\newcommand{\cis}{\mbox{cis}}
\newcommand{\vv}{\ensuremath{\vec{v}}}
\newcommand{\vu}{\ensuremath{\vec{u}}}
\newcommand{\vw}{\ensuremath{\vec{w}}}
\newcommand{\vx}{\ensuremath{\vec{x}}}
\newcommand{\vy}{\ensuremath{\vec{y}}}
\newcommand{\vb}{\ensuremath{\vec{b}}}
\newcommand{\vo}{\ensuremath{\vec{0}}}
\newcommand{\va}{\ensuremath{\vec{a}}}
\newcommand{\ve}{\ensuremath{\vec{e}}}
\newcommand{\deriv}{\frac{d}{dz}}

\usepackage{halloweenmath, tikzsymbols}

\newcommand{\R}{\mathbb{R}}
\newcommand{\Z}{\mathbb{Z}}
\newcommand{\C}{\mathbb{C}}
\newcommand{\N}{\mathbb{N}}
\newcommand{\Q}{\mathbb{Q}}
\newcommand{\Arg}{\mbox{Arg}}
\newcommand{\Log}{\mbox{Log}}

\newcommand{\cs}[1]{\color{blue}{#1}\normalcolor}
\newcommand{\ab}[1]{\color{red}{#1}\normalcolor}

\title{Complex Analysis}
\author{ajbergquist }
\date{August 2021}

\begin{document}
\fbox{proposition} Let $C_0$ be a positively oriented circle such that $|z-z_0|<R$ for some $z_0\in \C$ and for some $R\in \R$. It follows from a previous problem that $$\int_{C_0}(z-z_0)^{n-1}dz = \Big\{\begin{array}{cc}
     &  0, n\in \Z^\times\\
     & 2\pi i, n= 0
\end{array}.$$ Let $C$ be a positively oriented curve around the rectangle defined $0\le x\le 3$ and $0\le y \le 2$. Then $$\int_{C}(z-z_0)^{n-1}dz = \Big\{\begin{array}{cc}
     &  0, n\in \Z^\times\\
     & 2\pi i, n= 0
\end{array}.$$\\

\fbox{proof} For $n>1$, the integrand is analytic (in fact entire) everywhere, its derivative (everywhere) being $(n-1)(z-z_0)^{n-2}$. Since it is analytic everywhere, it is also analytic between the curves $C_0$ and $C$. \cs{Is that clear/true? I don't have $C$ in front of me. But some $R$ will work, anyway.}\\\ab{I think my argument was that it's analytic everywhere, either way it would have an antiderivative everywhere, so the integral would be zero right?}
Suppose then that $n\le 1$. Then at $z_0$, the integrand is undefined. Since $R$ can be any old real number, choose $R = 1$. This places $C_0$ well within the interior of $C$. Furthermore, notice that $(z-z)^{n-1}$ is analytic for all points other than $z_0$. Since $z_0$ is in the interior of $C_0$, and not between $C$ and $C_0$, it follows that the integrand is analytic at all points between $C_0$ and $C$. \\

Since $n$ is either less than one, or greater than or equal to one, it follows that for all $n$, the integrand is analytic between $C$ and $C_0$. Then by the principle of deformation it follows that 
$\int_{C}(z-z_0)^{n-1}dz = \int_{C_0}(z-z_0)^{n-1}dz$, for all $n$. Hence the proposition holds.

\cs{5/5}
\end{document}