\documentclass{article}
\usepackage[utf8]{inputenc}
\usepackage[dvips]{graphicx}
\usepackage{a4wide}
\usepackage{amsmath}
\usepackage{euscript}
\usepackage{amssymb}
\usepackage{amsthm}
\usepackage{amsopn}
\usepackage{mathtools}
\usepackage{polynom}

\theoremstyle{definition}
\newtheorem*{definition}{Definition}
\newtheorem{theorem}{Theorem}
\newcommand{\cis}{\mbox{cis}}
\newcommand{\vv}{\ensuremath{\vec{v}}}
\newcommand{\vu}{\ensuremath{\vec{u}}}
\newcommand{\vw}{\ensuremath{\vec{w}}}
\newcommand{\vx}{\ensuremath{\vec{x}}}
\newcommand{\vy}{\ensuremath{\vec{y}}}
\newcommand{\vb}{\ensuremath{\vec{b}}}
\newcommand{\vo}{\ensuremath{\vec{0}}}
\newcommand{\va}{\ensuremath{\vec{a}}}
\newcommand{\ve}{\ensuremath{\vec{e}}}
\newcommand{\deriv}{\frac{d}{dz}}

\usepackage{halloweenmath, tikzsymbols}

\newcommand{\R}{\mathbb{R}}
\newcommand{\Z}{\mathbb{Z}}
\newcommand{\C}{\mathbb{C}}
\newcommand{\N}{\mathbb{N}}
\newcommand{\Q}{\mathbb{Q}}
\newcommand{\Arg}{\mbox{Arg}}
\newcommand{\Log}{\mbox{Log}}
\newcommand{\defn}{\fbox{definition}}
\newcommand{\thm}{\fbox{theorem}}
\newcommand{\infsum}{\sum_{n = 1}^{\infty}}
\newcommand{\pf}{\fbox{proof}}
\newcommand{\cor}{\fbox{corollary}}
\newcommand{\psum}{\sum_{n = 0}^N}
\newcommand{\prop}{\fbox{proposition}}


\newcommand{\cs}[1]{\color{blue}{#1}\normalcolor}
\newcommand{\ab}[1]{\color{red}{#1}\normalcolor}

\title{Complex Analysis}
\author{ajbergquist }
\date{August 2021}

\begin{document}
\fbox{problem 1} Classify the singular point of $f(z) = \frac{\cos z}{z},$ and find the residue, $B$, around it.\\

\fbox{solution} The singular point is a pole, of order 1. To see this, notice that the Maclaurin series for $\cos z$ is $\cos z = \infsum (-1)^nz^{2n}/(2n)! = 1/z(1-)$ \cs{What?}, valid for all $|z| < \infty$. Hence, excluding the point where the function is undefined, we have $1/z\cos z = 1/z(1-z^2/2 + z^4/24 -z^6/720 + z^8/40320-z^{10}/3628800+...) = 1/z - ...$ $0< |z| < \infty$. The first term of -1st power is $1/z$, and this is the only such therm. Hence the principle part of the Laurent series is $1/z$. Hence we have here a pole of order $1$ of residue $1$.\\

\cs{5/5}

\fbox{problem} Classify the singular point of $f(z) = \frac{1-e^{2z}}{z^4}$, and find the resudue around it.\\


\\
\fbox{solution} See how $f(z) = \frac{1-e^{2z}}{z^4}$ has a singular point at zero, and this is the only one. Using the maclaurin expansion for $e^{z} = \infsum z^n/n!$ we have $1-e^{2z} = 1- \infsum(2z)^n/n! = 1/z^4(-2z - 2z^2 - 4/3z^3 - 16z^4/24...) =  -2/z^3 - 2/z^2 - 4/3/z - 16/24 -...$. The principle part of this has three terms, hence we have here a pole of order 3. The term of -1st power has a coefficient of -4/3, hence the residue around the singular point of $f(z)$ is -4/3.\\

\cs{5/5}

\cs{Sec 79: 10/10}


\end{document}

