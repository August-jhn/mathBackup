\documentclass{article}
\usepackage[utf8]{inputenc}

\title{Complex Analysis}
\author{ajbergquist }
\date{August 2021}

\begin{document}
\fbox{section 7}\\
\\

\fbox{definition} Polar form: $z = (z,y) = (r\cos(\theta),r\sin(\theta))$,\\
$r = |z|\ge 0$.\\
Note that $\tan(\theta) = \frac{y}{x}$.\\

\\
\fbox{Euler's Formula}: $e^{i\theta} = \cos{\theta}+i\sin{\theta}$\\
\\
\fbox{definition} The argument of $z$, $\arg{z}$, is the set of angles by which $z$ can be represented. That is, $\arg{z} = \{\theta+2k\pi:z=|z|e^{i\theta},k\in \Z\}$.\\
\\
\fbox{definiton} the principle argument of $z$, denoted $\Arg{z} = \theta$, where $z=|z|e^{i\theta}$, and where $-\pi <\theta\le \pi$.\\
\\
\fbox{theorem} For a circle on the complex plane of radius $R$ centered at the origin, this is described by the set of real solutions for $z$ where $z = Re^{i\theta}$.\\
\\

\fbox{Euler's idenity} $e^{i\pi} + 1 = 0$.
\\\\

\fbox{section 8}\\

\fbox{theorem} Some properties of polar form stuff. \\
$e^{i\theta_1}e^{i\theta_2} = e^{i(\theta_1\theta_2)}. $\\
for $z= re^{i\theta}$, $z^{-1} = \frac{1}{r}e^{-i\theta}.$ \\
Note that $Argz_1z_2$ does not necessarily equal $Argz_1 + Argz_2$. The generic argument adds, but the principle argument does not.\\
\fbox{theorem} DeMoivere's Formula: $(\cos\theta+i\sin\theta)^n = \cos n \theta + i\sin n \theta.$\\
\fbox{proof}
$(e^{i\theta})^n = e^{in\theta} = \cos n \theta + i\sin n \theta$.
\end{document}
