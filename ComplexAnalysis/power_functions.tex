\documentclass{article}
\usepackage[utf8]{inputenc}
\usepackage[dvips]{graphicx}
\usepackage{a4wide}
\usepackage{amsmath}
\usepackage{euscript}
\usepackage{amssymb}
\usepackage{amsthm}
\usepackage{amsopn}

\theoremstyle{definition}
\newtheorem*{definition}{Definition}
\newtheorem{theorem}{Theorem}
\newcommand{\cis}{\mbox{cis}}
\newcommand{\vv}{\ensuremath{\vec{v}}}
\newcommand{\vu}{\ensuremath{\vec{u}}}
\newcommand{\vw}{\ensuremath{\vec{w}}}
\newcommand{\vx}{\ensuremath{\vec{x}}}
\newcommand{\vy}{\ensuremath{\vec{y}}}
\newcommand{\vb}{\ensuremath{\vec{b}}}
\newcommand{\vo}{\ensuremath{\vec{0}}}
\newcommand{\va}{\ensuremath{\vec{a}}}
\newcommand{\ve}{\ensuremath{\vec{e}}}
\newcommand{\deriv}{\frac{d}{dz}}
\newcommand{\pv}{\mbox{P.V. }}
\newcommand{\Log}{\mbox{Log}}

\usepackage{halloweenmath, tikzsymbols}

\newcommand{\R}{\mathbb{R}}
\newcommand{\Z}{\mathbb{Z}}
\newcommand{\C}{\mathbb{C}}
\newcommand{\N}{\mathbb{N}}
\newcommand{\Q}{\mathbb{Q}}

\title{Complex Analysis}
\author{ajbergquist }
\date{August 2021}

\begin{document}
We already have some notions of how to do rational powers. So let's make a definition:\\
\fbox{definition}
$z^c \equiv e^{c\log z}$.\\


Using this we can easilty find $i^i$\\

If we take a branch cut $\alpha < \theta < \alpha + 2\pi$, where $z = re^{i\theta}$, then $z^c = e^{c\log z}$ is analytic there.\\

\fbox{theorem} $\deriv z^c = \deriv e^{c\log z} = e^{c\log z}c/z = cz^{c-1}.$\\

\fbox{definition} The principal value of $z^c$ is called $\pv z^c = e^{c\Log{z}}$\\

\fbox{remark} Once we take a specific branch cut for the logarithm, there are no surprises.


\end{document}

