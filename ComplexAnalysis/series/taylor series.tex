\documentclass{article}
\usepackage[utf8]{inputenc}
\usepackage[dvips]{graphicx}
\usepackage{a4wide}
\usepackage{amsmath}
\usepackage{euscript}
\usepackage{amssymb}
\usepackage{amsthm}
\usepackage{amsopn}
\usepackage{mathtools}

\theoremstyle{definition}
\newtheorem*{definition}{Definition}
\newtheorem{theorem}{Theorem}
\newcommand{\cis}{\mbox{cis}}
\newcommand{\vv}{\ensuremath{\vec{v}}}
\newcommand{\vu}{\ensuremath{\vec{u}}}
\newcommand{\vw}{\ensuremath{\vec{w}}}
\newcommand{\vx}{\ensuremath{\vec{x}}}
\newcommand{\vy}{\ensuremath{\vec{y}}}
\newcommand{\vb}{\ensuremath{\vec{b}}}
\newcommand{\vo}{\ensuremath{\vec{0}}}
\newcommand{\va}{\ensuremath{\vec{a}}}
\newcommand{\ve}{\ensuremath{\vec{e}}}
\newcommand{\deriv}{\frac{d}{dz}}

\usepackage{halloweenmath, tikzsymbols}

\newcommand{\R}{\mathbb{R}}
\newcommand{\Z}{\mathbb{Z}}
\newcommand{\C}{\mathbb{C}}
\newcommand{\N}{\mathbb{N}}
\newcommand{\Q}{\mathbb{Q}}
\newcommand{\Arg}{\mbox{Arg}}
\newcommand{\Log}{\mbox{Log}}
\newcommand{\defn}{\fbox{definition}}
\newcommand{\thm}{\fbox{theorem}}
\newcommand{\infsum}{\sum_{n = 1}^{\infty}}
\newcommand{\pf}{\fbox{proof}}
\newcommand{\cor}{\fbox{corollary}}
\newcommand{\psum}{\sum_{n = 1}^N}
\newcommand{\lemma}{\fbox{Lemma}}


\newcommand{\cs}[1]{\color{blue}{#1}\normalcolor}
\newcommand{\ab}[1]{\color{red}{#1}\normalcolor}

\title{Complex Analysis}
\author{ajbergquist }
\date{August 2021}

\begin{document}
\thm Given a function $f$ that is analytic throughout the disk $|z-z_0| < R_0$ for $z_0\in \C$ and $R_0 \in \R$, $f(z)$ can be represented as the series 
$$f(z) = \infsum a_n(z-z_0)^n$$
where $a_n = \frac{f^{(n)}(z_0)}{n!}$, for $n\in \N$.\\


\lemma In the case where $z_0 = 0$, Taylor's theorem holds. In other words if $f$ is analytic throughout $R$, where $R$ is the region defined by the disk $|z| < R_0$ for some $R_0\in \R$, it follows that $f(z) = \infsum \frac{f^{(n)}(z_0)}{n!}z^n$ for all $z\in R$.\\

\pf Let $z$ be an arbtirary point in $R$, and suppose that $f$ is analytic throughout $R$. Let $C_0$ be an arbitrary positively oriented simple closed contour contained within $R$. Since $f$ is by construction analytic on $C_0$, the Cauchy integral formula appplies,
$$f(z) = \frac{1}{2\pi i}\int_{C_0}\frac{f(s)ds}{s-z}$$



\end{document}