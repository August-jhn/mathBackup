\documentclass{article}
\usepackage[utf8]{inputenc}
\usepackage[dvips]{graphicx}
\usepackage{a4wide}
\usepackage{amsmath}
\usepackage{euscript}
\usepackage{amssymb}
\usepackage{amsthm}
\usepackage{amsopn}
\usepackage{mathtools}

\theoremstyle{definition}
\newtheorem*{definition}{Definition}
\newtheorem{theorem}{Theorem}
\newcommand{\cis}{\mbox{cis}}
\newcommand{\vv}{\ensuremath{\vec{v}}}
\newcommand{\vu}{\ensuremath{\vec{u}}}
\newcommand{\vw}{\ensuremath{\vec{w}}}
\newcommand{\vx}{\ensuremath{\vec{x}}}
\newcommand{\vy}{\ensuremath{\vec{y}}}
\newcommand{\vb}{\ensuremath{\vec{b}}}
\newcommand{\vo}{\ensuremath{\vec{0}}}
\newcommand{\va}{\ensuremath{\vec{a}}}
\newcommand{\ve}{\ensuremath{\vec{e}}}
\newcommand{\deriv}{\frac{d}{dz}}

\usepackage{halloweenmath, tikzsymbols}

\newcommand{\R}{\mathbb{R}}
\newcommand{\Z}{\mathbb{Z}}
\newcommand{\C}{\mathbb{C}}
\newcommand{\N}{\mathbb{N}}
\newcommand{\Q}{\mathbb{Q}}
\newcommand{\Arg}{\mbox{Arg}}
\newcommand{\Log}{\mbox{Log}}

\newcommand{\Re}{\mbox{Re}}
\newcommand{\Im}{\mbox{Im}}

\usepackage{ulem}
\usepackage{xcolor}
\newcommand{\cs}[1]{\color{blue}{#1}\normalcolor}
\newcommand{\cs}[1]{\color{blue}{#1}\normalcolor}
\newcommand{\ab}[1]{\color{red}{#1}\normalcolor}

\title{Complex Analysis}
\author{ajbergquist }
\date{August 2021}

\begin{document}
\maketitle
\fbox{proposition} Suppose that $f(z)$ is entire and that the harmonic function $u(x,y) = \Re[f(z)]$ has an upper bound $u_0$, that is, $u(x,y)\le u_0$ for all points $(x,y)$ in the $xy$ plane. Show that $u(x,y)$ must be constant throughout the plane.\\

\fbox{proof} Consider the function $g(z) = e^{f(z)}$. Let $v(x,y) = \Im(f(z))$. Then we have $g(z) = e^{u}e^{vi}$. Hence the modulus of $g(z)$ is $e^{u}$. Furthermore, since $u_0$ is the upper bound of $u$, $e^{u_0}$ must be the upper bound of $e^u$, hence the upper bound to the modulus of $g(z)$. Since $g(z)$ is bounded, and since entire functions are preserved under function composition (I think that could be proven using abstract algebra), it follows that $g(z)$ is entire as well. Hence by \sout{Louisville's} \cs{Liouville's} theorem it follows that $g(z)$ is constant. This being the case, the modulus of $g(z)$ doesn't change. The only way for this to happen is for $e^u$ to be constant, hence $u$ must be constant as well.

\cs{5/5}

\end{document}