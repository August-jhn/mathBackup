\documentclass{article}
\newcommand{\ii}{{\bf i}}
\newcommand{\jj}{{\bf j}}
\newcommand{\kk}{{\bf k}}
\newcommand{\id}{{\bf 1}}
\newcommand{\hur}{\frac{\id+\ii+\jj+\kk}{2}}%The "Hurwitz point"
\newcommand{\hurwitz}{\Z\left[\hur,\ii,\jj,\kk\right]}%The set of Hurwitz integers
\usepackage[utf8]{inputenc}
\usepackage[dvips]{graphicx}
\usepackage{a4wide}
\usepackage{amsmath}
\usepackage{euscript}
\usepackage{amssymb}
\usepackage{amsthm}
\usepackage{amsopn}
\usepackage[colorinlistoftodos]{todonotes}
\usepackage{graphicx}
\usepackage[T1]{fontenc}
\newcommand\mybar{\kern1pt\rule[-\dp\strutbox]{.8pt}{\baselineskip}\kern1pt}

\usepackage{ulem}
\usepackage{xcolor}
\newcommand{\cs}[1]{\color{blue}{#1}\normalcolor}

%Matrix commands
\newcommand{\ba}{\begin{array}}
\newcommand{\ea}{\end{array}}
\newcommand{\bmat}{\left[\begin{array}}
\newcommand{\emat}{\end{array}\right]}
\newcommand{\bdet}{\left|\begin{array}}
\newcommand{\edet}{\end{array}\right|}

%Environment commands
\newcommand{\be}{\begin{enumerate}}
\newcommand{\ee}{\end{enumerate}}
\newcommand{\bi}{\begin{itemize}}
\newcommand{\ei}{\end{itemize}}
\newcommand{\bt}{\begin{thm}}
\newcommand{\et}{\end{thm}}
\newcommand{\bp}{\begin{proof}}
\newcommand{\ep}{\end{proof}}
\newcommand{\bprop}{\begin{prop}}
\newcommand{\eprop}{\end{prop}}
\newcommand{\bl}{\begin{lemma}}
\newcommand{\el}{\end{lemma}}
\newcommand{\bc}{\begin{cor}}
\newcommand{\ec}{\end{cor}}
\newcommand{\lcm}{\mbox{lcm}}
\newcommand{\defn}{\fbox{definition}}
\newcommand{\prop}{\fbox{proposition}}
\newcommand{\stab}{\mbox{stab}}
\newcommand{\Aut}{\mbox{Aut}}



%sets of numbers
\newcommand{\N}{\mathbb{N}}
\newcommand{\Z}{\mathbb{Z}}
\newcommand{\Q}{\mathbb{Q}}
\newcommand{\R}{\mathbb{R}}


\begin{document}

\fbox{proposition 5} There are exactly $3$ subgroups of $\Aut(\Z_{35})$. \\

\fbox{proof}
It has been previously shown that $\Aut(\Z_{35})\cong U(35)$. Since isomorphism preserves subgroups, it suffices to show that there are exactly three cyclic subgroups of $U(35)$.\\
I will prove it below just in case it hasn't actually been in this class. For this part of the proof, assume it has.\\

Notice that $35 = 7*5$. By the Coralleries to Sunzi's remainder theorem, $U(35)\cong U(5)\oplus U(7)$. Furthermore, by Gauss' cool fact, we know that $U(5) = U(5^1)\cong \Z_4$ and $U(7)\cong \Z_6$. Since $\Z_4$ and $\Z_6$ are cyclic groups of order 6 and 3 respectively, it follows by the fourth and fifth properties of isomorphisms that $U(7)$ and $U(5)$ are cyclic groups of order $6$ and $7$ respectively. \\

By a corollary to the fundamental theorem of finite cyclic groups it follows that in $U(7)$ there are $\phi(6) = (3-1)(2-1) = 2$ elements of order 6, $\phi(3) = 2$ elements of order $3$, $\phi(2) = 1$ elements of order $1$, and $\phi(1) = 1$ elements of order $1$. Together, this accounts for the six elements in $U(7)$. Furthermore, by the same corollary it follows that in $U(4)$ there are $\phi(4) = 4-2 = 2$ elements of order $4$, $\phi(2) = 1$ element of order $1$, as well as one element of order $1$. This accounts for all $4$ elements of $U(5)$. Furthermore, by the fundamental theorem of finite cyclic groups there is exactly one cyclic subgroup in $U(7)$ of orders $1,2,3,6$, generated by the elements of those orders. Likewise, there is one cylcic subgroup in $U(5)$ of orders $1,2,4$, generated by elements of those orders. \\

Now we look at the possible combinations. Recall that according to the properties of external direct products, the orders of the elements of $U(7)\oplus U(5)$ are the least common multiple of the elements in the respective groups. Let $g = (a,b)$ be an element in $U(7)\oplus U(5)$. The possibilities for $|a|$ are, as we have shown, $6,3,2,1$. The possibilities for $|b|$ are $4,2,1$. Hence for elements of order six, and hence cyclic subgorups of order six, we can have $|a| = 6$ and $|b| = 1$, in which case $|g| = |(a,b)| = \lcm(6,1) = 6$. We can also have $|a| = 6, |b| = 2$, in which case $|g| = \lcm(6,2) = 6$. Finally, we can have $|a| = 3$ and $|b| = 2$, in which case $|g| = \lcm(3,2) = 6$. Since there are only three possible configurations for the orders of $a$ and $b$ in creating elements $(a,b)$ of order 6, there are exactly three cyclic subgroups of order $6$ in $U(7)\oplus U(5)$. \\
Since $U(7)\oplus U(5)\cong U(35)$, and $U(35) \cong \Aut(\Z_{35})$, and since isomorphism is transitive hence $U(7)\oplus U(5)\cong \AUt(\Z_{35})$, and since subgroups are preserved under isomorphism, it follows that there are exactly three cyclic subgroups of order six in $\Aut(\Z_{35}).$\\

\fbox{proposition 1} A subgroup $N$ of order $G$ is called the characteristic subgroup if $\psi(N) = N$ for all $\psi\in \Aut(G)$. Prove that every subgroup of a cyclic group is characteristic.\\

\fbox{proof} Let $H$ be a subgroup of a cyclic group $G$, and let $\psi \in \Aut(G)$. Define $\psi(H)$ to be the set of elements $\psi(H) = \{\psi(h): h\in H\}$. By the fundamental theorem of finite cyclic groups, $H$ is cyclic. Hence by definition of a cyclic group there exists some element $h\in H$ such that $H = <h>,$ and $|h| = |<h>| = |H|$. Since isomorphisms, including automorphisms, preserve element order $|\psi(h)| = |h|$. Hence $|<\psi(h)>| = |H|$.\\
Now we want to show that $\psi(H)$ is cyclic.
Let $a\in \psi(H)$. Then by definition of $\psi(H)$ it follows that $a = \psi(b)$ for some $b\in H$. Since $H = <h>$, by definition of cyclic groups it follows that $b = h^n$ for some $n\in \Z$. Then substituting we have that $a = \psi(b) = \psi(h^n)$. By property two of isomorphisms it follows that $a = \psi(h)^n$. Since $a$ is arbitrary in $\psi(H)$, it follows that for all elements $a\in \psi(H)$ there exists some $n\in \Z$ such that $a = \psi(h)^n$. Then by definition of cyclic groups, $\psi(H) = <\psi(h)>$. Since $|\psi(H)| = |<\psi(h)>| = |\psi(h)| = |h| = |H|$, we have that $|\psi(H)| = |H|.$\\

Since $H$ is a cyclic subgroup of order $|H|$ in $G$, and since $\psi(H)$ is also a cyclic subgroup of order $|H|$, and since by Lagrange's theorem $|H|\big | |G|$. Then by the fundamental theorem of finite cyclic groups it follows that $H = \psi(H)$, as there can be only one subgroup of $G$ whose order is a positive divisor of the order of $G$.\\

By definition of a characteristic subgroup, $H$ is a characteristic subgroup. Since $H$ is arbitrary in $G$, where $G$ is an arbitrary cyclic group, it follows that all subgroups of any cyclic group are characteristic subgroups.
\\

\fbox{proposition 3} If $H$ and $K$ are subgroups of $G$ such that $\gcd(|H|,|K|) = 1$, then $H\cap K = \{e\}$.\\
\fbox{proof} Let $H,K\le G$ for arbitrary subgropus $H,k$ in an arbtirary group $G$, such that $\gcd(|H|,|K|) = 1$. Since $H$ and $K$ are subgroups of $G$, they both must contain the identity. Hence $e\in H\cap K$.\\
Suppose by way of contradiction that there exists an element $a\in H\cap K$ such $a\ne e$. Since $a\ne e$, $|a| \ne 1$.\\

Since by the language of the GCD of the orders of $H$ and $K$ it is implied that both are finite subgroups. Then by a corollary to Lagrange's theorem it follows that $|a|\big ||H|$ and $|a|\big | |K|$. Hence $|a|\big|\gcd(|H|,|K|)$, and $|a|\big | 1 $. Since negative or zero order is meaningless, we have here that a positive integer that does not equal one divides one, which is a contradiction. Hence there does not exist an element other than $e$ in $H\cap K$. Hence $H\cap K = \{e\}$.\\


\end{document}