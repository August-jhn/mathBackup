\documentclass{article}
\usepackage[utf8]{inputenc}
\documentclass{article}
\usepackage[dvips]{graphicx}
\usepackage{a4wide}
\usepackage{amsmath}
\usepackage{euscript}
\usepackage{amssymb}
\usepackage{amsthm}
\usepackage{amsopn}

\theoremstyle{definition}
\newtheorem*{definition}{Definition}
\newtheorem{theorem}{Theorem}

\newcommand{\vv}{\ensuremath{\vec{v}}}
\newcommand{\vu}{\ensuremath{\vec{u}}}
\newcommand{\vw}{\ensuremath{\vec{w}}}
\newcommand{\vx}{\ensuremath{\vec{x}}}
\newcommand{\vy}{\ensuremath{\vec{y}}}
\newcommand{\vb}{\ensuremath{\vec{b}}}
\newcommand{\vo}{\ensuremath{\vec{0}}}
\newcommand{\va}{\ensuremath{\vec{a}}}
\newcommand{\ve}{\ensuremath{\vec{e}}}

\newcommand{\R}{\mathbb{R}}
\newcommand{\Z}{\mathbb{Z}}
\newcommand{\C}{\mathbb{C}}
\newcommand{\N}{\mathbb{N}}
\newcommand{\Q}{\mathbb{Q}}
\title{Abstract Algebra}
\author{August, Evelyn}
\date{August 2021}
\begin{document}


\fbox{definition} A permutation of a set S is a bijection from S to itself. We can use cycle notation to represent these. So $(123).$\\
\fbox{def} A group $G = (G,*)$ is a nonempty set $G$ and an operation $*$ defined on the elements of $G$ satisfying all the following.
\begin{itemize}
    \item $*$ is a binary operation on $G$, that is, $*$ is closed on $G$.
    \item There exists an element $e\in G$ such that for all elements $a$ in $G$, $e*a = a = a*e$. This is called the $*$-identity.
    \item There are inverses. For all elements, $a$, in $G$, there exists some $a^{-1}$, called the inverse of $a$, satisfying $a*a^{-1} = e = a^{-1}*a$.
    \item The operation $*$ is associative, that is, $a*(b*c) = (a*b)*c$ for all $a,b,c\in G$.
\end{itemize}\\

\fbox{theorem/definition} The symmetries of a regular $2n$-gon with composition form a group. This group is called the dihedral group of order $2n$.\\

\fbox{definition} Given a group $(G,*)$, the cardinality of the set is denoted $|(G,*)|$.\\

\fbox{theorem} Roots of unity are a group.\\

\fbox{theorem: uniquenesses} The (1) inverse and (2) identity in a group are unique.\\
\fbox{proof}
\begin{enumerate}
    \item Let $a\in G$ and let $x,y\in G$ be inverses of $a$. Then by definition of inverses we have $ax = ay = e = xa = ya$. Multiplying $x$ on both sides on the left, we have by the associative property $x(ax) = x(ay) = (xa)x = (xa)y = ex = ey$. By the idenity property, we have $x = y$. Hence the inverse of any element in a group is unique.
    \item Let $e_1$ and $e_2$ be identity elements in an group $G$. Then by definition of the identity, $e_1 e_2 = e_1$, so the identity is unique. So basically the proof is trivial.
\end{enumerate}
\fbox{theorem: cancelation} The left and right cancelation laws hold:
$$\begin{array}{cc}
     & ba = ca \Rightarrow b = c, \\
     & ab = ac \Rightarrow b = c.
   \end{array}
 $$
 \fbox{proof} Suppose $ba = ca$. By definition of a group, $a$ has an inverse $a^{-1}$, so by the associative and inverse properties of groups, $c = ce = c(aa^{-1})b(aa^{-1}) = be = b$. The other way is trivially similar.
 \\
 
 \fbox{theorem} For group elements $a,b\in G$, we have $(ab)^{-1} = b^{-1}a^{-1}$.\\
 
 
 \fbox{theorem} Given a group $G$ with order $|G| = 2n$ for some natural number $n$, it follows that there exists some element $a\in G$ such that $|a| = 2$.\\
 
 \fbox{proof} Let $S =\{x\in G : x \ne x^{-1}\}$. Because this maps each element to another element, $|S| = 2n$. Now let $H = \{x\in G: x = x^{-1}\}.$ 
 
\end{document}