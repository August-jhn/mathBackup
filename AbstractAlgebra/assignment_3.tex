\documentclass{article}
\newcommand{\ii}{{\bf i}}
\newcommand{\jj}{{\bf j}}
\newcommand{\kk}{{\bf k}}
\newcommand{\id}{{\bf 1}}
\newcommand{\hur}{\frac{\id+\ii+\jj+\kk}{2}}%The "Hurwitz point"
\newcommand{\hurwitz}{\Z\left[\hur,\ii,\jj,\kk\right]}%The set of Hurwitz integers
\usepackage[utf8]{inputenc}
\usepackage[dvips]{graphicx}
\usepackage{a4wide}
\usepackage{amsmath}
\usepackage{euscript}
\usepackage{amssymb}
\usepackage{amsthm}
\usepackage{amsopn}
\usepackage[colorinlistoftodos]{todonotes}
\usepackage{graphicx}
\usepackage[T1]{fontenc}
\newcommand\mybar{\kern1pt\rule[-\dp\strutbox]{.8pt}{\baselineskip}\kern1pt}

\usepackage{ulem}
\usepackage{xcolor}
\newcommand{\cs}[1]{\color{blue}{#1}\normalcolor}

%Matrix commands
\newcommand{\ba}{\begin{array}}
\newcommand{\ea}{\end{array}}
\newcommand{\bmat}{\left[\begin{array}}
\newcommand{\emat}{\end{array}\right]}
\newcommand{\bdet}{\left|\begin{array}}
\newcommand{\edet}{\end{array}\right|}

%Environment commands
\newcommand{\be}{\begin{enumerate}}
\newcommand{\ee}{\end{enumerate}}
\newcommand{\bi}{\begin{itemize}}
\newcommand{\ei}{\end{itemize}}
\newcommand{\bt}{\begin{thm}}
\newcommand{\et}{\end{thm}}
\newcommand{\bp}{\begin{proof}}
\newcommand{\ep}{\end{proof}}
\newcommand{\bprop}{\begin{prop}}
\newcommand{\eprop}{\end{prop}}
\newcommand{\bl}{\begin{lemma}}
\newcommand{\el}{\end{lemma}}
\newcommand{\bc}{\begin{cor}}
\newcommand{\ec}{\end{cor}}
\newcommand{\lcm}{\mbox{lcm}}

%sets of numbers
\newcommand{\N}{\mathbb{N}}
\newcommand{\Z}{\mathbb{Z}}
\newcommand{\Q}{\mathbb{Q}}
\newcommand{\R}{\mathbb{R}}
\title{Abstract Algebra}
\author{August, Evelyn}
\date{9/21/2021}
\maketitle
\begin{document}
\fbox{Ch3, 4, proposition} In any group, an element an its inverse have the same order.\\

\fbox{"lemma"} Let $x$ be an arbitrary element of a group. Let $n$ be an arbitrary positive integer such that $x^n = e$. Then it follows that $(x^{-1})^n = e$.\\

\fbox{proof} 
By the associative law of groups it follows that $x^n = xx^{n-1} = e$. Hence by definition of the inverse, $x^{n-1} = x^{-1}.$ Hence we have $(x^{-1})^n = (x^{n-1})^n = x^{n(n-1)} = (x^n)^{n-1} = e^{n-1} = e.$\\


\fbox{proof} Let $G$ be an arbitrary group, and let $x$ be an arbitrary element in $G$. This proof will be broken up into two cases: infinite and finite order.\\

Suppose that $|x| = \infty$. By the inverse property of groups (do I have to say this, or could I jump right to it and let this be implied by the fact that $G$ is a group), there exists some $x^{-1}\in G$ such that $xx^{-1} = e$. Suppose by way of contradiction that $x^{-1} = x^n$ for some positive integer $n$. Then we have $xx^n = e$. But it follows then that $x^{n+1} = e$, contradicting the supposition that $|x| = \infty.$\\

Suppose then that $|x| = n$ for some positive integer $n$. Then it follows that $x^n = e$. Furthermore, by the associative law of groups it follows that $x^n = xx^{n-1} = e$. Hence by definition of the inverse, $x^{n-1} = x^{-1}.$ Hence we have $(x^{-1})^n = (x^{n-1})^n = x^{n(n-1)} = (x^n)^{n-1} = e^{n-1} = e.$ \\

Having shown that $n$ is a positive integer such that $(x^{-1})^n = e$, it remains to be shown that it is the smallest such integer. Suppose then by way of contradiction that there exists some positive integer $m > n$ such that $(x^{-1})^m = e$. By the lemma it follows that $x^m = e$. But this is a contradiction.\\

\vspace{.5cm}

\fbox{Ch 3, 77, proposition} Let $x$ be an arbitrary element of a group, $G$, such that $|x| = m$. Let $n$ be an arbitrary positive integer. If $\gcd(m,n) =1$, then $x = y^n$ for some $y\in G.$\\
\fbox{proof}
Let $x$ be an arbitrary element of an arbitrary group $G.$ Suppose that $|x| = m$, and let $n$ be a positive integer such that $\gcd(m,n) = 1$. By Bazout's identity it follows that $1 = ma + nb$ for integers $a$ and $b$. Hence we have $x  = x^1= x^{ma + nb} = x^{ma}x^{nb} = (x^{m})^a (x^{b})^n = (e)^a (x^{b})^n = (x^b)^n$. Call $x^b = y$. To be extra meticulous, applying the closure property it follows that $y \in G$. Since $x$ is arbitrary, it follows that for all $x$ in a group, there exists some positive integer $n$ such that $gcd(n,m)=1$ and $x=y^m$.\\

\fbox{Ch 4, 39, find a group with exactly six subgroups}. Try $\Z_6$ under addition? There is of course the trivial subgroup, $\{0\}$. Then there is itself. Then there $\{2,4,0\} = <2>$. Then there is $<3> = \{0,3\}$ Nope, that is only four. How about I try $\Z_{2^5}$ under addition. Yep, then I'd have $<0 = 32>, <1>,<2>,<4>,<8>$ and $<16>.$ This is six. Here's my generalization: cyclic groups with orders which have $n$ divisors must have exactly $n$ subgroups. This is a direct corollary of the fundamental theorem of cyclic subgroups.\\
\newpage
\fbox{Ch 4, 40, thoughts} We want to find a generator for $<m>\cap<n>$ given arbitrary $m,n\in \Z$. Try this, $\lcm(m,n).$ This seems to follow intuitively, because this will have less things relatively prime to it, hence less things "asseccible" to it. Let's try it out.

\fbox{proposition} Let $m,n\in \Z$ under addition. Let $\lcm(m,n) = a$. Consider the cyclic subgroup $<a>$. The rest of this proof shall be a sort of set-equality proof (as the operation is inherited and these are groups by definition of cyclic groups), to show that $<a> \le <m>\cap<n>$ and $<m>\cap<n>\le <a>$. Let $g = pa$ for some integer $p$ be an arbitrary element of $<a>$. Definition of the least common multiple, $m|a$ and $n|a$. Hence there exist integer $q,r$ such that $qm = a$ and $rn = a$. Then by definition of $<a>$ and $<m>$, it follows that $a\in <m>$ and $a\in <n>$. Furthermore, also by the definition of cyclic groups, since $p$ is an integer, $g = pa\in <m>$ and $g = pa \in <n>$. Then by definition of intersection, $g\in <m>$ and $g\in <n>$. Since $g$ is arbitrary, this applies to all $g$, hence $<\lcm(m,n)>\subseteq <m>\cap<n>$\\

Now let $x$ be an arbitrary element in $<m>\cap <n>$. Then it follows by definition of intersection that $x\in <m>$ and $x\in <n>$. By definition of cyclic groups, there exists integers $p$ and $q$ such that $pn = x = qm$. By definition of division, it follows that $n|x$ and $m|x$. By definition of the least common multiple (and the easily proven property that anything which divides both of the numbers whose least common multiple is common to must also be a multiple of the least common multiple), it follows that $a|x$. By definition of divisibility, there exists some integer $y$ such that $ya = x$. By definition of $<a>$, it follows that $x\in <a>$. Since $x$ is arbitrary, this applies to all elements of $<m>\cap<n>$, hence $<m>\cap<n>\le \Z$.

By set equality and the fact that both of these things are groups, it follows that $$<m>\cap<n> = <\lcm(m,n)>.$$

\fbox{remark} So then, is it true that $<m>\cup<n> = <\gcd(m,n)>$?\\ 
\fbox{proof} This proof is left as an exercise for the reader.\\


\fbox{Ch 4, 82, proposition} Let $G = \{ax^2 + bx + c : a,b,c\in \Z_3\}$. Under addition (mod 3), assume that $G$ is a group. Then $|G| = 27$ and $G$ not cyclic.\\
\fbox{proof} First I shall prove that $|G| = 27$. Since there are three possible values for each respective coefficient, we have $3^3 = 27$ different possibilities, hence $|G| = 27$.\\
Suppose that $G$ is cyclic. By the fundamental theorem of cyclic groups, there must be exactly one subgroup of order $k$. Now let $g$ be an arbitrary element of $G$. By definition of $G$, $g = [a]x^2 + [b]x + [c]$ for some $[a],[b],[c]\in \Z/(3)$. Consider $3g = ([a]+[a]+[a])x^2 + ([b]+[b]+[b])x + ([c]+[c]+[c])$. By properties of addition of modular congruence classes, $g = [3a]x^2 + [3b]x + [3c] = [3][a]x^2 + [3][b]x + [3][c] = [0],$ which is the identity element in $G$. Since $g$ is arbitrary, it follows that $3g = e$ for all $g\in G$. But by the fundamental theorem there must exist some $h\in G$ such that $|h| = 27$. Since $3$ is less than $27$, and since by what we have just shown, $3h = e$, $27$ cannot be the order of $h$. Hence we arrive at a contradiction, so $G$ must not be cyclic.
\end{document}