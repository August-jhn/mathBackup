\documentclass{article}
\newcommand{\ii}{{\bf i}}
\newcommand{\jj}{{\bf j}}
\newcommand{\kk}{{\bf k}}
\newcommand{\id}{{\bf 1}}
\newcommand{\hur}{\frac{\id+\ii+\jj+\kk}{2}}%The "Hurwitz point"
\newcommand{\hurwitz}{\Z\left[\hur,\ii,\jj,\kk\right]}%The set of Hurwitz integers
\usepackage[utf8]{inputenc}
\usepackage[dvips]{graphicx}
\usepackage{a4wide}
\usepackage{amsmath}
\usepackage{euscript}
\usepackage{amssymb}
\usepackage{amsthm}
\usepackage{amsopn}
\usepackage{listings}

%Matrix commands
\newcommand{\ba}{\begin{array}}
\newcommand{\ea}{\end{array}}
\newcommand{\bmat}{\left[\begin{array}}
\newcommand{\emat}{\end{array}\right]}
\newcommand{\bdet}{\left|\begin{array}}
\newcommand{\edet}{\end{array}\right|}

%Environment commands
\newcommand{\be}{\begin{enumerate}}
\newcommand{\ee}{\end{enumerate}}
\newcommand{\bi}{\begin{itemize}}
\newcommand{\ei}{\end{itemize}}
\newcommand{\bt}{\begin{thm}}
\newcommand{\et}{\end{thm}}
\newcommand{\bp}{\begin{proof}}
\newcommand{\ep}{\end{proof}}
\newcommand{\bprop}{\begin{prop}}
\newcommand{\eprop}{\end{prop}}
\newcommand{\bl}{\begin{lemma}}
\newcommand{\el}{\end{lemma}}
\newcommand{\bc}{\begin{cor}}
\newcommand{\ec}{\end{cor}}
\newcommand{\N}{\mathbb{N}}
\newcommand{\C}{\mathbb{C}}
\newcommand{\lcm}{\mbox{lcm}}


\usepackage{ulem}
\usepackage{xcolor}
\newcommand{\cs}[1]{\color{blue}{#1}\normalcolor}

\title{Abstract Algebra}
\author{August, Evelyn, revised *}
\date{9/28/2021}


\begin{document}
\maketitle
\fbox{19} What are the cyclic subgroups of $U(30)$.\\
\fbox{answer} Finding the cyclic subgroups of $u(30)$ can be done easily by making a simple program in python. Using the following simple program: 


\begin{lstlisting}[language=Python]

import math

u30 = set()

for x in range(30):
  if math.gcd(30,x) == 1:
    u30.add(x)
    print(x)
cycle = set()
for e in u30:
  cycle = set()
  done = False
  n = 1
  while not done:
    a = e**n % 30
    if a in cycle:
      done = True
    cycle.add(a)
    n += 1
  print(cycle)
\end{lstlisting},
we obtain the following cyclic subgroups, $<1> = \{1\}$, $<7> = <13> = \{1,7,13,19\}$, $<11> = \{1,11\}$, $<17> = <23> =  \{1, 17,19,23\}$ $<19> = \{1,19\}$, and $<29> = \{1,29\}$. In total, this is $6$ distinct cyclic subgroups of $U(30)$, including the trivial subgroup.\\

\fbox{20, proposition} Let $G$ be an Abelian group of order $35$ such that every element in $G$ satisfies the equation $x^{35} = e$. Then $G$ is cyclic.\\

\fbox{proof} Suppose $G$ is an Abelian group of order $35$ with the property that for every element $x\in G$ $x^{35} = e$. Let $x$ be an arbitrary element in G. By corollary 2 of theorem 4.1, $|x| | 35$. Then for all elements $x\in G$, $|x|$ can be $1,5, 7$, or $35$. By a corollary to theorem 4.4, we know that the number of elements in $G$ of order $d$ must be a non-negative (a negative number wouldn't make sense) multiple of $\phi(d)$. Considering all of our possible orders, and since the identity is the only element of order 1 which must be unique, we have the equation $1+\phi(5)a + \phi(7)b + \phi(35) c = 35: a,b,c\in \N_0.$ This equation simplifies to $2a + 3b + 12 c  = 17$. Since 2,3,12 are not divisors of $17$, at least two of these coefficients must be nonzero. Hence we have four options for zero coefficients: $a,b$ or $c$ zero or none. If the first three options hold and c is not zero, we're done (at least to the step of showing there is an element with order 35). Assume then that $c$ is zero. Then there exist some elements $x$ of order $5$ and $y$ of order $7$. Furthermore, by the closure property of groups $xy\in G$. By a previously proven theorem, and since $G$ is Abelian, the order of $xy$ must divide $|x||y| = 35$. If $|xy| = 5$, we have $x^5 y^5 = y^5 \ne e$, hence $|xy| \ne 5$. Likewise, $|xy| = 7$ implies that $x^7y^7 = x^7 = x^2 \ne e$, hence $|xy| \ne 7$. Finally $|xy| \ne 1$, as this would imply that $xy = e$, but then $x = y^{-1}$, and by a previously proven theorem (in the homeworks) it would follow that $x$ and $y$ have the same order, and by supposition they do not. Hence $|xy| = 35$. Furthermore, by theorem it follows that $|<xy>| = |G|$, hence $<xy> = G$, so $G$ is cyclic. Q.E.D.
\\


\fbox{remark!} Additionally, the proof also works after replacing 35 by 33. \\

\fbox{remark!!!} Does this work so long as we have exactly 2 prime factors for our replacement integer? Are there any other integers that work?\\

\fbox{lemma} Let $x,y\in G$ such that $xy = yx$, then $|xy| = \lcm(|x|,|y|).$\\
\fbox{proof} Let $x,y$ be as stated. Let $a = |x|$ and $b = |y|$. Then 

\fbox{proposition} Let $G$ be an Abelian group of order $35$ such that every element in $G$ satisfies the equation $x^{35} = e$. If $n$ is square free and has $2$ prime factors, then $G$ is cyclic.\\


\fbox{49, proposition} For each $n\in \N$, there are exactly $\phi(n)$ elements of order $n$ in $\C^*$. \\

\fbox{proof *} (forgot to specify that $n$ is a positive divisor of $n$) Let $n$ be some arbitrary natural number. Let $x$ be an arbitrary element of $\C^*$ such that $x^n = 1$.. Then this becomes the equation for the roots of unity, $x^n - 1 = 0$. The $n$ the roots of unity. Hence we have the set $\{x: x = e^{2\pi k i/n} \}$ for $k = 1,\dots,n$. Simple calculation reveals that this is a cyclic group generated by $e^{2\pi i/n}$, which has order $n$, call it $<1^{1/n}>$. Then since $n|n$, by theorem 4.4 the number of elements of order $n$ in $1^{1/n}$ (this notation is used in our complex analysis textbook) is $\phi(n)$. Hence the number of elements in $\C^*$ with order $n$ is exactly $\phi(n)$.\\

Q.E.D.

\fbox{remark} This example shows that there can be an element in a group of infinite order with finite order! That seems strange and amazing!\\

\fbox{58, question} How many solutions are there to the equation $x^{15} = e$ in a cyclic group $G$ where $15||G|$?.\\

\fbox{thoughts} By the fundamental theorem of finite cyclic groups, we know that there must be exactly one subgroup of $G$ with order $15$, as $15$ must be a positive divisor of $|G|$. Call it $<x>$ for some $x\in G$. By another theorem, each set of elements whose orders divide 15 must be a subgroup of this group. So there are exactly 15 elements in $G$ which satisfy this equation.\\

Actually, since $G$ is cyclic it follows that for each divisor of the order of $G$, $d$, the number of elements whose orders are $d$ is $\phi(d)$. If $x^{15} = e$, then by another theorem it follows that $|x| | 15$. So to account for each $x\in G$ which satisfies this condition we should also account the divisors, which, since divisibility is "transative," are also divisors of $|G|$ (in this case 15). The positive divisors of $15$ are $1,3,5,15$. Hence the number of solutions in $G$ to the equation $x^{15} = e$ is $\phi(1) + ...\phi(15) = 1+ 2+ 4+ 8 = 15$. \\

Also, from number theory we have a theorem that says that the sum of $\phi(d)$ for all positive divisors such that $d|n$ is $n$. In summation notation, 

$$\sum_{d|n}\phi(n) = n.$$ \\
\\
So we can generalize the statement as follows:\\
\fbox{proposition: 58}
If $G$ is a cyclic subgroup and $n$ is a natural number such that $n||G|$, then the number of elements $x\in G$ such that $x^n = e$ is exactly $n$.\\


\fbox{proof} Let $G$ be as stated in the proposition. Since $G$ is cyclic, by a previously proven theorem (theroem 4.4) it follows that since $n$ is a positive divisor of the order of $G$, there exists $\phi(n)$ elements in $G$ whose order are $n$, hence these satisfy the condition that $x^{n} = e$. But by another previously proven theorem, $x^n = e$ if and only if $n||x|$. Hence for each positive divisor $d$ of $n$, we have $\phi(d)$ more solutions. In other words, we have 
$$\sum_{d|n}\phi(n)$$ elements $x$ in $G$ such that $x^n = e$. By a result from number theory, this is just $n$. Q.E.D.

\fbox{36, proposition} Suppose that G is a group that has exactly one nontrivial proper subgroup. Prove that G is cyclic and $|G|=p^2$, where p is prime. \\

\fbox{proof} Suppose that G is a group. Let H be its only nontrivial proper subgroup. Then there exists a $x\in G$ that is not in H. Next, consider the cyclic subgroup that is generated by x:  $<x>$. This subgroup cannot be another proper subgroup, nor is it the trivial subgroup. Hence, $<x>$ must be G, which implies that G is cyclic. \\
Furthermore, by Theorem 4.3 there exists exactly one subgroup for every divisor of $|G|$. Since G has only one proper nontrivial subgroup, this means that $|G|$ has only one divisor that is not 1 and not itself. Thus, it follows that $|G|=p^2$, which can only be divided by 1, p and $p^2$.
\\

\end{document}