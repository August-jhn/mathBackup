\documentclass{article}
\newcommand{\ii}{{\bf i}}
\newcommand{\jj}{{\bf j}}
\newcommand{\kk}{{\bf k}}
\newcommand{\id}{{\bf 1}}
\newcommand{\hur}{\frac{\id+\ii+\jj+\kk}{2}}%The "Hurwitz point"
\newcommand{\hurwitz}{\Z\left[\hur,\ii,\jj,\kk\right]}%The set of Hurwitz integers
\usepackage[utf8]{inputenc}
\usepackage[dvips]{graphicx}
\usepackage{a4wide}
\usepackage{amsmath}
\usepackage{euscript}
\usepackage{amssymb}
\usepackage{amsthm}
\usepackage{amsopn}
\usepackage[colorinlistoftodos]{todonotes}
\usepackage{graphicx}

\usepackage{ulem}
\usepackage{xcolor}
\newcommand{\cs}[1]{\color{blue}{#1}\normalcolor}

%Matrix commands
\newcommand{\ba}{\begin{array}}
\newcommand{\ea}{\end{array}}
\newcommand{\bmat}{\left[\begin{array}}
\newcommand{\emat}{\end{array}\right]}
\newcommand{\bdet}{\left|\begin{array}}
\newcommand{\edet}{\end{array}\right|}

%Environment commands
\newcommand{\be}{\begin{enumerate}}
\newcommand{\ee}{\end{enumerate}}
\newcommand{\bi}{\begin{itemize}}
\newcommand{\ei}{\end{itemize}}
\newcommand{\bt}{\begin{thm}}
\newcommand{\et}{\end{thm}}
\newcommand{\bp}{\begin{proof}}
\newcommand{\ep}{\end{proof}}
\newcommand{\bprop}{\begin{prop}}
\newcommand{\eprop}{\end{prop}}
\newcommand{\bl}{\begin{lemma}}
\newcommand{\el}{\end{lemma}}
\newcommand{\bc}{\begin{cor}}
\newcommand{\ec}{\end{cor}}

%sets of numbers
\newcommand{\N}{\mathbb{N}}
\newcommand{\Z}{\mathbb{Z}}
\newcommand{\Q}{\mathbb{Q}}
\title{Abstract Algebra}
\author{August, Evelyn}
\date{5/9/2021}
\maketitle
\begin{document}
\fbox{theorem} Given some $n\in \N$, $\Z_n$, the set of equivalence classes under $\mod n$ equivalence relation and addition form a group.\\

\fbox{theorem} The set $U(n) = (\{[a]\in \Z_n:\gcd(n,a) = 1\},x)$ forms a group.\\


\fbox{notation} \textbf{Additive groups} use $a+b$ for the operation and $e = 0$ as the identity, $-a$ for inverses, $na$ for multiplie times \textbf{Multiplicative groups} use $axb = ab$ for the operation, $e = 1$ for the identity, and $a^{-1}$ for inverses, $a^n$ for multiple times.
\\

\fbox{isomorphism} Given two groups $(G,*)$ and $(H,+)$, we say that $G$ and $H$ are isomorphic (denoted $G\cong H$) iff there exists a bijection $\phi:G\rightarrow H$ such that $\phi(a*b) = \phi(a)+\phi(b)$\\

\fbox{theorem} Given a group $G$, and a nonempty subset of $H\subseteq G$, 
\begin{enumerate}
    \item H is closed under $*$,
    \tiem For all elements $h\in H$, there exists some $h^{-1}$ in $H$.
\end{enumerate},
Then $H\le G$.
\end{document}