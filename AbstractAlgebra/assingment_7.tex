\documentclass{article}
\usepackage[utf8]{inputenc}
\newcommand{\ii}{{\bf i}}
\newcommand{\jj}{{\bf j}}
\newcommand{\kk}{{\bf k}}
\newcommand{\id}{{\bf 1}}
\newcommand{\hur}{\frac{\id+\ii+\jj+\kk}{2}}%The "Hurwitz point"
\newcommand{\hurwitz}{\Z\left[\hur,\ii,\jj,\kk\right]}%The set of Hurwitz integers
\usepackage{wrapfig}
\usepackage[utf8]{inputenc}
\usepackage[dvips]{graphicx}
\usepackage{a4wide}
\usepackage{amsmath}
\usepackage{euscript}
\usepackage{amssymb}
\usepackage{amsthm}
\usepackage{amsopn}
\usepackage[colorinlistoftodos]{todonotes}
\usepackage{graphicx}
\usepackage[T1]{fontenc}
\newcommand\mybar{\kern1pt\rule[-\dp\strutbox]{.8pt}{\baselineskip}\kern1pt}

\usepackage{ulem}
\usepackage{xcolor}
\newcommand{\cs}[1]{\color{blue}{#1}\normalcolor}

%Matrix commands
\newcommand{\ba}{\begin{array}}
\newcommand{\ea}{\end{array}}
\newcommand{\bmat}{\left[\begin{array}}
\newcommand{\emat}{\end{array}\right]}
\newcommand{\bdet}{\left|\begin{array}}
\newcommand{\edet}{\end{array}\right|}

%Environment commands
\newcommand{\be}{\begin{enumerate}}
\newcommand{\ee}{\end{enumerate}}
\newcommand{\bi}{\begin{itemize}}
\newcommand{\ei}{\end{itemize}}
\newcommand{\bt}{\begin{thm}}
\newcommand{\et}{\end{thm}}
\newcommand{\bp}{\begin{proof}}
\newcommand{\ep}{\end{proof}}
\newcommand{\bprop}{\begin{prop}}
\newcommand{\eprop}{\end{prop}}
\newcommand{\bl}{\begin{lemma}}
\newcommand{\el}{\end{lemma}}
\newcommand{\bc}{\begin{cor}}
\newcommand{\ec}{\end{cor}}
\newcommand{\lcm}{\mbox{lcm}}
\newcommand{\defn}{\fbox{definition}}
\newcommand{\prop}{\fbox{proposition}}
\newcommand{\stab}{\mbox{stab}}
\newcommand{\Aut}{\mbox{Aut}}
\newcommand{\orb}{\mbox{orb}}

\newcommand{\and}{\wedge}
\newcommand{\or}{\vee}



%sets of numbers
\newcommand{\N}{\mathbb{N}}
\newcommand{\Z}{\mathbb{Z}}
\newcommand{\Q}{\mathbb{Q}}
\newcommand{\R}{\mathbb{R}}

\title{Abstract Algebra}
\author{August, Evelyn}
\date{10/19/2021}

\begin{document}
\maketitle
%#6, 12, 42, 45, 48, 61


\fbox{6: question} Let $n\in \N$, and let $H = \{mn: m\in \Z\}$. How many left cosets of $H$ in $Z$ are there?\\

\fbox{solutoin/proposition} There are $n-1$ distinct left cosets of of $H$.\\

\fbox{proof} Let $H$ be instantiated as in the question. By the properties of cosets, we know that given an element $a\in \Z$, $a+H = H$ if and only if $a\in H$. By definition of $H$, this occurs only when there exists some $m\in \Z$ such that $a = mn$. In other words $a\equiv 0\pmod(n)$. Call the negation of this Condition 1). Furthermore, given arbitrary integers $a$ and $b$, we have by another property of cosets that $a+H = b+ H$ if and only if $a\in b+H$. By definition of $H$ and of left cosets, this only occurs when there exists some $x\in \Z$ such that $a = b + xn$. Equivalently, this only happens when $a-b = xn$ for some $x\in \Z$, which means $n|a-b$, which also means that $a\equiv b\pmod(n)$. Call the negation of this condition 2). \\

Let $a$ be an arbitrary integer satisfying condition 1). Furthermore, we need to find the number of elements, $b$, satisfying condition 2). These will be elements which are not in the modular equivalence class of $a$ modulo $n$. Since $[a]$ by assumption of condition 1) is not $[0]$, and neither is $[b]$, we have $n-2$ other options. Including $[a]$ into this we have in total $n-1$ options. Hence there are $ n-1$ distinct left inverses of $H$. Q.E.D.


\fbox{12: proposition} Given a group $G$ such that $|G| = 155$, and elements $a,b\in G$ such that $a$ and $b$ are not the identity element, and $|a|\ne |b|$, it follows that any subgroup containing both $a$ and $b$ is itself $G$.\\

\fbox{proof} Let $G$ be instantiated as stated in the proposition, and let $a,b$ be elements as stated in the proposition. Note that the prime factorization of $155$ is $155 = 31\cdot5$. Hence the only positive divisors of $155$ are $1,5,31$ and $155$. By corollary 2 of Lagrange's theorem, the order of any subgroup divides the order of the group. Hence, given an arbitrary subgroup $H\le G$, it follows that $|H| - 155,31,5$ or $1$. Furthermore, since $|<a>| = |a|$ and $|<b>| = |b|$, and since $a,b$ are not the identity element, there are only three options for the orders of $a$ and $b$. Either $|a| = 31$ and $|b| = 5$, $|a| = 155$ and $|b| = 5$, or $|a| = 155$ and $|b| = 31$. (of course, we could interchange $a$ with $b$ for six more cases, but since $a$ and $b$ are arbitrary, we can narrow the cases down to these three). Suppose $H$ contains both $a$ and $b$. Then since the cyclic subgroups generated by $a$ and $b$ must by the closure property be subgroups of $H$, it follows by Corollary-2 that $|a|,|b| | |H|$.\\
Hence in either of the last two cases, $155| |H|$. But since $|H| | 155$, it follows that $|H| = 155 = |G|$, hence $H = G$. Suppose then that the first case is true, and that $|a| = 31$ and $|b| = 5$. Then $31,5||H|$. Then by properties of division, it follows that since $31$ and $5$ are relatively prime, $155 ||H|$. Once again, taking into account that $|H| | 155$, it follows that $|H| = 155 = |G|$. Hence in this case also $H = G$.\\

Having shown that for all possible orders of $a$ and $b$, $a,b\in H$ implies $H = G$, it follows for all non-identity elements $a,b\in G$ with different orders, if $a,b\in H$ then $H = G$. \\

\newpage

\fbox{42: proposition} Given a group $G$ with order $n$, and an integer $k$ relatively prime to $n$, the map $g\rightarrow g^k$ for all $g\in G$ is injective. Furthermore, if $G$ is Abelian then this map is an automorphism on $G$. \\

\fbox{proof} Let $G$ be a group of order $n$ and let $k$ be a positive integer relatively prime to $n$. To show that the map $g\rightarrow g^k$ is injective, let $g_1$ and $g_2$ be arbitrary elements in $G$ such that $g_1^k = g_2^l$. \\

By a theorem from chapter 4, we know that $|g_1^k| = |g_1|/\gcd(k,|g_1|)$ and $|g_2^k| = |g_2|/\gcd(k,|g_2|).$ Since $|g_1|$ is the order of the cyclic group generated by $g_1$, and likewise for $g_2$, By a corollary to Lagrange's theorem it follows that $|g_1|\big | n$ and $|g_2| \big | n $. Furthermore, since $n$ and $k$ are relatively prime, it follows that $k$ is relatively prime to the orders of $g_1$ and $g_2$ as well. Hence $\gcd(k,|g_1|) = 1 = \gcd(k,|g_2|)$. Substituting in, we have $|g_1^k| = |g_1|$ and $|g_2^k| = |g_2|$. By supposition that $g_1^k = g_2^k$, it follows by substitution that $|g_1^k| = |g_2^k| = |g_1| = |g_2|$\\

Furthermore, as we have shown that $\gcd(|g_1|,k) = 1$ and $|g_2| = |g_1|$, it follows by Bezout's identity that $x|g_1| + yk = 1$ for integers $x$ and $y$. Likewise, by substitution $1 = x|g_2| + yk$. So we have $g_1 = g_1^1 = g_1^{x|g_1| + yk} = (g_1^{|g_1|})^xg_1^{yk} = g_1^{yk} = (g_1^k)^y$. Likewise, for $g_2$, we have $g_2 = (g_2^k)^y$. By substitution, $g_1 = (g_1^k)^y = (g_2^k)^y = g_2$. Since $g_1$ and $g_2$ are arbitrary elements in $G$, it follows that for all $g_1,g_2\in G$, $g_1^k = g_2^k$ implies $g_1 = g_2$. Hence the map $g\rightarrow g^k$ is injective.
\\

\fbox{proof that this is an automorphism} To show that the map $g\rightarrow g^k$ is an automorphism, we must show that it is surjective and operation preserving. To show that it is surjective, let $h$ be an arbitrary element in $G$. By a corollary to Lagrange's theorem, $|h|\big | n$, since $|h| = |<h>|\le G$. Hence $\gcd(|h|,k) = 1$, and by Bezout's identity it follows that $1 = x|h| + yk$ for integers $x,y$. Hence $h = h^1 = h^{x|h| + yk} = (h^{|h|})^x(h^y)^k = (h^y)^k.$ Hence there exists some $g = h^y\in G$ such that $h = g^k$. Since $h$ is arbitrary, it follows that for all $h \in G$ there exists some $g\in G$ such that $h = g^k$. Hence the codomain of the map $g\rightarrow g^k$ not only is of the same cardinality, but in fact is the domain. Hence the map $g\rightarrow g^k$ is surjective. Since it is injective as well, it follows that it is bijective.\\

It remains to be shown that $g\rightarrow g^k$ is operation preserving. To show this, let $g_1$ and $g_2$ be arbitrary elements in $G$. Then $g_1g_2$ maps to $(g_1g_2)^k$, which by associativity is equal to $g_1^kg_2^k$, if $k>0$. If $k<0$, $(g_1g_2)^k$ is equal to $g_2^kg_1^k$ by associativity and the socks shoes property. However, if G is Abelian, it follows that $g_2^kg_1^k = g_1^kg_2^k$. Hence $g\rightarrow g^k$ is operation preserving. Thus, we have shown that the mapping is an automorphism. \\

\fbox{45: problem} Let $G=\{(1), (12)(34), (1234)(56), (13)(24), (1432)(56), (56)(13), (14)(23), (24)(56)\}$\\

Find $\stab(1)$ and $\orb(1)$\\

$stab(1)=\{(1), (24)(56)\}, orb(1)=\{1, 2, 3, 4\}$\\

Find stab(3) and orb(3)\\

$stab(3)=\{(1), (24)(56)\}=stab(1), orb(3)=\{1, 2, 3, 4\}=orb(3)$\\

Find stab(5) and orb(5)\\

$stab(5)=\{(1), (12)(34), (13)(24), (14)(23)\}, orb(5)=\{5, 6\}$\\

\newpage

\fbox{48: proposition} Let G be a group of order pqr (p, q and r are distinct primes). If H and K are subgroups of G with $|H|=pq \wedge |K|=qr$, prove that $|H\cap K|=q$. \\

\fbox{proof} Let G, H and K be groups or subgroups as instantiated above. By problem 32 in Chapter 3 we know that $H \cap K$ forms a subgroup of $G$, and hence also of $K$ and $H$. By Lagrange's Theorem, we know that the order of a subgroup divides the order of a finite group, which means that $|H \cap K| \big| |H|$ and $|H \cap K| \big| |K|$. Thus, the order of $H \cap K$ must divide both $pq$ and $qr$. Since $p, q$ and $r$ are prime, it follows that $|H \cap K|$ either $1$ or $q$. By Theorem 7.2 is follows that $|HK|=|H||K| /|H \cap K| = pq^2r$ or $pqr$. Since $HK$ is a subset of $G$, the order of $HK$ cannot be higher than $|G|=pqr$, hence it follows that $|H \cap K|=q$.      QED.\\

\fbox{61: proposition} Let $G=\GL(2, \R)$. Let $H$ be the subgroup of matrices of determinant $+1$ or $-1$. If $a, b \in \G$ and $aH = bH$, what can be said about $\det(a)$ and $\det(b)$? Prove or disprove the converse. \\

\fbox{proof} Let $a, b$ be arbitrary elements in G s.t. $aH=bH$. By Lemma 4 in Chapter 7 it follows that $b \in aH$. This means that there exists some element $h \in H$ s.t. $b = ah$. Consider $\det(b)=\det(ah)=\det(a)\det(h)=|det(a)|$. Hence, $\det(b)=|\det(a)|$. \\

\fbox{proof of the converse} Let $a,b$ be arbitrary elements in G s.t. $\det(b)=|\det(a)|$. Consider $a^{-1}b$, which is in G by properties of closure and inverses. Then, it follows that $\det(a^{-1}b)=\det(a^{-1})\det(b)=\det(b)/\det(a^{-1})=|1|$. By definition of H, it follows that $a^{-1}b \in H$, which, by a Lemma of cosets implies that $aH = bH$. Hence, $\det(b)=|\det(a)|$ iff $aH = bH$. QED.\\
\end{document}