\documentclass{article}
\usepackage[utf8]{inputenc}
\newcommand{\ii}{{\bf i}}
\newcommand{\jj}{{\bf j}}
\newcommand{\kk}{{\bf k}}
\newcommand{\id}{{\bf 1}}
\newcommand{\hur}{\frac{\id+\ii+\jj+\kk}{2}}%The "Hurwitz point"
\newcommand{\hurwitz}{\Z\left[\hur,\ii,\jj,\kk\right]}%The set of Hurwitz integers
\usepackage{wrapfig}
\usepackage[utf8]{inputenc}
\usepackage[dvips]{graphicx}
\usepackage{a4wide}
\usepackage{amsmath}
\usepackage{euscript}
\usepackage{amssymb}
\usepackage{amsthm}
\usepackage{amsopn}
\usepackage[colorinlistoftodos]{todonotes}
\usepackage{graphicx}
\usepackage[T1]{fontenc}
\newcommand\mybar{\kern1pt\rule[-\dp\strutbox]{.8pt}{\baselineskip}\kern1pt}

\usepackage{ulem}
\usepackage{xcolor}
\newcommand{\cs}[1]{\color{blue}{#1}\normalcolor}

%Matrix commands
\newcommand{\ba}{\begin{array}}
\newcommand{\ea}{\end{array}}
\newcommand{\bmat}{\left[\begin{array}}
\newcommand{\emat}{\end{array}\right]}
\newcommand{\bdet}{\left|\begin{array}}
\newcommand{\edet}{\end{array}\right|}

%Environment commands
\newcommand{\be}{\begin{enumerate}}
\newcommand{\ee}{\end{enumerate}}
\newcommand{\bi}{\begin{itemize}}
\newcommand{\ei}{\end{itemize}}
\newcommand{\bt}{\begin{thm}}
\newcommand{\et}{\end{thm}}
\newcommand{\bp}{\begin{proof}}
\newcommand{\ep}{\end{proof}}
\newcommand{\bprop}{\begin{prop}}
\newcommand{\eprop}{\end{prop}}
\newcommand{\bl}{\begin{lemma}}
\newcommand{\el}{\end{lemma}}
\newcommand{\bc}{\begin{cor}}
\newcommand{\ec}{\end{cor}}
\newcommand{\lcm}{\mbox{lcm}}
\newcommand{\defn}{\fbox{definition}}
\newcommand{\prop}{\fbox{proposition}}
\newcommand{\stab}{\mbox{stab}}
\newcommand{\Aut}{\mbox{Aut}}
\newcommand{\orb}{\mbox{orb}}

\newcommand{\norm}{\righttriangle}

\newcommand{\and}{\wedge}
\newcommand{\or}{\vee}



%sets of numbers
\newcommand{\N}{\mathbb{N}}
\newcommand{\Z}{\mathbb{Z}}
\newcommand{\Q}{\mathbb{Q}}
\newcommand{\R}{\mathbb{R}}

\title{Abstract Algebra}
\author{August, Evelyn}
\date{11/30/2021}

\begin{document}
\maketitle

\fbox{11: proposition} Given a ring $R$ and elements $a,b,c\in \R$ the following are true (in the last two, assume that $R$ is a ring with unity)
\begin{enumerate}
    \item[3] $(-a)(-b) = ab$
    \item[4] $a(b-c) = ab - ac$ and $(b-c)a = ba- ca$.
    \item[5] $(-1)a = -a$
    \item[6] $(-1)(-1) = 1$
\end{enumerate}\\

\fbox{proof} 
Let $R$ be a ring and let $a,b,c$ be arbitrary elements in $R$. For 5 and 6 let $R$  be a ring with unity.
\begin{enumerate}

    \item[3] We want to show that $(-a)(-b) = ab$. By property 2, $a(-b) = -(ab)$. Applying property 2 again, $(-a)(-b) = -(-(ab)).$ It remains to be shown that $-(-(ab)) = ab$. This is clearly the case, as $ab + (-ab) = ab - ab = 0$, so this follows from the definition of the additive inverse.\\
    
    \item[4] We want to show that $a(b-c) = ab - ac$. It follows that $a(b-c) = a(b + (-c))$ from the definition of subtraction notation. By property six of rings, this is just $ab + a(-c)$. Applying property 2, this is $ab + (-ac)$, which by the notational convention is just $ab - ac$. Likewise for $(b-c)a$.\\
    
    \item[5] We want to show that $(-1)a = -a$ ($1$ being the unity of $R$). To show this, consider $a + (-1)a$. By property 2, $(-1)a = -(1a)$. By definition of the multiplicative identity, $ 1a = a$. Substituting, we have $-(1a) = -a$. Substituting again, $a + (-1)a = a - (1a) = a- a$. By definition of the additive inverse, $ a - a = 0$. Hence $(-1)a$ is the additive inverse of $a$, which is $-a$.\\
    
    \item[6] We want to show that $(-1)(-1) = 1$, where $1$ is the unity of $R$. Consider $(-1)(-1)-1$. Then, it follows by properties of rings that $(-1)(-1)+(-1)\cdot1=(-1)(-1+1)=(-1)\cdot0=0$. Thus $(-1)(-1)$ is the additive inverse of $1$, which is $-1$. \\
    
\end{enumerate}\\


\fbox{27: proposition} The units of a [commutative] ring divides every element in the ring.\\

\fbox{proof}  Let $R$ be a ring and let $u\in U(R)\subseteq R$. By the properties of rings there exists some element $1\in R$ that acts as the multiplicative identity. Then by definition of the units, there exists some $u^{-1}\in R$ such that $uu^{-1} = 1$. Let $a\in R$ be arbitrary. By definition of the multiplicative identity it follows that $1a = a$. Substituting, $(uu^{-1})a = a$. By the associative law of multiplication in rings, $(uu^{-1})a = u(u^{-1}a) = a$. Since $u^{-1},a\in R$, it follows by the closure of multiplication in a ring that $q = u^{-1}a\in R$. Hence $a = uq$ for some $q\in R$, so by definition of divisibility $u|a$. Since $a$ is arbitrary in $R$ it follows that $u$ divides each element in $R$. Hence the units of a commutative ring divide every element in the ring. Q.E.D.\\

\newpage

\fbox{43: problem} Let $R=Z\oplus Z\oplus Z$ and $S=\{(a,b,c) \in R | a+b=c\}$. Prove or disprove that $S$ is a subring of $R$.\\

\fbox{solution} Consider the following elements in $R$: $x=(1,0,1)$ and $y=(0,1,1)$. We know that $x,y \in R$, since $1+0=1$ and $0+1=1$. Consider $xy=(1,0,1)(0,1,1)=(1 \cdot 0, 0 \cdot 1, 1 \cdot 1)=(0,0,1)$. However, $xy$ is not an element in $R$, since $0+0 \ne 1$. Thus, $S$ is not a subring of $R$.

\end{document}