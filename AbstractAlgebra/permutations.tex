\documentclass{article}
\newcommand{\ii}{{\bf i}}
\newcommand{\jj}{{\bf j}}
\newcommand{\kk}{{\bf k}}
\newcommand{\id}{{\bf 1}}
\newcommand{\hur}{\frac{\id+\ii+\jj+\kk}{2}}%The "Hurwitz point"
\newcommand{\hurwitz}{\Z\left[\hur,\ii,\jj,\kk\right]}%The set of Hurwitz integers
\usepackage[utf8]{inputenc}
\usepackage[dvips]{graphicx}
\usepackage{a4wide}
\usepackage{amsmath}
\usepackage{euscript}
\usepackage{amssymb}
\usepackage{amsthm}
\usepackage{amsopn}
\usepackage[colorinlistoftodos]{todonotes}
\usepackage{graphicx}
\usepackage[T1]{fontenc}
\newcommand\mybar{\kern1pt\rule[-\dp\strutbox]{.8pt}{\baselineskip}\kern1pt}

\usepackage{ulem}
\usepackage{xcolor}
\newcommand{\cs}[1]{\color{blue}{#1}\normalcolor}

%Matrix commands
\newcommand{\ba}{\begin{array}}
\newcommand{\ea}{\end{array}}
\newcommand{\bmat}{\left[\begin{array}}
\newcommand{\emat}{\end{array}\right]}
\newcommand{\bdet}{\left|\begin{array}}
\newcommand{\edet}{\end{array}\right|}

%Environment commands
\newcommand{\be}{\begin{enumerate}}
\newcommand{\ee}{\end{enumerate}}
\newcommand{\bi}{\begin{itemize}}
\newcommand{\ei}{\end{itemize}}
\newcommand{\bt}{\begin{thm}}
\newcommand{\et}{\end{thm}}
\newcommand{\bp}{\begin{proof}}
\newcommand{\ep}{\end{proof}}
\newcommand{\bprop}{\begin{prop}}
\newcommand{\eprop}{\end{prop}}
\newcommand{\bl}{\begin{lemma}}
\newcommand{\el}{\end{lemma}}
\newcommand{\bc}{\begin{cor}}
\newcommand{\ec}{\end{cor}}
\newcommand{\lcm}{\mbox{lcm}}

%sets of numbers
\newcommand{\N}{\mathbb{N}}
\newcommand{\Z}{\mathbb{Z}}
\newcommand{\Q}{\mathbb{Q}}
\newcommand{\R}{\mathbb{R}}
\title{Abstract Algebra}
\author{August, Evelyn}
\date{9/21/2021}

\begin{document}
\maketitle



\fbox{definition} A permutation of a set $A$ is a bijection from $A\rightarrow A$. \\

\fbox{remark} This is not interesting yet.\\

\fbox{notation} Function notation: $\sigma(1) = 1$, $\sigma(2) = 3$ et cetera. Then there is array notation: $\begin{array}{cccc}
     1 & 2 & 3 & 4 \\
     1 & 3 & 4 & 2 \\
\end{array}$, and then there is cylce notation $(1234)(1)$\\

\fbox{theorem} Every permutation can be written as a product of disjoint cycles.\\

\fbox{theorem} Disjoint cycles commute.\\

\fbox{theorem} the order of a single cycle permutation is just the length of the cycle.\\

\fbox{theorem} Given a permutation, the order is the lcm of all the disjoint factors.\\

\fbox{definition} The set of permutations on n-elements under function composition is a group called the symmetric group of degree $n$. It is denoted $S_n$. \\

\fbox{theorem} $|S_n| = n!$.\\

\fbox{theorem} (Cayley's theorem) Every group is isomporhpic to a group of permutations. If $G$ is a finite group, then subgroup isomorphic $S_n$ for osme number $n$. \\

\fbox{theorem} Every permutation can be written as a product of 2-cycles aka transpositions.\\

\fbox{theorem}(it's always even in $A_n$) Given a permutation $\sigma$, the number of two cycles in any 2-cycle representation of $\sigma$ must have the same parity. \\

\fbox{proof} Lemma: the identity element is a product of an even number of 2-cycles. Proof: lets not. Actual proof: Let $\sigma$ be an arbitrary element in $S_n$ and let $\sigma = \alpha_1\alpha_2\dots \alpha_k = \beta_1\dots\beta_l$. We not that $\sigma\sigma^{-1} = (\alpha_1\dots\alpha_k)(\beta_1\dots\beta_l)^-1$ by socks shes propety good stuff happens. So you get an even number. \\

\fbox{definition/theorem} The set of even permutations forms a subgroup of $S_n$ called the Alternating Group of degree $n$, denoted $A_n$. \\

\fbox{proof that $A_n \le S_n$}. Since $A_n$ is by definition a subgroup, and $|S_n| = n!$, we have that $A_n$ is a subset of a finite group. Then it suffices to show that $A_n$ is closed. The identity can be represented as the product of an even number of two cycles, hence $A_n$ is never empty. Let $\alpha \in A_n$ be an element of $A_n$ with the cyclic representation$ (a_11,a_12)(a21, a22)...(a_n1,an2)$ for some even number $n$. Likewise let $\beta \in A_n$ be represented as $\beta = (b_11,b_12)(b21, b22)...(b_m1,bm2)$. Then $\alpha\beta = (a_11,a_12)(a21, a22)...(a_n1,an2)(b_11,b_12)(b21, b22)...(b_m1,bm2)$ for some even number $m$. This representation will be composed of $m+n$ subgroups. Since the sum of two even numbers is even, we can represent $\alpha\beta$ as an even number of two cycles, hence $\alpha\beta$ is even. By definition of $A_n$ it follows that $\alpha\beta \in A_n$. Since $\alpha,\beta$ are both arbtirary elements in $A_n$, this applies to all elements in $A_n$ and $A_n$ is closed under its operation. Since $A_n$ is both nonempty and closed under the operation, by the finite subgroup test it follows that $A_n\le S_n$.\\

\fbox{theorem} $|A_n| = n!/2$. \\
\fbox{proof} Let $\alpha$ be an arbtirary element in $S_n$. If $\alpha$ is odd,  
\end{document}