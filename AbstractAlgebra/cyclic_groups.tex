\documentclass{article}
\usepackage[utf8]{inputenc}
\newcommand{\R}{\mathbb{R}}
\newcommand{\Z}{\mathbb{Z}}
\newcommand{\C}{\mathbb{C}}
\newcommand{\N}{\mathbb{N}}
\newcommand{\Q}{\mathbb{Q}}

\title{Abstract Algebra}
\author{ajbergquist }
\date{August 2021}

\begin{document}
\fbox{theorem}(classification of finite simple groups) There are only 5 categories of finite simple groups (1 of which is cyclic).\\
\fbox{remark} Wow, this is cool!\\
\fbox{1-step subgroups test} Suppose $H$ is a nonempty subset of $G$. If for all $a,b\in H$, $ab^{-1}\in H$, then $H$ is a subgroup.
\fbox{proof} will happen soon\\

\fbox{finite subgroup test} Gien a finite group $G$ and a nonempty subset $H\subseteq G$, if $H$ is closed under the operation of $G$, then $H\le G$. 
\fbox{proof} Let $n = |H|$ as in the number of elements in $H$.  \\

\fbox{proposition} Given subgroups $H,K\le G$ where $G$ is Abelian, $HK = \{hk |h\in H,k\in K\}\le G$.\\

\fbox{theorem1} Given a group $G$, the center of $G$, denoted $Z(G) = \{x\in G: xa = ax,\forall a\in G\}$.\\
\fbox{theorem2} Given an element $a\in G$, the centralizer of $a$, denoted $c(a) = \{y\in G : a = ay\}$ is a subgroup of $G$.
\fbox{poof 1} later\\


\fbox{definition/thm} Given a group $G$ and an element $a\in G$, the cyclic subgroup generated by $a$ is $<a> = \{a^n:n\in \Z\}.$\\
\fbox{proof that this is a subgroup} Let $x,y\in G$ be arbitrary elements. \\


\fbox{definition} A group $G$ is called a cyclic group iff there exists some $g\in G$ such that $G = <g>$. Furthermore, $g$ is called the generator of $G$. \\

\fbox{theorem} (classification of finite simple groups) All finite simple groups are cyclic, alternating, sporadic, Lie Type, or a Tits group. \\
\fbox{definition} A simple group is a group for which the only normal subgroups are just the set with the identity element in it. \\
\fbox{definitoin} Given a subtroup $H\le G$ iff for all elements in $G$, the set $gH = \{gh | h \in H\} = \{hg | h \in H\} = Hg.$\\

\fbox{theorem} (fundamental theorem of finite Abelian groups) Every finite Abelian gorup is the direct product of cyclic groups of prime power order. Moreover, the number of terms in the product and the orders of the components are uniquely determined by the group.\\

\fbox{theorem} Given a cyclic group $G$, if $|G| = n$ then $G\cong \Z_n$. If $|G| = \infity$ then $G\cong \Z$.\\

\fbox{theorem} Given a group $G$ and an element $g\in G$, the order of $g$, denoted $|g|$ is the smallest positive integer $n$ such that $g^n = e$.\\

\fbox{theorem} Given a gorup $G$ and an element $g$ of order $n$, the following statements are true:
\begin{enumerate}
    \item $g^i = g^j$ iff $n|(i-j).$
    \item $<g> = \{e, g, g^2, \dots, g^{n -1}\}.$
\end{enumerate}\\

\fbox{theorem} Given an element $g$ of order $n$ in a group $G$, $<g^k> - <g^>$
\end{document}

