\documentclass{article}
\newcommand{\ii}{{\bf i}}
\newcommand{\jj}{{\bf j}}
\newcommand{\kk}{{\bf k}}
\newcommand{\id}{{\bf 1}}
\newcommand{\hur}{\frac{\id+\ii+\jj+\kk}{2}}%The "Hurwitz point"
\newcommand{\hurwitz}{\Z\left[\hur,\ii,\jj,\kk\right]}%The set of Hurwitz integers
\usepackage[utf8]{inputenc}
\usepackage[dvips]{graphicx}
\usepackage{a4wide}
\usepackage{amsmath}
\usepackage{euscript}
\usepackage{amssymb}
\usepackage{amsthm}
\usepackage{amsopn}
\usepackage[colorinlistoftodos]{todonotes}
\usepackage{graphicx}
\usepackage[T1]{fontenc}
\newcommand\mybar{\kern1pt\rule[-\dp\strutbox]{.8pt}{\baselineskip}\kern1pt}

\usepackage{ulem}
\usepackage{xcolor}
\newcommand{\cs}[1]{\color{blue}{#1}\normalcolor}

%Matrix commands
\newcommand{\ba}{\begin{array}}
\newcommand{\ea}{\end{array}}
\newcommand{\bmat}{\left[\begin{array}}
\newcommand{\emat}{\end{array}\right]}
\newcommand{\bdet}{\left|\begin{array}}
\newcommand{\edet}{\end{array}\right|}

%Environment commands
\newcommand{\be}{\begin{enumerate}}
\newcommand{\ee}{\end{enumerate}}
\newcommand{\bi}{\begin{itemize}}
\newcommand{\ei}{\end{itemize}}
\newcommand{\bt}{\begin{thm}}
\newcommand{\et}{\end{thm}}
\newcommand{\bp}{\begin{proof}}
\newcommand{\ep}{\end{proof}}
\newcommand{\bprop}{\begin{prop}}
\newcommand{\eprop}{\end{prop}}
\newcommand{\bl}{\begin{lemma}}
\newcommand{\el}{\end{lemma}}
\newcommand{\bc}{\begin{cor}}
\newcommand{\ec}{\end{cor}}
\newcommand{\lcm}{\mbox{lcm}}
\newcommand{\defn}{\fbox{definition}}
\newcommand{\prop}{\fbox{proposition}}

%sets of numbers
\newcommand{\N}{\mathbb{N}}
\newcommand{\Z}{\mathbb{Z}}
\newcommand{\Q}{\mathbb{Q}}
\newcommand{\R}{\mathbb{R}}
\title{isomorphisms}


\begin{document}
\maketitle
Given $A,B$ sets with operations $*,+$ respectively, we say that the ordered pairs $(A,*)$ and $(B,+)$ are isomorphic if there exists some bijective function $\phi : A\rightarrow B$ such that for all $a,b\in A$ $\phi(a*b) = \phi(a+b)$\\

\fbox{theorem} ( Fundamental theorem of cyclic gorups ) \\
\begin{itemize}
    \item Every subgroup of a cyclic group is a cyclic group.
    \item Morevoer, if the order of $G$ is $n$, then the order of every subgroup of $G$ divides $n$. 
    \item And for each positive divisor of $K$ there exists exactly one subgroup of order $k$.
\end{itemize}\\

\defn (permutation) A permutation, $\sigma$ on a nonempty set $S$ is a bijection $\sigma: S\rightarrow S$. \\

\fbox{theorem} Every goup is isomorphic to a group of permuations.

\\
\fbox{proof} Let $G$ be a group. \\

For ever element in $a$, define $\sigma_a :G\rightarrow G$ by $\sigma_a(x) = ax$ for all $x\in G$. To show that $\sigma_a$ is well defined, we invoke the closure of $G$. We wish to show that $\sigma_a$ is a permutation. To do this, we merely need to show that $\sigma_a$ is a bijection. This follows from the fact that $\sigma_a$ is a map from $G\rightarrow G$, hence it must be bijective if it is injective. Let $x,y\in G$. By the left cancelation rule if $ax = ay$, hence $x = y$, hence $\sigma_a$ is bijective, and hence it is a permutation. Let $H = (\{\sigma_a : a\in G\}, *)$, where $*:G\rightarrow G$ is function composition. We need to show that $H$ is a group. Clearly it is nonempty, as $G$ is a group hence it cannot be empty, so there exists some $a\in G$, so we have $\sigma_a\in H$. Now to show that there is an identity, consider $\sigma_e$ for the identity $e\in G$. Let $\sigma_a$ be an arbtirary element in $G$. We can show that $\sigma_e$ is the identity by showing $\sigma_e$ doesn't do anything to $\sigma_a$. Next we prove they are isomorphic, but I can do this and I'm hungry/thirsty/tired.

Q.E.D.\\

\fbox{definition} Given a group $G$, and an element $a\in G$, the inner automorphism induced by $a$ is th automoprhism $\phi_a:G\rightarrow G$ defined $\phi_a(x) = ax$ for all $x\in G$. The set of all inner automorphisms is denoted $Inn(G)$\\

\fbox{theorem} Given a group $G$, the set $\Aut(G)$ with composition forms a group. Furthermore, $Inn(G)$ with function composition is a subgroup of $Aut(G).$
\end{document}