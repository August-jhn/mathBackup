\documentclass{article}
\usepackage[utf8]{inputenc}
\newcommand{\ii}{{\bf i}}
\newcommand{\jj}{{\bf j}}
\newcommand{\kk}{{\bf k}}
\newcommand{\id}{{\bf 1}}
\newcommand{\hur}{\frac{\id+\ii+\jj+\kk}{2}}%The "Hurwitz point"
\newcommand{\hurwitz}{\Z\left[\hur,\ii,\jj,\kk\right]}%The set of Hurwitz integers
\usepackage{wrapfig}
\usepackage[utf8]{inputenc}
\usepackage[dvips]{graphicx}
\usepackage{a4wide}
\usepackage{amsmath}
\usepackage{euscript}
\usepackage{amssymb}
\usepackage{amsthm}
\usepackage{amsopn}
\usepackage[colorinlistoftodos]{todonotes}
\usepackage{graphicx}
\usepackage[T1]{fontenc}
\newcommand\mybar{\kern1pt\rule[-\dp\strutbox]{.8pt}{\baselineskip}\kern1pt}

\usepackage{ulem}
\usepackage{xcolor}
\newcommand{\cs}[1]{\color{blue}{#1}\normalcolor}

%Matrix commands
\newcommand{\ba}{\begin{array}}
\newcommand{\ea}{\end{array}}
\newcommand{\bmat}{\left[\begin{array}}
\newcommand{\emat}{\end{array}\right]}
\newcommand{\bdet}{\left|\begin{array}}
\newcommand{\edet}{\end{array}\right|}

%Environment commands
\newcommand{\be}{\begin{enumerate}}
\newcommand{\ee}{\end{enumerate}}
\newcommand{\bi}{\begin{itemize}}
\newcommand{\ei}{\end{itemize}}
\newcommand{\bt}{\begin{thm}}
\newcommand{\et}{\end{thm}}
\newcommand{\bp}{\begin{proof}}
\newcommand{\ep}{\end{proof}}
\newcommand{\bprop}{\begin{prop}}
\newcommand{\eprop}{\end{prop}}
\newcommand{\bl}{\begin{lemma}}
\newcommand{\el}{\end{lemma}}
\newcommand{\bc}{\begin{cor}}
\newcommand{\ec}{\end{cor}}
\newcommand{\lcm}{\mbox{lcm}}
\newcommand{\defn}{\fbox{definition}}
\newcommand{\prop}{\fbox{proposition}}
\newcommand{\stab}{\mbox{stab}}
\newcommand{\Aut}{\mbox{Aut}}
\newcommand{\orb}{\mbox{orb}}

\newcommand{\norm}{\righttriangle}

\newcommand{\and}{\wedge}
\newcommand{\or}{\vee}



%sets of numbers
\newcommand{\N}{\mathbb{N}}
\newcommand{\Z}{\mathbb{Z}}
\newcommand{\Q}{\mathbb{Q}}
\newcommand{\R}{\mathbb{R}}

\title{Abstract Algebra}
\author{August, Evelyn}
\date{12/09/2021}

\begin{document}
\maketitle

\fbox{7: proposition} Let $R$ be a finite commutative ring with unity. Then for all $r\ne 0\in R$, $r$ is a unit or a zero divisor.\\

\fbox{proof} Suppose $r$ is not a zero divisor. Since $R$ is finite, it follows that the sequence ${r^n}_{n\in \Z}$ must be finite as well by closure. Hence $r^i = r^j$ for some integers $i,j$ such that $i\ne j $. Without loss of generality assume $i> j$. We know that $r^i \ne 0$ and $r^j \ne 0$, because if they were, by the supposition that $r$ is a not a zero divisor $rr^{i-1} = 0$ hence $r^{i-2}$ all the way to $r = 0$, contradicting the supposition that $r$ is nonzero. By properties of additive inverses in a Ring, we know that $r^j$ exists. Consider $r^{i}-r^j=0$. By properties of distribution it follows that $r^{i}-r^j=r^j(r^{i-j}-1)=0 $. Since $r$ is not a zero-divisor, we know that either $r^{i}$ or $(r^{j-i}-1)$ is equal to zero. We explained earlier that $r^{i} \ne 0$. Thus, $(r^{j-i}-1)$ must equal zero, which implies that $r^{j-i}=1$ or in other words $rr^{j-i-1}=1$, saying that $r^{j-i-1}$ is the inverse of $r$. Hence, $r$ is a unit. QED.\\

\fbox{35: proposition} Let $F$ be a field of order $2^n$. Prove that $charF=2$. \\

\fbox{proof} Let $F$ be a field with order $2^n$. Since F is a field, it is also an integral domain, an by Theorem 13.4 it follows that $charF$ is $0$ or $p$ for some prime $p$. Since fields are also rings with unity 1, it follows by Theorem 13.3 that the order of $1$ is equal to $charF$, so the order of $1$ is either $0$ or prime. By corollary 2 of Lagrange's Theorem, it follows that the order of each element in $F$ divides the order of $F$, since $|F|$ is finite. Thus, the order of $1$ needs to divide $2^n$, so it cannot be $0$, leaving us with the prime option. By the Fundamental Theorem of Arithmetic, up to rearrangement of the factors, the prime factorization of a natural number greater than 1 is unique. Hence, the only prime that divides $2^n$ is $2$, so the order of $1$ is $2$ and thus $charF=2$. QED.\\ 

\fbox{lemma} given a prime number $p$, each $p$th 

\fbox{lemma} (The generalized N$\emptyset\emptyset $B binomial theorem) \\
\fbox{proof}

\fbox{63: proposition} Let $F$ be a field with char$F=p$ for some prime $p$. Prove that $K=\{x \in F | x^p=x\}$ is a subfield in $F$.\\

\fbox{proof} Notice that by definition of unity and the additive identity, $0^p = 0$ and $1^p = 1$, hence $0,1\in K$ and there are at least two elements in $K$. We proceed then by the finite subfield test. Let $F$ and $K$ be defined as above. This proof is going to apply the subfield test, so we want to show that $a-b \in K$ and $ab^{-1} \in K$ for some $a,b \in K$. \\
Let $a,b$ be arbitrary elements in $K$. Consider $a-b = a+(-b) = a+(-1)b$. Since $a,b \in K$, it follows by definition of $K$ that $a-b=a^p+(-1)b^p$. Because $p$ is odd by restriction of it being prime other than 2 (if it's 2 then...), $(-1)^p = (-1)$, hence $a-b = a^p + (-1b)^p = a^p + (-b)^p$. By problem $49a$ in this chapter it follows that $a^p+(-b)^p=(a+(-b))^p = (a-b)^p$ and hence by definition of K we know that $a-b=(a-b)^p \in K$. If $\char F = 2$, then $(a-b)^2 = (a-b)(a-b) = a^2 -2\cdot ab +(-b)(-b) = a^2 + b^2$. But since char$F$=2, we know that $2a=a+a=0$, which means that $a=-a$ for all $a \in F$. It follows that $a^2+b^2=a^2-b^2$. \\
Furthermore, consider $ab^{-1}$. We want to show that $(ab^{-1})^p = ab^{-1}$, as this would imply that $ ab^{-1} \in K$. By the associative and commutative property of multiplication in a field, $(ab^{-1})^p = a^p(b^{-1})^p = a(b^{-1})^p$, as follows from the definition of $K$ and the fact that $a\in K$. It remains to be shown that $(b^{-1})^p = b^{-1}$. Clearly $(b^{-1})^p \in F$ by definition of a field and closure. Consider $(b^{-1})^pb$. Since by the associative and commutative properties of multiplication in a field, and by definition of the multiplicative inverse $b^p = b$, $(b^{-1})^pb = (b^{-1})^pb^p = (b^{-1})^pb^p = (b^{-1}b)^p = 1^p = 1$, where $1$ is unity in $F$. Hence by definition of the multiplicative inverse,  $(b^{-1})^p = b^{-1}$, so by substitution $(ab^{-1})^p = ab^{-1}$, hence by definition of $K$ $ab^{-1}\in K$.\\

Since $a$ and $b$ are arbitrary elements of $K$, it follows that for all $a,b\in K$, $a-b\in K$ and $ab^{-1}\in K$. So by the subfield field test it follows that $K$ is a subfield of $F$.
QED.\\ 

\fbox{3: proposition} Verify that $I=<a_1, ... a_n>=\{r_1a_1+...+r_na_n|r_i \in R\}$ for $a_1, ... a_n \in R$ and $R$ is a commutative ring with unity. \\

\fbox{proof} Let $x, y$ be arbitrary elements in $I$. Then $x=r_1a_1+...+r_na_n$ and $y=r_1'a_1+...+r_n'a_n$ for $a_i, r_i, r_i' \in R$. Clearly, $I$ is non-empty. Consider $x-y=r_1a_1+...+r_na_n-(r_1'a_1+...+r_n'a_n)=r_1a_1+...+r_na_n-r_1'a_1-...-r_n'a_n=r_1a_1-r_1'a_1+...+r_na_n-r_n'a_n=(r_1-r_1')a_1+...+(r_n-r_n')a_n$ (By the properties of a ring and theorem 12.1). But since $R$ is closed by properties of a ring, $r_i-r_i' \in R$ and thus $x-y \in I$. \\
Next, consider $rx$ for $r \in R, x \in I$. Then $rx=r(r_1a_1+...+r_na_n)=rr_1a_1+...+rr_na_n = (rr_1)a_1+...+(rr_n)a_n$ by the distributive property and associative property of multiplication in a ring. But since $R$ is closed under multiplication, $rr_i \in R$ and $rx = rr_1a_1+...+rr_na_n \in I$. Likewise for $xr$, as the commutativity of $R$ implies that $xr = rx$. Since $x$ and $y$ were arbitrary, it follows for all $x,y\in I$ and for all $r\in R$, $x-y \in I$ and $ar =ra \in R$. Hence by the Ideal Test it follows that $I$ is an ideal of $R$. QED. \\

\fbox{3: proposition II} If $J$ is any ideal of $R$ that contains $a_1, ... a_n$, then $I \subseteq J$. \\

\fbox{proof} Let $I$ and $J$ be defined as above. Let $x$ be an arbitrary element in $I$. Then $x=r_1a_1+...+r_na_n$ for $r_i, a_i \in R$, as follows by definition of $I$. Furthermore, since $r_1,\dots,r_n\in R$, it follows by definition of an ideal that, as $J$ is an ideal of $R$, $r_1a_1,\dots,r_na_n\in J$. Furthermore, since ideals are subrings, which are closed under the additive operation, it follows that $r_1a_1 + r_2a_2 \in J$. Hence by associativity $(r_1a_1 + r_2a_2) + r_3a_3 = r_1a_1+ r_2a_2 + r_3a_3 \in J$. Continuing on until each multiple is added in, we arrive at $x = r_1a_1 + \dots r_n a_n \in J$. Since $x$ is an arbitrary element of $I$, it follows that for all $x\in I$, $x\in J$ as well. Hence by definition of a subset, $I\subseteq J$.\\
Q.E.D.\\

\fbox{proposition : 29} In $\Z[x]$, let $I = \{f(x)\in \Z[x] : f(0) =0\}$. Then $I = <x>$.\\

\fbox{proof} First, we digest what this statement means. Note that $<x> = \{g(x)x: g(x)\in \Z[x]\}$. Both $I$ and $<x>$ are sets, hence this will be a set equality proof, as the operations are directly inherited from the ring which these two things are ideals of.\\
To prove that $<x>\subseteq I$, take some arbitrary $f(x)\in <x>$. By definition of $<x>$, $f(x) = g(x)x$ for some $g(x)\in \Z[x]$. By definition of $\Z[x]$, $g(x) = a_1 + a_2x + \dots a_n x^{n -1}$ for some $a_1, \dots a_n\in \Z$ and non-negative integer $n-1$. Substituting back in and applying the distributive and associative laws, and using the fact that polynomial multiplication is commutative, we find $f(x) = g(x)x = (a_ 1 + a_2 x\dots a_nx^{n-1})x = a_1 x + a_2 x^2 + \dots a_n x^n$. Now evaluating $f(0)$, we have, by the properties of a ring, $a_1(0) + \dots a_n(0)^n = 0.$ Hence by definition of $I$, $f(x) \in I$. Since $f(x)$ is an arbitrary element in $<x>$, it follows that for all $f(x)\in I$, $f(x)\in I$. Hence $<x>\subseteq I$.\\
To prove that $I\subseteq <x>$, let $f(x)\in I$ be arbitrary. Since $f(x)\in \Z[x]$ by definition of $I$, $f(x) = a_0 + a_1x + \dots a_n x^n$ for some $a_1,\dots, a_n \in \Z$ and some $n\in \N^0$, and since $f(0) = a_0 + a_1x + \dots a_n x^n = a_0 + a_1 0 + \dots a_n 0^n = a_0 + 0 = a_0$, we have that $a_0 = 0$. Hence by the distributive property of rings $f(x) = a_1x + \dots a_n x^n = x(a_1+ \dots + a_n x^{n-1})$. By definition of $\Z[x]$, $a_1 + \dots + a_n x^{n-1}\in \Z[x]$, hence by definition of $<x>$, $f(x) \in <x>$. Since $f(x)$ is arbitrary it follows that for all elements $f(x)\in I$, $f(x)\in <x>$ as well. Hence $I \subseteq <x>$.\\
Since $<x> \subseteq I$ and $I \subseteq <x>$, it follows by definition of set equality that $<x> = I$.\\
Q.E.D.\\

\fbox{problem : 39} Let $I = <x^2 + x + 2>$ be the principle ideal of $x^2 + x +2$ in $\Z_5[x]$. Find the multiplicative inverse of $2x + 3 + I$ in the factor ring $\Z_5[x]/I$. \\

\fbox{solution} First, note that the multiplicative identity is $1 + I$, not $I$. This can be seen easily, as $(1 + I)(f(x) + I) = 1f(x) + I = f(x) + I$ for all $f(x)\in \Z_5[x]$. Hence we are looking for an element in the factor ring, $f(x) + I \in \Z_5[x]/I$, represented by some polynomial $f(x)\in \Z_5[x]$, such that $(f(x) + I) (2x+ 3 + I) = 1 + I$.\\

Notice that, should such an element exist, it would follow by definition of the factor ring and by properties of cosets that, $(f(x) + I) (2x+ 3 + I) = f(x)(2x +3) + I = 1 + I$, hence $(f(x)) (2x+ 3 ) - 1 \in I$. By definition of $I$, this means that $ (f(x)) (2x+ 3) - 1  = (x^2 + x + 2)g(x)$ for some $g(x) \in \Z_5[x].$ Luckily, only a couple guesses lead to the observation that $(2x + 3)(3x + 1) = 6x^2 + 11x + 3 = x^2 + x + 3 = (x^2 + x + 2) + 1$. So the multiplicative inverse of $2x + 3 + I$ in $\Z_5[x]/I$ is $3x + 1 + I$. 

\end{document}