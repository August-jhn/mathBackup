\documentclass{article}
\usepackage[utf8]{inputenc}
\newcommand{\ii}{{\bf i}}
\newcommand{\jj}{{\bf j}}
\newcommand{\kk}{{\bf k}}
\newcommand{\id}{{\bf 1}}
\newcommand{\hur}{\frac{\id+\ii+\jj+\kk}{2}}%The "Hurwitz point"
\newcommand{\hurwitz}{\Z\left[\hur,\ii,\jj,\kk\right]}%The set of Hurwitz integers
\usepackage{wrapfig}
\usepackage{calligra}
\usepackage[utf8]{inputenc}
\usepackage[dvips]{graphicx}
\usepackage{a4wide}
\usepackage{amsmath}
\usepackage{euscript}
\usepackage{amssymb}
\usepackage{amsthm}
\usepackage{amsopn}
\usepackage[colorinlistoftodos]{todonotes}
\usepackage{graphicx}
\usepackage[T1]{fontenc}
\newcommand\mybar{\kern1pt\rule[-\dp\strutbox]{.8pt}{\baselineskip}\kern1pt}

\usepackage{ulem}
\usepackage{xcolor}
\newcommand{\cs}[1]{\color{blue}{#1}\normalcolor}

%Matrix commands
\newcommand{\ba}{\begin{array}}
\newcommand{\ea}{\end{array}}
\newcommand{\bmat}{\left[\begin{array}}
\newcommand{\emat}{\end{array}\right]}
\newcommand{\bdet}{\left|\begin{array}}
\newcommand{\edet}{\end{array}\right|}
\newcommand{\inv}[1]{#1^{-1}}

%Environment commands
\newcommand{\be}{\begin{enumerate}}
\newcommand{\ee}{\end{enumerate}}
\newcommand{\bi}{\begin{itemize}}
\newcommand{\ei}{\end{itemize}}
\newcommand{\bt}{\begin{thm}}
\newcommand{\et}{\end{thm}}
\newcommand{\bp}{\begin{proof}}
\newcommand{\ep}{\end{proof}}
\newcommand{\bprop}{\begin{prop}}
\newcommand{\eprop}{\end{prop}}
\newcommand{\bl}{\begin{lemma}}
\newcommand{\el}{\end{lemma}}
\newcommand{\bc}{\begin{cor}}
\newcommand{\ec}{\end{cor}}
\newcommand{\lcm}{\mbox{lcm}}
\newcommand{\defn}{\fbox{definition}}
\newcommand{\prop}{\fbox{proposition}}
\newcommand{\stab}{\mbox{stab}}
\newcommand{\Aut}{\mbox{Aut}}
\newcommand{\orb}{\mbox{orb}}

\newcommand{\norm}{\righttriangle}

\newcommand{\and}{\wedge}
\newcommand{\or}{\vee}



%sets of numbers
\newcommand{\N}{\mathbb{N}}
\newcommand{\Z}{\mathbb{Z}}
\newcommand{\Q}{\mathbb{Q}}
\newcommand{\R}{\mathbb{R}}

\title{Abstract Algebra III}
\author{August Bergquist}

\begin{document}

\maketitle


\fbox{problem 1} In a PID, every nontrivial prime ideal is maximal. \\

\fbox{proof} Let $D$ be a PID, and let $<a>$ be a prime ideal of $D$ generated by some $a\ne 0\in D$ (we are guaranteed that every ideal takes this form by definition of a PID, and the trivial ideal is $<0>$).\\



\fbox{proof} Let $D$ be a PID, and let $<a>$ be a prime ideal of $D$ generated by some $a\ne 0\in D$ (we are guaranteed that every ideal takes this form by definition of a PID, and the trivial ideal is $<0>$). Let $<b>$ be another ideal such that $<a>\subseteq <b> \subseteq D$.\\

Suppose that $<b>\ne <a>$. Then $<b>\not \subseteq <a>$. Furthermore, since $ <a>\subseteq <b> $, and since $a\in <a>$, $a\in <b>$ as well, hence there is some $n\in D$ such that $a = nb$. Since $b$ is not in $<a>$, or else $<b>\susbeteq <a>$, and since $nb= a\in <a>$ which is prime, it follows that $n\in <a>$. Since $n\in <a>$ it follows that $n = qa$ for some $q\in <a>$. Substituting, and since PIDs are commutative rings with unity, $nb = nqa = a = a1$. Since we're in an integral domain, left cancelation applies, so $nb = 1$. Hence $1\in <b>$, and by a previously proven theorem this means that $<b> = D$.\\

Hence by definition of a maximal ideal, $<a>$ is maximal.

\fbox{lemma} I believe we've proven this in class, but I can't remember when, nor do I have any reference of it in the textbook or my non-existent notes. I've got a nice little proof for it anyway, so why not present it here. Given a field $F$ and a polynomial $p(x) \in F[x]$, every element of $F[x]/<p(x)>$ can be represented by a polynomial $g(x)$ of degree less than $p(x)$ or zero.\\

\fbox{proof} This follows pretty much directly from the division algorithm.\\

\fbox{problem 3}
\begin{itemize}
    \item[a.] Prove that for any integers $m$ and $n$, the polynomial $p(x) = x^3 + (5m + 1)x + 5n + 1$ is irreducible over $\Q$\\
    
    \fbox{proof} Consider the prime number $5$, and reduce the coefficients of $p(x)$ modulo $5$ to get the polynomial $\overline{p}(x) = x^3 + x + 1\in \Z_5[x]$. Suppose by way of contradiction that $\overline{p}(x)$ is reducible over $\Z_5$. Since $\overline{p}(x)$ is a polynomial of degree 3 over $\Z_5$ which is a field, it follows by Theorem 17.1 that $\overline{p}(x)$ has a zero in $\Z_5$, call it $a$. Then we have 
    $\overline{p}(a) = (a)^3 + a + 1 = 0$. Equivalently, $a^3 + a = -1 = 4$. The cubes mod 5 are $1^3 = 1$, $2^3 = 8 = 3$, $3^3 = 27 = 2$, $4^3 = 64 = 4$, and of course, $5^3 = 0$. Adding the thing being cubed to each of these respectively we have $1^3 + 1 = 2 \ne 4$, $2^3 + 2 = 0 \ne 4$, $3^3 + 3 = 0\ne 4$, $4^3 + 4 = 8 = 3 \ne 4$, and $0^3 + 0 = 0\ne 4$. Then $a$ cannot be a zero of $\overline{p}(x)$, contradicting the assumption that it is, hence $\overline{p}(x)$ is irreducible over $\Z_5$. Since $\deg \overline{p}(x) = \deg p(x)$, it follows by the Mod p Irreducibility Test that $p(x)$ is irreducible over $\Q$.
    \item[b.] Construct a field of order $125$. \\
    \fbox{solution} We have already shown that $\overline{p}(x)$ (as defined in the proof) is irreducible over the field $\Z_5$. Hence by Corollary 1 of Theorem 17.5 it follows that $\Z_5[x]/<\overline{p}(x)> = \Z_5[x]<x^3 + x + 1>$ is a field. By the "lemma" we know that any elements of $\Z_5[x]/<\overline{p}(x)>$ can be represented by polynomials of degree less than $\overline{p}(x)$ or zero. (This is pretty much just be a direct result of the division algorithm in $F[x]$). Hence finding the order of $\Z_5[x]/<x^3 + x + 1>$ reduces to finding the number of polynomials of degree less than $3$ or zero. These are polynomials of the form $g(x) = ax^2 + bx + c$, for $a,b,c\in \Z_5$. Since there are five options for each, the number of elements in $\Z_5[x]/<x^3 + x + 1>$ is $5\cdot 5\cdot 5 = 125$. 
\end{itemize}

\newpage

\fbox{problem 2} Uf $R$ us a commutative ring and $I$ and $J$ are two proper ideals of $R = I + J$, then 
$$R/(I \cap J)\cong R/I \oplus R/J.$$

\fbox{proof} Consider the function $\phi:R\rightarrow R/I \oplus R/J$ defined $a \rightarrow (a + I, a+ J)$. We must show that this is well defined, and that it's operation preserving, and hence a homomorphism. \\

Suppose that $a = b$ for $a,b\in R$. Then $\phi(a) = (a + I, a + J)$, and by substitution, $(a + I, a + J) = (b + I, b + J)$, hence $\phi$ is well defined. (Do I need to show this, or is this annoyingly trivial?)\\

Now let $a$ and $b$ be arbitrary in $R$. Then $\phi(a + b) = (a + b + I, a + b + J) = ((a + I) + (b + I), (a + J) + (b+ J)) = (a + I, a + J) + (b + I, b + J) = \phi(a) + \phi(b)$. Similarly, 
$\phi(ab) = (ab + I, ab + J) = ((a + I)(b + I), (a + J)(b+ J)) = (a + I, a + J)(b + I, b + J) = \phi(a)\phi(b)$. Since $a$ and $b$ were arbitrary, it follows that $\phi$ is operation preserving.\\

Now to show that $\phi$ is surjective, let $v = (a + I, b + J)$ be an arbitrary element in $R/I \oplus R/J$, where $a$ and $b$ are the coset representatives of $a + I$ and $b + I$ in $R$. Since $R = I + J$, it follows that there exists $x,z\in I$ and $y,w\in J$ such that $x + y = a$ and $z + w = b$. Then by properties of cosets it follows that $v = (a + I,b + J) =(x + y + I,z + w + J)= (y + I, z + b)$. Now consider the element $y + z\in R$. Then $\phi(y + z) = (y + z + I, y + z + J)$, and since $z\in I$ and $y \in J$, by properties of cosets we have $\phi(y + z) = (y + I, z + J) = v$, so there is an element in $R$ that maps to $v$ under $\phi$. Since $v$ is arbitrary, it follows that all elements in the codomain $R/I\oplus R/J$ are mapped to by some element in the domain $R$. Hence $\phi$ is surjective.\\

Now to show that $\ker\phi = I\cap J$ First we must show that $I\cap J \susbeteq \ker\phi$. To do so, let $x$ be arbitrary in $I\cap J$. Then by definitino of intersection it follows that $x\in I$ and $x\in J$ Then by properties of cosets $\phi(x) = (x + I, x + J) = (I, J)$, hence $x\in \ker\phi$ by definition of the kernel. Sine $x$ was arbitrary in $I\cap J$, it follows that $I\cap J\subseteq \ker\phi$.\\

It remains to be shown that $\ker\phi\subseteq I\cap J$. To show this, let $a$ be an arbitrary element of $\ker\phi$. Then by definition of the kernel, we have $\phi(a) = (a + I, a + J) = (I, J)$, hence $a$ gets absorbed by both $I$ and $J$. By properties of cosets this means that $a\in I$ and $ a\in J$. Hence by definition of the intersection $a\in I\cap J$. Sine $a$ was arbitrary in $\ker\phi$, it follows that $\ker\phi \subseteq I\cap J$.\\

Since $\ker\phi\susbeteq I\cap J$ and $I\cap J\subseteq \ker\phi$, it follows that $\ker\phi = I\cap J$.\\

Since $\ker\phi = I \cap J$, and since $\phi$ is a homomorphism from $R$ to $R/I\oplus R/J$, it follows by the first isomorphism theorem for rings that $R/I\cap J \cong R/I \oplus R/J$. Q.E.D.

\newpage

\fboox{problem 4} Prove that if $V$ is an $F$-vector space and $S$ is a subspace of $V$, then the quotient space $V/S$ is an $F$-vector space.\\

\fbox{proof} We will need to show that all requirements of the definition of an $F$ vector space are met. Let $u + S$ and $v + S$ be elements of $V/S$ represented by $u,v\in V$, and let $a,b\in F$.
\begin{enumerate}
    \item First we will need to show that $(V/S,+)$ forms an abelian group. Since $V$ is an abelian group, and since subspaces are actually subgroups under the same operatino, and also since every subgroup of an abelian group is normal, $V/S$ can be viewed as the factor group under the operation $(v + S)+ (u + S) = v + u + S$ (see Theorem 9.2).
    \item Now consider $a(v + S)$. As the operation of scalar multiplication is defined on the quotient space, $a(v + S) = av + S$. Furthermore, since $v\in V$, and $V$ is an $F$-vector space, it follows from the properties of vector spaces that $av\in V$. Hence $av + S\in V/S$ by definition of the quotient space.
    \item Now consider $(a + b)(v+S)$. As we've defined scalar multiplication and vector addition in $V/S$, and since $V$ is an $F$-vector space, we have $(a+b)(v + S) = (a+ b)v + S = av + bv + S = (av + S) + (bv + S) = a(v + S) + b( + S).$ Hence scalar multiplication distributes over scalar addition.
    \item Now consider $a[(u + S)+(v+S)]$. By definition of addition as defined by in $V/S$, and since $V$ is an $F$-vector space $a[(u + S)+(v+S)] = a[u + v + S] = a(u + v) + S = (au + S) + (av + S) = a(u + S) + a(v + S)$. So scalar multiplication distributes over vector addition. 
    \item Now consider $(ab)(v + S)$. As we've defined scalar addition, and since $V$ is an $F$-vector space, $(ab)(v + S) = (ab)v + S = a(bv) + S = a(bv + S)$. So scalar multiplication is associative.
    \item Finally, consider $1(v + S)$, where $1$ is unity in $F$. Since $V$ is an $F$-vector space, and by our definitino of scalar multiplication, this is just $1(v + S) = 1v + S = v + S$. 
\end{enumerate}

Having shown that all of the necessary requirements are met, it follows that $V/S$ is an $F$-vector space. Q.E.D.

\end{document}