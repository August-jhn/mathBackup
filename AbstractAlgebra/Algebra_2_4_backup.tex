\documentclass{article}
\usepackage[utf8]{inputenc}
\newcommand{\ii}{{\bf i}}
\newcommand{\jj}{{\bf j}}
\newcommand{\kk}{{\bf k}}
\newcommand{\id}{{\bf 1}}
\newcommand{\hur}{\frac{\id+\ii+\jj+\kk}{2}}%The "Hurwitz point"
\newcommand{\hurwitz}{\Z\left[\hur,\ii,\jj,\kk\right]}%The set of Hurwitz integers
\usepackage{wrapfig}
\usepackage{calligra}
\usepackage[utf8]{inputenc}
\usepackage[dvips]{graphicx}
\usepackage{a4wide}
\usepackage{amsmath}
\usepackage{euscript}
\usepackage{amssymb}
\usepackage{amsthm}
\usepackage{amsopn}
\usepackage[colorinlistoftodos]{todonotes}
\usepackage{graphicx}
\usepackage[T1]{fontenc}
\newcommand\mybar{\kern1pt\rule[-\dp\strutbox]{.8pt}{\baselineskip}\kern1pt}

\usepackage{ulem}
\usepackage{xcolor}
\newcommand{\cs}[1]{\color{blue}{#1}\normalcolor}

%Matrix commands
\newcommand{\ba}{\begin{array}}
\newcommand{\ea}{\end{array}}
\newcommand{\bmat}{\left[\begin{array}}
\newcommand{\emat}{\end{array}\right]}
\newcommand{\bdet}{\left|\begin{array}}
\newcommand{\edet}{\end{array}\right|}
\newcommand{\inv}[1]{#1^{-1}}

%Environment commands
\newcommand{\be}{\begin{enumerate}}
\newcommand{\ee}{\end{enumerate}}
\newcommand{\bi}{\begin{itemize}}
\newcommand{\ei}{\end{itemize}}
\newcommand{\bt}{\begin{thm}}
\newcommand{\et}{\end{thm}}
\newcommand{\bp}{\begin{proof}}
\newcommand{\ep}{\end{proof}}
\newcommand{\bprop}{\begin{prop}}
\newcommand{\eprop}{\end{prop}}
\newcommand{\bl}{\begin{lemma}}
\newcommand{\el}{\end{lemma}}
\newcommand{\bc}{\begin{cor}}
\newcommand{\ec}{\end{cor}}
\newcommand{\lcm}{\mbox{lcm}}
\newcommand{\defn}{\fbox{definition}}
\newcommand{\prop}{\fbox{proposition}}
\newcommand{\stab}{\mbox{stab}}
\newcommand{\Aut}{\mbox{Aut}}
\newcommand{\orb}{\mbox{orb}}

\newcommand{\norm}{\righttriangle}

\newcommand{\and}{\wedge}
\newcommand{\or}{\vee}



%sets of numbers
\newcommand{\N}{\mathbb{N}}
\newcommand{\Z}{\mathbb{Z}}
\newcommand{\Q}{\mathbb{Q}}
\newcommand{\R}{\mathbb{R}}

\title{Abstract Algebra III}
\author{August J. Bergquist and Alexandra N. Walker}

\begin{document}

\maketitle

\fbox{definition} The given an $F$-vector space $V$ and a subspace $U$ of $V$, the \textbf{quotient space} $V/U$ is the set $V/U = \{v + U: v\in V\}$, where $v+U$ denotes the set $v+ U = \{v + u: u\in U\}$ for all $v\in V$, and is called the \textbf{left coset generated by} $v$. Scalar multiplication and vector addition on $V/U$ is defined as follows for all $v,w\in V$ and for all $a\in F$:

\begin{itemize}
    \item $(v+ U)+(w+U) = (v+w) + U$
    \item $a(v+ U) = av + U$.
\end{itemize}

\fbox{definition} Given vector spaces $V$ and $W$, a linear transformation $\phi:V \rightarrow W$\textbf{vector space isomorphism} if and only if $\phi$ is bijective. In this case, we say that $V$ is isomorphic to $W$, and write $V\cong W$.\\

%\fbox{lemma} Given $F$-vector spaces $V$ and $W$, and a linear transformation $\phi:V\rightarrow W$, $\phi$ is a linear isomorphism if and only if $\ker\phi = \{0\}$. In other words, the only element in $V$ which $\phi$ maps to the identity element in the abelian group $W$ is the identity element in the abelian group $V$.\\


\fbox{problem 1: First isomorphism theorem for vector spaces}\\

Given $F$-vector spaces $W$ and $V$, and a linear transformation $\phi: V\rightarrow W$, $V/\ker\phi\cong \phi(V)$.\\

\fbox{proof} We need to find a vector space isomorphism between $V/\ker\phi$ and $\phi(V)$. Consider the function $\psi : V/\ker\phi \rightarrow \phi(V)$ defined $\psi(v + \ker\phi) = \phi(v)$ for all $v+ \ker\phi\in V/\ker\phi$. We will have to do prove three thigns about $\psi$: (1) that $\psi $ is well-defined, (2) that $\psi$ is a linear transformation, and (3) that $\psi$ is bijective.
\begin{enumerate}
    \item Suppose that $v+ \ker\phi = w + \ker\phi$. In order to show that the images of $v+ \ker\phi$ and $w+ \ker\phi$ are the same, we will need to show that $\phi(v) = \phi(w)$. Since $v + \ker\phi = w + \ker\phi$, it follows that $(v + \ker\phi) - (w + \ker\phi) = (v- w) + \ker\phi =  (w + \ker\phi) - (w + \ker\phi) = (w- w) + \ker\phi = 0+ \ker\phi$. Hence $ w - v $ gets absorbed by $\ker\phi$ and $w- v\in \ker\phi$. So by our definition of $\psi$ (well or not), our construction $\phi$ as a homomorphism, and by definition of a kernel, it follows that $\psi((v- w) + \ker\phi) = \phi(w-v) = 0 =  \phi(w) - \phi(v)$. Hence $\phi(w) - \phi(v)  + \phi(v) = = \phi(w) + 0 = \phi(w) =  0 + \phi(v)  = \phi(v)$ as follows by the properties of abelian groups. Having established that $\phi(v) = \phi(w)$, it follows by definition of $\psi$ that $\psi(v + \ker\phi) = \psi(w +\ker\phi)$. Hence $\psi$ is well defined.
    \item Now we must show that $\psi$ is a linear transformation. Let $u + \ker\phi,v + \ker\phi\in V/\ker\phi$ and $a\in F$ be arbitrary in their respective sets. Then
    $$\psi\big((u + \ker\phi) + (v + \ker\phi)\big) = \psi\big( (u + v) + \ker\phi \big)$$this is boring...
\end{enumerate}

\newpage

\fbox{Theorem} Given an $F$ vector space $V$ and a subspace $S$ of $V$, $\dim(V/S) = \dim V - \dim S$.\\
% We should use the comments rather than use colored writing for the edits, just a thought
\fbox{proof} For no good reason at all, let $Q = V/S$. Let $\dim V = m$ and let $\dim S = n$. Let $k = m - n$Then by definition of dimension, and by a theorem from linear algebra, there exists bases $\mathcal{B}_S = \{s_1,\dots,s_n\}$ and $\mathcal{B}_V = \{s_1,\dots, s_n, v_1,\dots, v_k\}$ for vectors $s_1,\dots, s_n\in S$ and $v_1,\dots, v_k \in V$. Consider the set $\mathcal{B}_Q = \{v_1 + S,\dots, v_k + S\}$. Notice that $\mathcal{B}_Q$ contains $k = \dim V - \dim S$, hence the proof that $\dim(V/S) = \dim V - \dim S$ amounts to proving that $\mathcal{B}_Q$ is a basis for $V/S$. This in turn amounts to two things in showing that $\mathcal{B}_Q$ fills the requirements of the definition of a basis: (1) showing that $\mathcal{B}_Q$ is linearly independent over $F$, and (2) showing that each element of $V/S$ can be written as a linear combination of elements in $\mathcal{B}_Q$.
\begin{enumerate}
    \item To show that $\mathcal{B}_Q$ is linearly independent over $F$, suppose by way of contradiction that 
    $$a_1(v_1 + S) + \dots a_k(v_k + S) = 0 + S$$ for elements $a_1,\dots, a_k\in F$ where one some $a_j$ is non-zero. Since the numbering is arbitrary, without loss of generality let $a_1 $ be non-zero. Then by the operations defined on the quotient space it follows that 
    $$0 + S = (a_1v_1 + \dots + a_k v_k) + S.$$
    In other words, the linear combination $a_1v_1 + \dots + a_kv_k$ gets absorbed by $S$, hence $v\equiv a_1v_1 + \dots +a_k v_k\in S$. Since $\mathcal{B}_S$ is a basis for $S$, it follows by definition of a basis that $v$ can be written as a linear combination of elements in $\mathcal{B}_S$. In other words, 
    $$v = a_1 v_1 + \dots + a_k v_k = b_1 s_1 + \dots b_n s_n.$$ Since $a_1$ is a non-zero element of a field, and by properties of vector spaces, we have 
    $$
    \begin{array}{cc}
         & \inv{a_1}(a_1v_1 + \dots + a_k v_k) = 1_Fv_1 + \dots (\inv{a_1}a_k) \\
         & = v_1 + \dots + a'_k = \inv{a_1}(b_1s_1 + \dots + b_n s_n)\\
         & = \inv{a_1}b_1 s_1 + \dots + \inv{a_1}b_n s_n = b_1' s_1 + \dots + b_n' s_n
    \end{array} 
    $$
    where $x'$ denotes $\inv{a_1}x$ for all $x\in F$. "Subtracting" $a'_2 v_2 + \dots a'_k v_k$ on both sides of the equation we find
    $$ v_1 = b'_1s_1 + \dots  + b'_n s_n + a'_2v_2 + \dots + a'_k v_k.$$
    This contradicts the assumption that $\mathcal{B}_V$ is a basis, since if it were, none of its elements (including $v_1$) could be written as a linear combination of the others. Hence $\mathcal{B}_Q$ is linearly independent over $F$.
    \item Now let $v + S$ be an arbitrary element of $V/S$ for some $v\in V$. Since $\mathcal{B}_V$ is a basis for $v$, it follows that there exists elements $a_1, \dots , a_m\in F$ such that $v = a_1v_1 + \dots + a_k v_k + a_{k+ 1}s_1 + \dots + a_m s_n$. Let $x = a_1v_1 + \dots + a_k v_k$ and $y = a_{k+ 1}s_1 + \dots + a_m s_n$. Substituting, we have $v = x + y$. Notice that since $y$ is a linear combination of the elements $\mathcal{B}_S$, it follows that $y\in S$. Substituting this into the coset and applying the properties of addition and scalar multiplication as defined in the quotient space, we find $$
    \begin{array}{cc}
         & v + S = x + y + S = (x + S) + (y + S) = (x + S) + (0 + S) = x + S \\
         & = a_1v_1 + \dots + a_k v_k + S = (a_1 v_1 + S) + \dots + (a_k v_k + S) \\
         & = a_1(v_1 + S) + \dots + a_k(v_k + S),
    \end{array}$$ a linear combination of the elements of $\mathcal{B}_Q$! Since $v + S$ was arbitrary in $V/S$, it follows that every element of $V/S$ is a linear combination of elements in $\mathcal{B}_Q$. 
\end{enumerate}
Having shown that $\mathcal{B}_Q$ is linearly independent over $F$ and that each element of $V/S$ is a linear combination of elements in $\mathcal{B}_Q$, it follows by definition of a basis that $\mathcal{B}_Q$ is a basis for the vector space $V/S$. Since $V/S$ has a basis of $\dim V - \dim S$ elements, it follows by definition of dimension that 
$$ \dim (V/S) = \dim V - \dim S.$$ Q.E.D.

\end{document}
