\documentclass{article}
\newcommand{\ii}{{\bf i}}
\newcommand{\jj}{{\bf j}}
\newcommand{\kk}{{\bf k}}
\newcommand{\id}{{\bf 1}}
\newcommand{\hur}{\frac{\id+\ii+\jj+\kk}{2}}%The "Hurwitz point"
\newcommand{\hurwitz}{\Z\left[\hur,\ii,\jj,\kk\right]}%The set of Hurwitz integers
\usepackage[utf8]{inputenc}
\usepackage[dvips]{graphicx}
\usepackage{a4wide}
\usepackage{amsmath}
\usepackage{euscript}
\usepackage{amssymb}
\usepackage{amsthm}
\usepackage{amsopn}
\usepackage[colorinlistoftodos]{todonotes}
\usepackage{graphicx}
\usepackage[T1]{fontenc}
\newcommand\mybar{\kern1pt\rule[-\dp\strutbox]{.8pt}{\baselineskip}\kern1pt}

\usepackage{ulem}
\usepackage{xcolor}
\newcommand{\cs}[1]{\color{blue}{#1}\normalcolor}

%Matrix commands
\newcommand{\ba}{\begin{array}}
\newcommand{\ea}{\end{array}}
\newcommand{\bmat}{\left[\begin{array}}
\newcommand{\emat}{\end{array}\right]}
\newcommand{\bdet}{\left|\begin{array}}
\newcommand{\edet}{\end{array}\right|}

%Environment commands
\newcommand{\be}{\begin{enumerate}}
\newcommand{\ee}{\end{enumerate}}
\newcommand{\bi}{\begin{itemize}}
\newcommand{\ei}{\end{itemize}}
\newcommand{\bt}{\begin{thm}}
\newcommand{\et}{\end{thm}}
\newcommand{\bp}{\begin{proof}}
\newcommand{\ep}{\end{proof}}
\newcommand{\bprop}{\begin{prop}}
\newcommand{\eprop}{\end{prop}}
\newcommand{\bl}{\begin{lemma}}
\newcommand{\el}{\end{lemma}}
\newcommand{\bc}{\begin{cor}}
\newcommand{\ec}{\end{cor}}

%sets of numbers
\newcommand{\N}{\mathbb{N}}
\newcommand{\Z}{\mathbb{Z}}
\newcommand{\Q}{\mathbb{Q}}
\newcommand{\R}{\mathbb{R}}
\title{Abstract Algebra}
\author{August, Evelyn}
\date{9/14/2021}
\maketitle
\begin{document}
\fbox{Ch2, 37} Let $G$ be a finite group. Then the number of elements $x$ such that $x^3 = e$ is odd, and the number of elements $y$ such that $y^2 \ne e$ is even.\\

\fbox{proof} Let $G$ be an arbitrary group and let $|G|$ be the order of $G$. 
\begin{itemize}
    \item Let $S$ be the set of all elements $x$ in $G$ such that $x^3 = e$. Clearly $e\in S$, as by definition $e^3 = e$. This proof will first show that each element besides $e$ has a distinct inverse in $S$. Adding $e$ into the mix would show that $|S| = 2n + 1$ for some non-negative integer $n$, hence $n$ would be odd. To show this, we will break this into two steps.
    \begin{itemize}
        \item First we shall show that for each element in $x\in S$, $x^{-1}\in S$. By the group axiom of associativity, it follows that $x^3 = x(x^2) = (x^2)x= e.$ By definition of inverses, $x^{-1} = x^2$. Furthermore, observe that $(x^{-1})^3 = (x^2)^3 = (x^3)^2 = e^2 = e.$\\
        Hence $x^{-1}\in S$.\\
        \item Now we shall show that for all $x\in S$ such that $x \ne e$, $x^{-1} \ne x$. We shall proceed by contraction. Suppose by way of contradiction that $x \ne e$ and $x = x^{-1}.$ Then by supposition $e = x^{-1}x = x^2$. Operating on both sides, we have $ex = x = x^3 = e.$ But by supposition, $x \ne e$, hence a contradiction. Thus it follows by way of contradiction that whenever an element $x\in S$ is not $e$, then $x$ is distinct from $x^{-1}$.
    \end{itemize}
    Having shown that for every $x$ in $S$ not equal to $e$, there exists another element $x^{-1}$ in $S$, it follows that $|S- \{e\}| = 2n$ for some non-negative integer $n$. Adding in $e$, $|S| = 2n +1$, hence by definition of $S$ the number of elements $x\in G$ such that $x^3 = e$ is odd. Q.E.D.
    
    
    \item Let $S$ be a subset of $G$ such that each element $x\in S$ has the property that $x^2 \ne e$. Note that $S$ could be empty, in which case the proposition holds. Suppose then that $S$ is not empty. Let $x$ be an arbitrary element of $S$. First we shall show that each element $x\in S$, $x^{-1}\in S$. Then we shall show that if each $x\in S$ is distinct from its inverse.
    \begin{itemize}
        \item Let $x$ be an arbitrary element in $S$. By the inverse property of groups, $x^{-1} \in G$. By the associative property of groups, it is clear that $(x^{-1})^2x^2 = x^{-1}x^{-1}xx = x^{-1}(x^{-1}x = e)x = e$, hence $(x^{-1})^2 = (x^2)^{-1}$. Furthermore, since $x^2 \ne e$, it follows that $(x^{-1})^2x^2 \ne (x^{-1})^2 e$, hence $e\ne (x^{-1})^2$. By definition of $S$, it follows that $x^{-1}\in S$.
        \item  Once again, let $x$ be an arbitrary element in $S$. Now to show that $x^{-1}$ is distinct from $x$, suppose that the contrary is true. Then we have $x = x^{-1}$. It would follows by the associative and inverse properties of groups that $x^2 = xx^{-1} = e$, contradicting the supposition that $x\in S$. Hence it follows that for all $x$ in $S$, the inverse of $x$ is distinct from $x$.
    \end{itemize} By the uniqueness property of inverses in a group, it follows that for every element in $S$ there is exactly one other element in $S$ (it's inverse). In other words, $|S| = 2n$ for some nonzero integer $n$, so there is an even number of elements $x$ in $G$ such that $x^2 \ne e$. Q.E.D.
\end{itemize}
\newpage

\fbox{Ch 3, 31} For each divisor k>1 of n, let $U_k(n)=\{ x\in U(n)|x \mod k=1\}$.\\
a) List the elements of $U_4(20), U_5(20), U_5(30),$ and $U_{10}(30)$.\\
b) Prove that $U_k(n)$ is a subgroup of $U(n)$.\\ 
c) Let $H=\{x\in U(10)|x mod 3=1\}$. Is H a subgroup of U(10)?\\

a)
$$U(n)=\{[x]_n : gcd(x, n)=1\}$$
$$U(20)=\{1, 3, 7, 9, 11, 13, 17, 19\}$$
$$U(30)=\{1, 7, 11, 13, 17, 19, 23, 29\}$$
$$U_4(20)=\{1, 9, 13, 17\}$$
$$U_5(20)=\{1, 11\}$$
$$U_5(30)=\{1, 11\}$$
$$U_{10}(30)=\{1, 11\}$$
\\

b) $$U_k(n)\leq U(n)$$ (finite subgroup test)\\
Let $n$ be an arbitrary positive integer greater than $1$.
Let $k>1$ be an arbitrary divisor of $n.$\\
We know that $U(n)$ is a finite group, which implies that the subset $U_k(n)$ is also finite. 
Furthermore, the identity element, 1, is an element of U(n), since $\gcd(k, 1)=1$ for all $1<k \in \mathbb{N}$. Furthermore, 1 is also an element of $U_k(n)$, since $1 \in U(n)$ and $1 \mod k=1$ for all $k \in \mathbb{N}$. Thus, the subset is non-empty.\\

Let a and b be arbitrary elements of $U_k(n)$. We know, by definition, that $a mod k = 1$ and $b \mod k = 1$. Since $[a] \cdot [b] = [a \cdot b]$, we know that $ab \mod k = 1$. Thus, $ab \in U_k(n)$ for all $a,b\in U_k(n)$. \\

c)
$$U(10)=\{1, 3, 7, 9\}$$
$$H = U_k(10)=\{1, 7\}$$
Yes, H is a subgroup of U(10). This can be seen by showing that $[7][7] = [1][1] = [1]$. Also we can use the previously proven result.\\

\fbox{Ch 3, 32: proposition} If $G$ is a group, and $H$ and $K$ are subgroups of $G$, then it follows that $H\cap K$ is a subgroup of $G$.\\

\fbox{proof} Let $G$ be an arbitrary group, and let $H$ and $K$ be arbitrary subgroups of $G$. First we must show that $H\cap K$ is nonempty. By definition of a subgroup, $H$ and $K$ must share the identity element. Hence $H\cap K\ne \emptyset$. Let $a$ be an arbitrary element of $H\cap K$. By intersection, it follows that $a\in H$ and $a\in K$. Furthermore, by the group axioms it follows that there is an inverse, $a^{-1}$ in both $H$ and $K$. Hence the inverse property is satisfied.\\
Now let $a,b\in H\cap K$ be arbitrary elements. By intersection it follows that $a,b\in H$ and $a,b\in K$. Furthermore, since $H$ and $K$ are subgroups, by the group axioms it follows that $ab \in H$ and $ab\in K$. By intersection, it follows that $ab\in H\cap K$. Since $a$ and $b$ are arbitrary, it follows that this works for all elements in $H\cap K$, hence $H\cap K$ is closed under the group operation of $G$. By the two step subgroup test it follows that $H\cap K$ is a subgroup of $G$. Q.E.D.\\
\newpage

\fbox{proposition} Given any number of subgroups of $G$, the intersection of all of these subgroups is also a subgroup.\\
\fbox{proof}\\
Let $H$ and $K$ be subgroups of $G$ for some group $G$. By the previously proven result, $H\cap K \le G$. We shall proceed by induction, so let this be the base case.\\
Now for the induction hypothesis, suppose that for some $n  \in \N$, 
$$H = \bigcap_{i\le n}H_i: \mbox{for } H_i\le G$$ such that $H\le G$. For the induction step, we have for subgroups of $G$, $H_1,\dots, H_{n+1}$, $$\bigcap_{i\le n+1}H_i = H \cap H_{n+1}.$$ By the previously proven result, this is a subgroup of $G$. Hence by way of induction, it follows that the intersection of any collection of subgroups in a group is also a subgroup. Q.E.D. (Induction may have been overkill). \\



\fbox{Ch3, 68: proposition} Let $H = \{A \in GL(2,\R) : \det A = 2^p,\exists  p\in \Z\ ;\mbox{for } m,n = 1,2; \}$. Then $H\le GL(2,\R)$.\\
\fbox{proof} Consider the identity matrix $I_4$. Clearly $\det I_4 = 1 = 2^0$, and $0\in \Z$. Hence by definition of $H$, $I_4\in H$ and $H$ is nonempty. Let $A$ be an arbitrary element in $H$. By definition of $H$, since $0$ is not an integer power of $2$, the determinate of $A$ cannot be zero, hence by the results of linear algebra there must exist some $A^{-1}$ in $GL(2,\R)$. Furthermore, by definition of $H$, there must exist some integer $n\in \Z$ such that $\det A = 2^p$. Furthermore, by the results of linear algebra, $\det A^{-1} = 1/2^p = 2^{-p}.$ Since $\Z$ is a group, $-p \in \Z$ :). By definition of $H$, it follows that $A^{-1}\in H$.\\
Now let $A$ and $B$ be arbitrary elements in $H$. By definition of $H$, we know $\det A = 2^m$ and $\det B = 2^n$ for some $m,n \in \Z$. Applying the results of linear algebra, $\det(AB) = \det A \det B = 2^m 2^n = 2^{m + n}.$ Since $\Z$ is a group, it follows by the closure property of groups that $m,n \in \Z$. Hence by definition of $H$ $AB \in H$. Since $A$ and $B$ are arbitrary elements in $H$, it follows that $H$ is closed under the group operation.\\
By the two step subgroup test, it follows that since each element in $H$ has an inverse, and since $H$ is closed under the group operation, $H\le GL(2,\R).$\\

\fbox{Ch 3, 70} Let $(G,\cdot)$, (from now on call it $G$), be a group real valued functions under multiplication $f:\R\rightarrow\R^*$ for some set $\R^*\subseteq \R$, where multiplication is defined $f\cdot g:\R\rightarrow \R^*$ such that $f\cdot g(x) = f(x)g(x)$. Let $H$ be a subset of $G$ defined $H = \{f\in G| f(2) = 1\}$.\\

\fbox{proof} Let $H$ be an arbitrary subset of $G$.
First, we must show that $H$ is nonempty. Consider the function $e(n) = 1$ for all $e\in \R$. By definition $f\in H$Let $f,g$ be arbitrary elements in $H$. By definition of $H$, $f(2) = 1 = g(2)$. Furthermore, since by definition of function multiplication, $f\cdot g(2) = f(2)g(2) = (1)(1) = 1$, we find that the function $f\cdot g$ is also in $H$. Hence $H$ is closed under the group operation of $G$.\\
Now let $f$ be an arbitrary element in $H$. Since $f\in G$, by the group axioms $f$ must have an inverse $f^{-1}$. Confusingly, this will not be the identity map on the reals, as our group operation here is not function composition but function multiplication. Hence $f^{-1}$ is the function defined $f^{-1}(x) = 1/f(x).$ Furthermore, observe that $1/f(2) = 1/1 = 1$, hence $f^{-1}\in H$. Since $f$ is arbitrary, it follows that each element in $H$ has an inverse.\\
By the two step subgroup test, $H\le G$. Q.E.D.
\end{document}