\documentclass{article}
\usepackage[utf8]{inputenc}
\newcommand{\ii}{{\bf i}}
\newcommand{\jj}{{\bf j}}
\newcommand{\kk}{{\bf k}}
\newcommand{\id}{{\bf 1}}
\newcommand{\hur}{\frac{\id+\ii+\jj+\kk}{2}}%The "Hurwitz point"
\newcommand{\hurwitz}{\Z\left[\hur,\ii,\jj,\kk\right]}%The set of Hurwitz integers
\usepackage{wrapfig}
\usepackage[utf8]{inputenc}
\usepackage[dvips]{graphicx}
\usepackage{a4wide}
\usepackage{amsmath}
\usepackage{euscript}
\usepackage{amssymb}
\usepackage{amsthm}
\usepackage{amsopn}
\usepackage[colorinlistoftodos]{todonotes}
\usepackage{graphicx}
\usepackage[T1]{fontenc}
\newcommand\mybar{\kern1pt\rule[-\dp\strutbox]{.8pt}{\baselineskip}\kern1pt}

\usepackage{ulem}
\usepackage{xcolor}
\newcommand{\cs}[1]{\color{blue}{#1}\normalcolor}

%Matrix commands
\newcommand{\ba}{\begin{array}}
\newcommand{\ea}{\end{array}}
\newcommand{\bmat}{\left[\begin{array}}
\newcommand{\emat}{\end{array}\right]}
\newcommand{\bdet}{\left|\begin{array}}
\newcommand{\edet}{\end{array}\right|}

%Environment commands
\newcommand{\be}{\begin{enumerate}}
\newcommand{\ee}{\end{enumerate}}
\newcommand{\bi}{\begin{itemize}}
\newcommand{\ei}{\end{itemize}}
\newcommand{\bt}{\begin{thm}}
\newcommand{\et}{\end{thm}}
\newcommand{\bp}{\begin{proof}}
\newcommand{\ep}{\end{proof}}
\newcommand{\bprop}{\begin{prop}}
\newcommand{\eprop}{\end{prop}}
\newcommand{\bl}{\begin{lemma}}
\newcommand{\el}{\end{lemma}}
\newcommand{\bc}{\begin{cor}}
\newcommand{\ec}{\end{cor}}
\newcommand{\lcm}{\mbox{lcm}}
\newcommand{\defn}{\fbox{definition}}
\newcommand{\prop}{\fbox{proposition}}
\newcommand{\stab}{\mbox{stab}}
\newcommand{\Aut}{\mbox{Aut}}
\newcommand{\orb}{\mbox{orb}}

\newcommand{\and}{\wedge}
\newcommand{\or}{\vee}



%sets of numbers
\newcommand{\N}{\mathbb{N}}
\newcommand{\Z}{\mathbb{Z}}
\newcommand{\Q}{\mathbb{Q}}
\newcommand{\R}{\mathbb{R}}

\title{Abstract Algebra}
\author{August, Evelyn}
\date{11/02/2021}

\begin{document}
\maketitle
%#7

\fbox{proposition: 11} Let $G=\Z_4 \bigoplus U(4)=\{(0,1),(0,3),(1,1),(1,3),(2,1),(2,3),(3,1),(3,3)\}, H=<(2,3)>=\{(2,3), (0,1)\}$, and $K=<(2,1)>=\{(2,1), (0,1)\}$. Show that $G/H$ is not isomorphic to $G/K$. \\

\fbox{proof} Let $G, H$ and $K$ be defined as above. Consider the function $\phi: G/H \longrightarrow G/K$, such that $G/H=\{aH|a \in G\}$ and $G/K=\{bK|b \in G\}$. \\
Finding the factor groups, we have $G/H = \{[(0,1)H= (2,3)H= \{(2,3),(0,1)\}],[(0,3)H = (2,1)H =  \{(2,1),(0,3)\}], [(1,1)H = (3,3)H \{(3,3),(1,1)\}],[(1,3)H =(3,1)H =  \{(3,1),(1,3)\}]\}$. Furthermore, $G/K = \{[(0,1)K = (2,1)K = \{(2,1),(0,1)\}], [(0,3)K = (2,3)K = \{(2,3),(0,3)\}], [(1,1)K = (3,1)K = \{(1,1),(3,1)\}],[(1,3)K = (3,3)K = \{(1,3),(3,3)\}]\}$.\\

Notice that $[(3,1)K][(3,1)K] = [(3,1)(3,1)]K = (2,1)K = K$. Hence the element order $(3,1)K$ is $2$. Likewise, $((1,1)K)^2 = (2,1)K = K$, $[(1,3)K]^2 = (2,1)K = K $, $K$ is the identity hence its order is one, hence the orders of these elements are $2, 2,$ and $1$ respectively. Note that there are no elements of order $4$.\\

Luckily, we only need to type one calculation to finish the proof. Notice that $[(1,1)H]^4 = H$. Also, none of the elements in the cycle for $(1,1)$ are in $H$, besides $(0,1)$ which is the identity. Hence for all elements in this cycle other than the identity $aH \ne H$. Hence the order of $H$ is also the order of $(1,1)$, which is four.\\

Since there is an element in $G/H$ of order four, and since there isn't one in $G/K$, it follows since isomorphisms preserve element order that $G/H$ is not isomorphic to $G/K$.
Q.E.D.
\\
\newpage
\fbox{proposition 55} In $D_4$, let $K=\{R_0, D\}$ and let $L=\{R_0, D, D', R_{180}\}$. Show that $K$ is normal in $L$ and $L$ is normal in $D_4$, but $K$ is not normal in $D_4$, hence normality is not transitive.\\

\fbox{proof} Let us first show that $K$ is normal in $L$. Let $K \and L$ be defined as above. We have to show that $xKx^{-1} \subseteq K$, for all $x \in L$. We are thus going to look at all the elements separately. Obviously, for $R_0$, it follows that $xR_0x^{-1}=xx^{-1}=R_0 \in K$ for all elements in $L$. Let us therefore consider $D$. The cases $x=R_0$ and $x=D$ are clear and follow by properties of closure and inverses. Furthermore, by analyzing the caley table for $D_4$ we see that $D'DD'^{-1}=D'DD'=D \in K$ and $R_{180}DR_{180}^{-1}=R_{180}DR_{180}=D \in K$. Thus, $K$ is normal in $L$.\\

Let us now show that $L$ is normal in $D_4$. Again, $xR_0x^{-1}=xx^{-1}=R_0 \in L$ for all $x \in D_4$. We have already looked at the cases $x \in \{R_0, D, D', R_{180}\}$ in $xDx^{-1}$. It remains to show the cases such that $x \in \{R_{90}, R_{270}, V, H\}$. Therefore, consider $R_{90}DR_{90}^{-1}=R_{90}DR_{270}=D' \in L$, $R_{270}DR_{90}=D' \in L$, $VDV^{-1}=VDV=D' \in L$ and $HDH^{-1}=HDH=D' \in L$. Analogously, by replacing $D$ by $D'$, it follows that $xD'x^{-1}=y \in L$ for all $x \in D_4$. Lastly, we still need to consider $xR_{180}x^{-1}$ for all $x \in D_4$. Again, $x=R_0 \and x=R_{180}$ follow by group properties. Consider:  $R_{90}R_{180}R_{270}=R_{90}R_{180}R_{270}=R_{180} \in L$, and $xR_{180}x^{-1}xR_{180}x=R_{180} \in L$, for $x \in \{V, H, D, D'\}$. Hence, it follows that $xLx^{-1} \subseteq L$ for all $x \in D_4$ and L is normal in $D_4$. \\

In order to show that $K$ is not normal in $D_4$, consider, for example, $R_{90}DR_{270}=D'$, which is not in $K$. Thus, $K$ is not normal in $D_4$ and normality is not transitive.\\

\fbox{proposition 68} If $N$ is a characteristic subgroup of $G$, show that $N$ is a normal subgroup of $G$. \\

\fbox{proof} Let $G$ and $N$ be defined as above. Consider the inner automorphism of $G$ induced by $a \in G$: $\phi_a(y)=aya^{-1}$. Since $\phi_a$ is an automorphism, it follows by properties of characteristic subgroups that $\phi_a(N)=aNa^{-1}=N$ for some $a \in G$. Since this is exactly the definition of a normal subgroup, it follows that $N$ is normal in $G$. qed. \\

\end{document}