\documentclass[10pt]{beamer}
\usepackage[utf8]{inputenc}

\usepackage{mathtools, amsmath, amssymb, amsthm, graphicx, fancyhdr, textcomp, enumerate, diagbox, tcolorbox, esvect, xcolor, soul, tikz, xeCJK, utopia, adjustbox} %utopia font
\usetikzlibrary{cd}
\usetheme{Boadilla}
%\usecolortheme{dolphin}
% custom colors
\definecolor{myNewColorA}{RGB}{12,12,110}
\definecolor{myNewColorB}{RGB}{16,85,154}
\definecolor{myNewColorC}{RGB}{150,158,197}
\setbeamercolor*{palette primary}{bg=myNewColorC}
\setbeamercolor*{palette secondary}{bg=myNewColorB, fg = white}
\setbeamercolor*{palette tertiary}{bg=myNewColorA, fg = white}
\setbeamercolor*{titlelike}{fg=myNewColorA}
\setbeamercolor*{title}{bg=myNewColorA, fg = white}
\setbeamercolor*{item}{fg=myNewColorA}
\setbeamercolor*{caption name}{fg=myNewColorA}
\usefonttheme{default}
\usepackage{natbib}
\usepackage{hyperref}

%------------------------------------------------------------------------

% Fields
\newcommand{\Z}{\mathbb{Z}} 
\newcommand{\R}{\mathbb{R}}
\newcommand{\Q}{\mathbb{Q}}
\newcommand{\N}{\mathbb{N}}
\newcommand{\C}{\mathbb{C}}
\newcommand{\F}{\mathbb{F}}
\newcommand{\cI}{\mathcal{I}}

\newcommand{\cL}{\mathcal{L}}

% Groups
\newcommand{\gen}[1]{\langle#1\rangle} % generated group
\newcommand{\Aut}{\mathrm{Aut}} % automorphism group
\newcommand{\mult}[1]{(\Z/#1\Z)^{\times}} % easy quotient group
\newcommand{\Syl}[1]{\mathrm{Syl}_{#1}} % Sylow subgroup
\newcommand{\triv}[1]{\{e_{#1}\}} % trivial subgroup {e}

% Group operations
\newcommand{\stab}{\mathrm{stab}} % stabilizer
\newcommand{\orb}{\mathrm{orb}} % orbit
\newcommand{\normal}{\trianglelefteq} % normal subgroup
\newcommand{\subnormal}{\trianglerighteq} % subnormal subgroup 
\newcommand{\actson}{\curvearrowright} % left group action
\newcommand{\hitby}{\curvearrowleft} % right group action
\newcommand{\ncl}{\mathrm{ncl}} % normal closure
\newcommand{\Fix}{\mathrm{Fix}} % set of elements fixed by action

% Quandles
\newcommand{\qgen}[1]{\langle\langle#1\rangle\rangle}
\newcommand{\Conj}{\mathrm{Conj}} % conjugate quandle

% Quandle operations
\newcommand{\sq}{\preccurlyeq} % subquandle
\newcommand{\thru}{\rhd} % through
\newcommand{\bthru}{\inv{\thru}} % backthrough
\newcommand{\Inn}{\mathrm{Inn}} % inner automorphism group
\newcommand{\Trans}{\mathrm{Trans}} % transvection group
\newcommand{\Adconj}{\mathrm{Adconj}} % adconj group
\newcommand{\bigthru}{\scalebox{1.5}{$\rhd$}} % big through
\newcommand{\psq}{\prec} %subquandle 
% Other 
\newcommand{\inv}[1]{#1^{-1}}
\newcommand{\im}{\text{im}} % image
\DeclareMathOperator*{\restrict}{\upharpoonright} % restriction
\newcommand{\Id}{\mathrm{Id}} % identity map
\newcommand{\Mod}[1]{\ (\mathrm{mod}\ #1)} % mod
\renewcommand{\phi}{\varphi}

% Environments
\theoremstyle{plain}
\newtheorem{remark}{Remark}
\newtheorem{proposition}{Proposition}
\newtheorem{claim}{Claim}
\newtheorem{subclaim}{Subclaim}
\newtheorem{tool}{Tool}
\newtheorem{conjecture}{Conjecture}

%------------------------------------------------------------------------


\setbeamerfont{title}{size=\large}
\setbeamerfont{subtitle}{size=\small}
\setbeamerfont{author}{size=\small}
\setbeamerfont{date}{size=\small}
\setbeamerfont{institute}{size=\small}
\title[Subquandle Lattices]{Subquandle Lattices}
\author[Amsberry, Bergquist, Horstkamp, Lee ]{Kieran Amsberry, August Bergquist, Thomas Horstkamp, Meghan Lee \\ \vspace{0.15in} Mentor: Dr. David Yetter}




\institute[ ]{Kansas State University REU}
\date[August 5, 2022]
{August 5, 2022}

\titlegraphic{
 \includegraphics[width=1.25cm]{images/BC_logo.png}
    \hspace{0.5cm}
    \includegraphics[width=1.25cm]{images/WU_logo.png} 
    \hspace{0.5cm}
    \includegraphics[width=1.25cm]{images/cmu_logo.png}
    \hspace{0.5cm}
    \includegraphics[width=1.25cm]{images/oxy_logo.png} 
    \hspace{0.5cm}
    \includegraphics[width=1.25cm]{images/kansas-state-university-seal.png}
  % \begin{tikzpicture}[overlay,remember picture]
%\node[left=0.2cm] at (current page.30){
   % \includegraphics[width=1.5cm]{images/kansas-state-university-seal.png}
%};
%\end{tikzpicture}

}

\logo{
\includegraphics[width=1cm]{images/NSF LOGO.png}

 
}





%------------------------------------------------------------
%This block of commands puts the table of contents at the 
%beginning of each section and highlights the current section:
%\AtBeginSection[]
%{
%  \begin{frame}
%    \frametitle{Contents}
%    \tableofcontents[currentsection]
%  \end{frame}
%}
\AtBeginSection[]{
  \begin{frame}
  \vfill
  \centering
  \begin{beamercolorbox}[sep=8pt,center,shadow=true,rounded=true]{title}
    \usebeamerfont{title}\insertsectionhead\par%
  \end{beamercolorbox}
  \vfill
  \end{frame}
}
%------------------------------------------------------------



\begin{document}

\frame{\titlepage}

\begin{frame}{Motivation}
\begin{itemize} %Maybe add application to Topology
    \item Quandles are useful in knot theory, expressing a complete invariant of unoriented knots; they also have interesting properties as an algebraic structure.
    \vspace{0.25in}
    \item As with other algebraic structures such as groups and rings, quandles have sub-objects called subquandles.%emphasize that we're looking at subquandles; buzz words are nice 
    \vspace{0.25in}
    \item We consider inner automorphism groups, lattice structure, and complemented properties of subquandles.
\end{itemize}
\end{frame}

\begin{frame}{Definitions} % ADD CITATIONS
    \begin{definition} 
    A \textbf{quandle} is a set $Q$ equipped with binary operations $\thru$ and $\bthru$ satisfying the following conditions for all $x,y,z\in Q$:
    \begin{enumerate}
    \item[Q1.] Idempotence: $x \thru x = x$.
    \pause
    \item[Q2.] Inversion: $(x \thru y) \bthru y = x = (x \bthru y) \thru y$.
    \pause
    \item[Q3.] Distributivity: $(x \thru y)\thru z = (x\thru z)\thru(y\thru z)$.
    \end{enumerate}
   \end{definition}
    \pause
    \vspace{0.2in}
    We say $Q'\subseteq Q$ is a \textbf{subquandle} of $Q$ if it closed under $\thru$ and $\bthru$. Denote this by $Q'\sq Q$, or $Q'\psq Q$ if $Q'\neq Q.$
\end{frame}

\begin{frame}{Definitions}
    \begin{definition}
\begin{itemize}
    \item If $(Q,\thru)$ and $(R,\thru_1)$ are quandles, a \textbf{quandle homomorphism} $f:Q\to R$ is a function satisfying $f(a\thru b)=f(a)\thru_1 f(b)$ for every $a,b\in Q$.
    \pause
    \item A bijective quandle homomorphism is a quandle \textbf{isomorphism}.
    \pause
    \item A quandle isomorphism with equal domain and codomain is a quandle \textbf{automorphism}.
    \pause
    \item The quandle automorphisms form the \textbf{automorphism group}, $\Aut(Q)$.
\end{itemize}
\end{definition}
\pause
\begin{definition}
Given a quandle $Q$ and an element $y\in Q$, the \textbf{symmetry} at $y$ is the automorphism of $Q$ of the form $S_y : x \mapsto x\thru y$. \\
\vspace{0.1in}
The \textbf{inner automorphism group} of $Q$ is $\Inn(Q)= \gen{\{S_q \mid q \in Q\}}.$ Note that $\Inn(Q)\normal \Aut(Q)$ \cite{Joyce1982}.
\end{definition}

In fact, if $Q'\sq Q$, $\Inn(Q)$ acts on $Q'$ by functional application. \\
This action also induces an action on the subquandle lattice.
\end{frame}

\begin{frame}{Multiplication Tables For Finite Subquandles}

\begin{example}
The Tait quandle $(\mathbf{T}_3,\thru)$ with underlying set $\{1,2,3\}$ has operation multiplication table
\begin{center}
    \begin{tabular}{c|c c c} 
        $\thru$ &$1$ &$2$ &$3$ \\
        \hline
         $1$ & $1$ & $3$ & $2$\\
         $2$ & $3$ & $2$ & $1$ \\
         $3$ & $2$ & $1$ & $3$
    \end{tabular}
\end{center}
\end{example}

    
\end{frame}

\begin{frame}{Subquotient Theorem} %State that WE proved this theorem! last names
    \begin{theorem}
    Suppose $Q'\sq Q$. Then, $\Inn(Q')$ is a subquotient of $\Inn(Q)$. That is, there is some subgroup $S_{Q'}\leq \Inn(Q)$ and some normal subgroup $K_{Q'}\normal S_{Q'}$, such that $\Inn(Q') \cong S_{Q'}/K_{Q'}$. 
    \end{theorem}
    \vspace{0.25in}
    \pause
    \begin{proof}
    Define $S_{Q'}=\gen{S_x \mid x \in Q'} \leq \Inn(Q).$ Note that $\tau:S_{Q'}\to \Inn(Q')$ via $\tau(f)=f|_{Q'}$ is a well-defined surjective homomorphism with image $\Inn(Q')$. Hence, $\Inn(Q')\cong S_{Q'}/\ker(\tau).$
    \end{proof}
\end{frame}

\begin{frame}{Complemented Subquandle Lattices}

\begin{definition}
The set of subquandles of any quandle $Q$ under inclusion forms a lattice \cite{SakiKiani2021}, which we denote as $\cL(Q)$. 
\end{definition}
\pause

\begin{definition}
Given two subquandles $Q_1, Q_2\sq Q$, their \textbf{meet} is $Q_1\wedge Q_2 = Q_1\cap Q_2$ and their \textbf{join} is $Q_1\vee Q_2 =  \qgen{Q_1\cup Q_2}$.
\end{definition}
\pause

\begin{definition}
A subquandle $Q_1\sq Q$ is \textbf{complemented} in $Q$ if there is some $Q_2\sq Q$ such that $Q_1 \wedge Q_2 = \emptyset$, and $Q_1\vee Q_2 = Q$. The subquandle lattice $\cL(Q)$ is complemented if every subquandle is complemented.
\end{definition}
\end{frame}


\begin{frame}{Saki And Kiani's Paper}

Saki and Kiani \cite{SakiKiani2021} showed that the subrack lattice of a finite rack is complemented. Since quandles are racks, this holds for quandles as a corollary.

\begin{corollary}
The subquandle lattice of every finite quandle is complemented.
\end{corollary}
\vspace{0.2in}
\pause

\begin{theorem}[Saki and Kiani]
Consider the quandle $\Q$ with the dihedral structure, $x\thru^* y = 2x-y$, $\forall x,y\in \Q$. The subquandle $\{0\}$ is not complemented.
\end{theorem}
\end{frame}



\begin{frame}{Ind-finite Quandles}
% Investigate whether Saki and Kiani's results about finite quandles specializes to certain classes of infinite quandles that are more similar to finite quandles: ind-finite quandles and profinite quandles.

\begin{definition}
A nonempty quandle $Q$ is \textbf{ind-finite} if there is a family of finite subquandles $\{Q_i\}_{i \in \cI}$ indexed by a directed set $\cI$ such that each $Q_i\sq Q$, $|Q_i|<\infty$, $Q_i\psq Q_{i+1}$, and $Q=\bigcup_{i\in\cI} Q_i.$ 
\end{definition}    
\pause

\begin{lemma}
Suppose $Q$ is a quandle. Then, $Q$ is ind-finite if and only if every finitely generated subquandle of $Q$ is finite.
\end{lemma}
\pause

\begin{example}
Let the quandle operation $\thru^*$ be given by $x\thru^* y = 2x-y$.\\
\begin{itemize}
\item Neither $(\Q,\thru^*)$ nor $(\Q/\Z,\thru^*)$ has a complemented sublattice.
\item $(\Q / \Z, \thru^*)$ is ind-finite but $(\Q, \thru^*)$ is not.
\end{itemize}
\end{example}
\end{frame}

\begin{frame}{Strongly Complemented Subquandles}
\begin{definition}
A subquandle $Q'\sq Q$ is \textbf{strongly complemented} if $Q\setminus Q'\sq Q$.
\end{definition}

\begin{example}
In the quandle $Q$ represented by the matrix \[M_Q =\begin{bmatrix}
1 & 1 & 1\\
3 & 2 & 2\\
2 & 3 & 3
\end{bmatrix}.\]
The subquandles
\[\begin{bmatrix}2 & 2\\
3 & 3\end{bmatrix}, [1]\] are strongly complemented.

\end{example}
\end{frame}


\begin{frame}{Semidisjoint Union}
\begin{definition}[Ehrman et al.]<1->
Given a sequence of quandles $Q_1, \dots , Q_n$ and a $nxn$ matrix of group homomorphisms $(M)_{ij} = g_{ij} : $, Ehrman et al. \cite{EGTY2008} defined the \textbf{semidisjoint union} as follows:
\begin{align*}
    \#(Q_1, \dots, Q_n, M) = \Big(\coprod_{i = 1}^nQ_i, \thru\Big).
\end{align*}
\end{definition}
\pause
\begin{itemize}
    \item Each entry of the matrix $g_{ij}:\Adconj (Q_i)\to \Aut(Q_j)$ is a group homomorphism. 
    
    \item $\thru$ is defined as $x\thru y = x\cdot g_{ij}(|y|_{Q_i})$ for $x\in Q_i$ and $y\in Q_j$.
    
    \item Note that we are not guaranteed that the semidisjoint union is a quandle. If the matrix $M$ gives rise to a quandle, it is called a \textbf{mesh}.
    \item Ehrman et al. provided a necessary and sufficient condition for $M$ to be a mesh.
\end{itemize}

\end{frame}

\begin{frame}{Orbit Decomposition Theorem}

\begin{theorem}<1->[Ehrman et al.]
Let $Q$ be a quandle, and let $Q_1,\dots, Q_n$ be its orbits under the inner automorphism group. Then we can construct a mesh $M$ such that 
\[Q = \#(Q_1, \dots, Q_n, M).\]
\end{theorem}
\vspace{0.15in}
\begin{itemize}
  
    \item<2-> Note that the orbits need not be connected. Hence the orbits themselves may be decomposable via the previous theorem. 

\end{itemize}
%     \begin{definition}<3->
% Given a quandle $Q$, the \textbf{decomposition depth}, $D(Q)$, is the number of iterations it takes to decompose $Q$ into a semidisjoint union of connected quandles.
% \end{definition}

    
\end{frame}


\begin{frame}{Two Actions}

\begin{itemize}
    \item We've shown how $\Inn(Q)$ acts on $Q$ by functional application.
    \item This action allows us to construct an action of $\Inn(Q)$ upon $\mathcal{L}(Q)$.
\end{itemize}

\vspace{0.5cm}
\begin{definition}
The action of $\Inn (Q)$ on Q' is also given by functional application, denoted $Q'\cdot \Inn(Q)$. \\
\vspace{0.15in}
The action of $\Inn(Q)$ upon $\mathcal{L}(Q)$ is defined $Q'\sigma = \sigma(Q')$ for all $Q'\in \mathcal{L}(Q)$, and for all $\sigma\in \Inn(Q)$.
The orbit of $Q'$ under this action is denoted by $[Q']\cdot \Inn(Q) = \{ Q'\sigma: \sigma \in \Inn(Q)\}$.\\
\end{definition}
\end{frame}

\begin{frame}{Strongly Complemented Classification}
Using both of these group actions, we managed to classify strongly complemented subquandles using the following theorem:


\begin{theorem}
Let $Q$ be a quandle, and let $Q'\sq Q$. Denote the subquandle lattice of $Q$ by $\mathcal{L}(Q)$. The following are equivalent:
\begin{itemize}
    \item $Q\setminus Q'\sq Q$,
    \item $Q'$ is a union of orbits under the action of $\Inn(Q)$ on $Q$,
    \item $Q'$ is a fixed point of the action of $\Inn(Q)$ on $\mathcal{L}(Q)$,
    \item $Q = \#(Q',Q\setminus Q', M)$ for a mesh $M$ as constructed in the orbit decomposition theorem of Ehrman et al.
\end{itemize}
\end{theorem}

\end{frame}

\begin{frame}{Strongly Complemented Within Strongly Complemented}


\begin{lemma}[1]
Let $Q''\sq Q' \sq Q$ such that $Q''$ is strongly complemented within $Q'$, and $Q'$ is strongly complemented within $Q$. Then for any $\gamma\in \Inn(Q)$, $Q''\gamma$ is strongly complemented within $Q'$.
\end{lemma}
\pause
\begin{lemma}[2]
Let $Q''\sq Q' \sq Q$ be as in Lemma (1). Then for all $\sigma\in \Inn(Q)$, there exists some $\sigma'\in \langle S(Q\setminus Q')\rangle$ such that $Q''\sigma = Q''\sigma'$.
\end{lemma}
\begin{proof}
\begin{itemize}
    \item Suppose $\sigma = S_{x_1}^{\epsilon_1}\dots S_{x_n}^{\epsilon_n}$ for each $x_j\in Q$, $\epsilon_j \in \{-1,1\}$.
    \item Suppose $\exists x_i \in Q'$ whose symmetry appears in $\sigma$ which we cannot remove without changing the action.
    \item By Lemma 1, $Q''$ acted upon by the string of $\sigma$ up until $x_i$, denote it by $\gamma$, is strongly complemented in $Q'$.
\end{itemize}
\end{proof}
\end{frame}

\begin{frame}{Strongly Complemented Within Strongly Complemented}

\begin{proof}[Proof (\textit{continued})]

\begin{itemize}
    \item Since $x_i\in Q'$, $S_{x_i}|_{Q'}$ acts the same upon $Q'$ as $S_{x_i}$
    \item Then, $Q''\gamma S_{x_i} = Q''\gamma$ by the equivalence theorem. Hence, removing $S_{x_i}$ won't change anything, so there never was a counterexample.
    \item Thus, we can construct $\sigma'$ by removing symmetries represented by elements in $Q'$.
\end{itemize}
 
\end{proof}


\begin{lemma}[3]

Suppose that $Q''$ is strongly complemented within $Q$, while $Q''\sq Q'\sq Q$. Then, $Q''$ is strongly complemented within $Q'$.

\end{lemma}

\end{frame}

\begin{frame}{Strongly Complemented Within Strongly Complemented}

\begin{theorem}
Suppose $Q$ is a quandle, with subquandles $Q''\sq Q'\sq Q$, such that $Q''$ is strongly complemented within $Q'$, while $Q'$ is strongly complemented within $Q$. Then $Q''$ is complemented within $Q$ by the subquandle $Q\setminus Q''\cdot \Inn(Q)$.
\end{theorem}
\pause

\begin{proof}
\begin{itemize}
    \item By Lemma 3 and the equivalence theorem, $Q''$ is strongly complemented within $Q''\cdot \Inn(Q)$, which in turn is strongly complemented within $Q$.
    \item $Q''\wedge Q\setminus Q''\cdot \Inn(Q) = \emptyset$
    \item Using Lemmas 2 and 1, we show that $Q''\cdot \Inn(Q) \subset Q''\vee Q\setminus Q''\cdot \Inn(Q)$, hence $Q''\vee Q\setminus Q''\cdot \Inn(Q) = Q$.
    \item Hence $Q''$ has $Q\setminus Q''\cdot \Inn(Q)$ as a complement.
\end{itemize}
\end{proof}
    
\end{frame}

\begin{frame}{Profinite Quandles}

\begin{enumerate}
    \item We consider the dual to ind-finite quandles:
    \vspace{0.1in}
    \pause
    \item A quandle $Q$ is \textit{profinite} it is the inverse limit of an inverse system composed of a family of finite quandles and their morphisms.
    \vspace{0.1in}
    \begin{itemize}
        \pause
        \item Proved profinite quandles are quandles under coordinatewise operations.
        \vspace{0.1in}
        \pause
        \item Proved profinite abelian groups are profinite Takasaki quandles ($x\thru y=2y-x$) under coordinatewise operations.
    \end{itemize}
    \vspace{0.2in}
    \pause
    \item Are the subquandle lattices of profinite quandles complemented?
\end{enumerate}
\end{frame}


%I don't think we'll want this in the presentation since we killed it

% \begin{frame}{Quandle multiplication tables}

% \end{frame}

% \begin{frame}[fragile]{Span}
%     Suppose $Q'\sq Q$. Using the previous subquotient theorem, we obtain a span of $Q'$: 
%     \begin{center}
%     \begin{tikzcd}[row sep=tiny]
%                      &        & S_{Q'} \arrow[dl, hook', "\subseteq"] \arrow[dr, two heads, "\tau"] \\
%     Q \arrow[r,hook, "\ell"]  & \Inn(Q) & & \Inn(Q')
%     \end{tikzcd}
%     \end{center}
%     \pause
%     When is $Q'$ uniquely determined by its span?
%     \pause
%     \begin{definition}
%         $Q$ is \textbf{behaviorally distinct} if the map $\ell:Q\to\Inn(Q)$ via $\ell(q)=S_q$ is injective.
%     \end{definition}
% \end{frame}



% \begin{frame}{Complemented subquandles}


% \begin{definition}
%     $Q'\sq Q$ is \textbf{complemented} if there exists some subquandle $Q'' \sq Q$ such that $Q'\vee Q''=Q$ and $Q'\wedge Q'' = \O$. Note that $Q'\vee Q''=\gen{Q'\cup Q''}$, and $Q'\wedge Q'' = Q'\cap Q''.$ \\
%     \vspace{0.1in}
%     We denote this as $Q'\sq_{c} Q.$
% \end{definition}

% \vspace{0.25in}
% A central goal of our project was to characterize when subquandles are complemented. It turns out that all finite subquandles are complemented \cite{SakiKiani2021}. 


% \end{frame}
    
\begin{frame}{Acknowledgments}
   We would like to thank Dr. David Yetter for his mentorship and support, as well as  Dr. Marianne Korten, Dr. Kim Klinger-Logan, and Kansas State University for making this research experience possible.
   
   \vspace{0.25in}
   
   This research was conducted under the support of NSF grant DMS-1659123.
 \end{frame}

%%%%%%%%%%%%%%%%%%%%%%%%

\subsection{References}
\begin{frame}{References}
\begin{thebibliography}{10}

\bibitem{Joyce1982}
    David, Joyce. ``A classifying invariant of knots, the knot quandle." \emph{Journal of Pure and Applied Algebra}, vol. 23, pp. 37-65, 1982.

\bibitem{EGTY2008}
    G. Ehrman, A. Gurpinar, M. Thibault, D.N. Yetter. ``Toward a classification of finite quandles." \emph{Journal of Knot Theory and Its Ramifications}, vol. 17, pp. 511-520, 2008.

\bibitem{SakiKiani2021}
    A. Saki and D. Kiani, ``Complemented Lattices of Subracks." \emph{Journal of Algebraic Combinatorics}, vol. 53, pp. 455-468, 2021.
\end{thebibliography}
\end{frame}

\begin{frame}
\begin{center}
    \huge Questions?
\end{center}
   
\end{frame}
%%%%%%%%%%%%%%%%%%%%%%%%

% \subsection{Acknowledgements}
%     \begin{frame}{Acknowledgments}
%         Insert NSF support thing later, \\
%         NSF grant #DMS-1659123
%     \end{frame}




\end{document}



