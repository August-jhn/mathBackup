\documentclass{article}
\usepackage[utf8]{inputenc}
\newcommand{\ii}{{\bf i}}
\newcommand{\jj}{{\bf j}}
\newcommand{\kk}{{\bf k}}
\newcommand{\id}{{\bf 1}}
\newcommand{\hur}{\frac{\id+\ii+\jj+\kk}{2}}%The "Hurwitz point"
\newcommand{\hurwitz}{\Z\left[\hur,\ii,\jj,\kk\right]}%The set of Hurwitz integers
\usepackage{wrapfig}
\usepackage{calligra}
\usepackage[utf8]{inputenc}
\usepackage[dvips]{graphicx}
\usepackage{a4wide}
\usepackage{amsmath}
\usepackage{mathtools}
\usepackage{euscript}
\usepackage{amssymb}
\usepackage{amsthm}
\usepackage{amsopn}
\usepackage[colorinlistoftodos]{todonotes}
\usepackage{graphicx}
\usepackage[T1]{fontenc}
\newcommand\mybar{\kern1pt\rule[-\dp\strutbox]{.8pt}{\baselineskip}\kern1pt}

\usepackage{ulem}
\usepackage{xcolor}
\newcommand{\cs}[1]{\color{blue}{#1}\normalcolor}

%Matrix commands
\newcommand{\ba}{\begin{array}}
\newcommand{\ea}{\end{array}}
\newcommand{\bmat}{\left[\begin{array}}
\newcommand{\emat}{\end{array}\right]}
\newcommand{\bdet}{\left|\begin{array}}
\newcommand{\edet}{\end{array}\right|}
\newcommand{\inv}[1]{#1^{-1}}

%Environment commands
\newcommand{\be}{\begin{enumerate}}
\newcommand{\ee}{\end{enumerate}}
\newcommand{\bi}{\begin{itemize}}
\newcommand{\ei}{\end{itemize}}
\newcommand{\bt}{\begin{thm}}
\newcommand{\et}{\end{thm}}
\newcommand{\bp}{\begin{proof}}
\newcommand{\ep}{\end{proof}}
\newcommand{\bprop}{\begin{prop}}
\newcommand{\eprop}{\end{prop}}
\newcommand{\bl}{\begin{lemma}}
\newcommand{\el}{\end{lemma}}
\newcommand{\bc}{\begin{cor}}
\newcommand{\ec}{\end{cor}}
\newcommand{\lcm}{\mbox{lcm}}
\newcommand{\defn}{\fbox{definition}}
\newcommand{\prop}{\fbox{proposition}}
\newcommand{\stab}{\mbox{stab}}
\newcommand{\Aut}{\mbox{Aut}}
\newcommand{\orb}{\mbox{orb}}

\newcommand{\norm}{\righttriangle}

\newcommand{\and}{\wedge}
\newcommand{\or}{\vee}



%sets of numbers
\newcommand{\N}{\mathbb{N}}
\newcommand{\Z}{\mathbb{Z}}
\newcommand{\Q}{\mathbb{Q}}
\newcommand{\R}{\mathbb{R}}

\newcommand{\topT}{\mathcal{T}}
\newcommand{\standtop}{\mathcal{T}_{STD}}
\newcommand{\cc}{\mathcal{C}}


\documentclass{article}
\usepackage[utf8]{inputenc}
\newcommand{\ii}{{\bf i}}
\newcommand{\jj}{{\bf j}}
\newcommand{\kk}{{\bf k}}
\newcommand{\id}{{\bf 1}}
\newcommand{\hur}{\frac{\id+\ii+\jj+\kk}{2}}%The "Hurwitz point"
\newcommand{\hurwitz}{\Z\left[\hur,\ii,\jj,\kk\right]}%The set of Hurwitz integers
\usepackage{wrapfig}
\usepackage{calligra}
\usepackage[utf8]{inputenc}
\usepackage[dvips]{graphicx}
\usepackage{a4wide}
\usepackage{amsmath}
\usepackage{euscript}
\usepackage{amssymb}
\usepackage{amsthm}
\usepackage{amsopn}
\usepackage[colorinlistoftodos]{todonotes}
\usepackage{graphicx}
\usepackage[T1]{fontenc}
\newcommand\mybar{\kern1pt\rule[-\dp\strutbox]{.8pt}{\baselineskip}\kern1pt}

\usepackage{ulem}
\usepackage{xcolor}
\newcommand{\cs}[1]{\color{blue}{#1}\normalcolor}

%Matrix commands
\newcommand{\ba}{\begin{array}}
\newcommand{\ea}{\end{array}}
\newcommand{\bmat}{\left[\begin{array}}
\newcommand{\emat}{\end{array}\right]}
\newcommand{\bdet}{\left|\begin{array}}
\newcommand{\edet}{\end{array}\right|}
\newcommand{\inv}[1]{#1^{-1}}

%Environment commands
\newcommand{\be}{\begin{enumerate}}
\newcommand{\ee}{\end{enumerate}}
\newcommand{\bi}{\begin{itemize}}
\newcommand{\ei}{\end{itemize}}
\newcommand{\bt}{\begin{thm}}
\newcommand{\et}{\end{thm}}
\newcommand{\bp}{\begin{proof}}
\newcommand{\ep}{\end{proof}}
\newcommand{\bprop}{\begin{prop}}
\newcommand{\eprop}{\end{prop}}
\newcommand{\bl}{\begin{lemma}}
\newcommand{\el}{\end{lemma}}
\newcommand{\bc}{\begin{cor}}
\newcommand{\ec}{\end{cor}}
\newcommand{\lcm}{\mbox{lcm}}
\newcommand{\defn}{\fbox{definition}}
\newcommand{\prop}{\fbox{proposition}}
\newcommand{\stab}{\mbox{stab}}
\newcommand{\Aut}{\mbox{Aut}}
\newcommand{\orb}{\mbox{orb}}

\newcommand{\norm}{\righttriangle}

\newcommand{\and}{\wedge}
\newcommand{\or}{\vee}



%sets of numbers
\newcommand{\N}{\mathbb{N}}
\newcommand{\Z}{\mathbb{Z}}
\newcommand{\Q}{\mathbb{Q}}
\newcommand{\R}{\mathbb{R}}

\newcommand{\topT}{\mathcal{T}}
\newcommand{\standtop}{\mathcal{T}_{STD}}
\newcommand{\cc}{\mathcal{C}}


\title{Topology}
\author{August bergquist}


\begin{document}

\maketitle

\fbox{definition} Let $f:X\rightarrow Y$ be a continuous function. Then $f_* : \pi_1(X,x_0)]\rightarrow \pi_1(Y, f(x_0))$ defined by $f_*([\alpha]) = [f\circ \alpha]$ is called the induced homomorphism on fundamental groups.\\

\fbox{Exercise 12.23} Check that for a continuous function $f:X\rightarrow Y$, the induced homomorphism $f_*$ is well-defined (that is, the image of an equivalence class is independent of the chosen representative). Show that it is indeed a group homomorphism.\\

\fbox{proof} First, suppose that $\alpha$ and $\beta$ are homotopic loops in $X$ based at $x_0$, that is, $[\alpha] = [\beta]$. We will show that the images of $[\alpha]$ and $[\beta]$ are equal. In other words, that $f([\alpha]) = f([\beta])$. Or, as is equivalent from the definition of $f_*$, that $[f\circ \alpha] = [f\circ \beta]$.\\

Before we jump in, let's make sure that $f\circ \beta$ and $f\circ \alpha$ are actually loops based at $x_0$, and hence that $f_*([\alpha])[f\circ\alpha]\in \pi_1(Y,,)$. Otherwise the codomain of $f$ is not even what it is claimed to be. Since $f$ is a continuous function from $X$ to $Y$, and $\alpha$ and $\beta$ are continuous functions from $[0,1]$ to $X$, it follows that the compositions $f\circ\alpha$ and $f\circ \beta$ are well defined. Furthermore, the composition of continuous functions is continuous, hence $f\circ \alpha$ and $f\circ \beta$ are continuous functions from $[0,1]$ to $Y$. It remains to be shown that $f\circ \alpha(0) = f\circ \alpha(1) = f(x_0)$ and $f\circ \beta(0) = f\circ \beta(1) = f(x_0)$. By composition, and since $\alpha(0) = x_0$, it follows that $f\cric\alpha(0) = f(\alpha(0)) = f(x_0)$. Similarly, by composition and since $\alpha(1) = x_0$, it follows that $f\circ \alpha(1) = f(\alpha(1)) = f(x_0)$. Hence $f\circ \alpha$ really is a loop in $Y$ based at $f(x_0)$, thus $f_*([\alpha]) = [f\circ\alpha]\in \pi_1(Y,f(x_0))$. That is, any element in the domain $\pi_1$ Since $\alpha$ and $\beta$ are arbitrary, the same argument applies to $f_*(\beta)$ as well.\\

Having shown that $f_*([\alpha])$ and $f_*(\beta)$ really are elements of $\pi_1(Y, f(x_0))$, and since path equivalence is an equivalence relation, it follows that in proving that $f_*([\alpha]) = [f\circ \alpha] f_*([\beta]) = [f\circ \beta]$, it suffices to show that $f\circ \alpha \sim f\circ \beta$. To show this, we must find a homotopy. \\

We can do something nifty here. Since $\alpha\sim \beta$ in $X$, there must exist some homotopy between $\alpha$ and $\beta$. Call this homotopy $H: [0,1]\times[0,1]\rightarrow X$. Consider the composition $F 
\coloneqq f\circ H : [0,1]\times [0,1]\rightarrow Y$. We will show that $F$ is in fact a homotopy between $f\circ \alpha$ and $f\circ \beta$, and hence $f\circ\alpha\sim f\circ \beta$, and thus by transitivity of equivalence relations $[f\circ \alpha] = [f\circ \beta]$ (that is, $f_*([\alpha]) = f_*([\beta])$). This is actually super straightforward:
\begin{enumerate}
    \item Since by definition of a homotopy, $H(s, 0) = \alpha(s)$ for all $s\in [0,1]$, it follows that $F(s,0) = f\circ H(s,0) = f(H(s,0)) = f(\alpha(s)) = f\circ \alpha(s)$ for all $s\in [0,1]$. This satisfies the first requirement.
    \item Since by definition of a homotopy, $H(s,1) = \beta(s)$ for all $s\in [0,1]$, it follows that $F(s, 1) = f\circ H(s,1) = f(H(s,1)) = f(\beta(s)) = f\circ \beta(s)$ for all $s\in [0,1]$. This satisfies the second requirement.
    \item Since by definition of a homotopy, $H(0,t) = \alpha(0) = \beta(0)$ for all $t\in [0,1]$, it follows by construction of $F$ that $F(0,t) = f\circ H(0,t) = f(H(0,t))$. Hence by substitution $F(0,t) = f(\alpha(0)) = f\circ \alpha(0) = f(x_0) = f\circ \beta(0)$ (since $f\circ \alpha$ and $f\circ \beta$) are loops in $Y$ based at $f(x_0)$, as we have shown in the first part of this proof.) This satisfies the third requirement.
    \item Since by definition of a homotopy, 
\end{enumerate}



 \end{document}
