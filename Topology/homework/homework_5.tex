\documentclass{article}
\usepackage[utf8]{inputenc}
\newcommand{\ii}{{\bf i}}
\newcommand{\jj}{{\bf j}}
\newcommand{\kk}{{\bf k}}
\newcommand{\id}{{\bf 1}}
\newcommand{\hur}{\frac{\id+\ii+\jj+\kk}{2}}%The "Hurwitz point"
\newcommand{\hurwitz}{\Z\left[\hur,\ii,\jj,\kk\right]}%The set of Hurwitz integers
\usepackage{wrapfig}
\usepackage{calligra}
\usepackage[utf8]{inputenc}
\usepackage[dvips]{graphicx}
\usepackage{a4wide}
\usepackage{amsmath}
\usepackage{euscript}
\usepackage{amssymb}
\usepackage{amsthm}
\usepackage{amsopn}
\usepackage[colorinlistoftodos]{todonotes}
\usepackage{graphicx}
\usepackage[T1]{fontenc}
\newcommand\mybar{\kern1pt\rule[-\dp\strutbox]{.8pt}{\baselineskip}\kern1pt}

\usepackage{ulem}
\usepackage{xcolor}
\newcommand{\cs}[1]{\color{blue}{#1}\normalcolor}

%Matrix commands
\newcommand{\ba}{\begin{array}}
\newcommand{\ea}{\end{array}}
\newcommand{\bmat}{\left[\begin{array}}
\newcommand{\emat}{\end{array}\right]}
\newcommand{\bdet}{\left|\begin{array}}
\newcommand{\edet}{\end{array}\right|}
\newcommand{\inv}[1]{#1^{-1}}

%Environment commands
\newcommand{\be}{\begin{enumerate}}
\newcommand{\ee}{\end{enumerate}}
\newcommand{\bi}{\begin{itemize}}
\newcommand{\ei}{\end{itemize}}
\newcommand{\bt}{\begin{thm}}
\newcommand{\et}{\end{thm}}
\newcommand{\bp}{\begin{proof}}
\newcommand{\ep}{\end{proof}}
\newcommand{\bprop}{\begin{prop}}
\newcommand{\eprop}{\end{prop}}
\newcommand{\bl}{\begin{lemma}}
\newcommand{\el}{\end{lemma}}
\newcommand{\bc}{\begin{cor}}
\newcommand{\ec}{\end{cor}}
\newcommand{\lcm}{\mbox{lcm}}
\newcommand{\defn}{\fbox{definition}}
\newcommand{\prop}{\fbox{proposition}}
\newcommand{\stab}{\mbox{stab}}
\newcommand{\Aut}{\mbox{Aut}}
\newcommand{\orb}{\mbox{orb}}

\newcommand{\norm}{\righttriangle}

\newcommand{\and}{\wedge}
\newcommand{\or}{\vee}



%sets of numbers
\newcommand{\N}{\mathbb{N}}
\newcommand{\Z}{\mathbb{Z}}
\newcommand{\Q}{\mathbb{Q}}
\newcommand{\R}{\mathbb{R}}

\newcommand{\topT}{\mathcal{T}}
\newcommand{\standtop}{\mathcal{T}_{STD}}
\newcommand{\cc}{\mathcal{C}}


\documentclass{article}
\usepackage[utf8]{inputenc}
\newcommand{\ii}{{\bf i}}
\newcommand{\jj}{{\bf j}}
\newcommand{\kk}{{\bf k}}
\newcommand{\id}{{\bf 1}}
\newcommand{\hur}{\frac{\id+\ii+\jj+\kk}{2}}%The "Hurwitz point"
\newcommand{\hurwitz}{\Z\left[\hur,\ii,\jj,\kk\right]}%The set of Hurwitz integers
\usepackage{wrapfig}
\usepackage{calligra}
\usepackage[utf8]{inputenc}
\usepackage[dvips]{graphicx}
\usepackage{a4wide}
\usepackage{amsmath}
\usepackage{euscript}
\usepackage{amssymb}
\usepackage{amsthm}
\usepackage{amsopn}
\usepackage[colorinlistoftodos]{todonotes}
\usepackage{graphicx}
\usepackage[T1]{fontenc}
\newcommand\mybar{\kern1pt\rule[-\dp\strutbox]{.8pt}{\baselineskip}\kern1pt}

\usepackage{ulem}
\usepackage{xcolor}
\newcommand{\cs}[1]{\color{blue}{#1}\normalcolor}

%Matrix commands
\newcommand{\ba}{\begin{array}}
\newcommand{\ea}{\end{array}}
\newcommand{\bmat}{\left[\begin{array}}
\newcommand{\emat}{\end{array}\right]}
\newcommand{\bdet}{\left|\begin{array}}
\newcommand{\edet}{\end{array}\right|}
\newcommand{\inv}[1]{#1^{-1}}

%Environment commands
\newcommand{\be}{\begin{enumerate}}
\newcommand{\ee}{\end{enumerate}}
\newcommand{\bi}{\begin{itemize}}
\newcommand{\ei}{\end{itemize}}
\newcommand{\bt}{\begin{thm}}
\newcommand{\et}{\end{thm}}
\newcommand{\bp}{\begin{proof}}
\newcommand{\ep}{\end{proof}}
\newcommand{\bprop}{\begin{prop}}
\newcommand{\eprop}{\end{prop}}
\newcommand{\bl}{\begin{lemma}}
\newcommand{\el}{\end{lemma}}
\newcommand{\bc}{\begin{cor}}
\newcommand{\ec}{\end{cor}}
\newcommand{\lcm}{\mbox{lcm}}
\newcommand{\defn}{\fbox{definition}}
\newcommand{\prop}{\fbox{proposition}}
\newcommand{\stab}{\mbox{stab}}
\newcommand{\Aut}{\mbox{Aut}}
\newcommand{\orb}{\mbox{orb}}

\newcommand{\norm}{\righttriangle}

\newcommand{\and}{\wedge}
\newcommand{\or}{\vee}



%sets of numbers
\newcommand{\N}{\mathbb{N}}
\newcommand{\Z}{\mathbb{Z}}
\newcommand{\Q}{\mathbb{Q}}
\newcommand{\R}{\mathbb{R}}

\newcommand{\topT}{\mathcal{T}}
\newcommand{\standtop}{\mathcal{T}_{STD}}
\newcommand{\cc}{\mathcal{C}}


\title{Topology}
\author{August bergquist}


\begin{document}

\maketitle
\fbox{Theorem 3.35} Show that the product topology on $X\times Y$ is the same as the topology generated by the subbasis of inverse images of open sets under the projection fucntions, that is, the subbasis is $\mathcal{S} = \{\inv{\pi_X}(U): U\in \topT_X\}\cup \{\inv{\pi_Y}(V) : V\in \topT_Y \}$.\\

\fbox{proof} We will need to show that the set of finite intersections of elements in $\mathcal{S}$ forms a basis for $\topT_{X\times Y}$. Let $\mathcal{B}$ denote the set of finite intsersections of elements in $\mathcal{S}$. We will now proceed by showing that $\mathcal{B}$ meets the requirements of Theorem 3.1 in the case of $\topT_{X\times Y}$.
\begin{itemize}
    \item We will first need to show that $\mathcal{B}\subseteq \topT_{X\times Y}$. By definition of a topology, it will suffice to show that each element in $\mathcal{S}$ is open in $\topT_{X\times Y}$. This is because $\mathcal{B}$ consists of finite intersections of elements of $\mathcal{S}$, and by definition (Theorem...)of a topology the finite intersection of open sets must be open.\\
    
    Let $W\in \mathcal{S}$ be arbitrary. By definition of the union and construction of $\mathcal{S}$ we know that $W\in \{\inv{\pi_X}(U): U\in \topT_X\}$ or $W\in \{\inv{\pi_Y}(V): V\in \topT_Y\}$. Call these cases one and two respectively.\\
    
    In the first case, we have $W = \inv{\pi_X}(U)$ for some open set $U$ in $\topT_X$. Recall from the properties of the projection function in foundations that $W = \inv{\pi_X}(U) = U\times Y$. Since both $U$ and $Y$ are open in their respective spaces, and since open sets in the product topology are the unions of such sets, we conclude that $W\in \topT_{X\times Y}$.\\
    
    Likewise, in the second case notice that $W = \inv{\pi_Y}(V) = X\times V$ for some open set $V$. By the same reasoning in the first case, we conclude that $W\in \topT_{X\times Y}$.\\
    
    Since in all cases $W\in \topT_{X\times Y}$, and since $W$ was arbitrary in $\mathcal{S}$, it follows that all elements of $\mathcal{S}$ are in $\topT_{X\times Y}.$ And as we have shown, this implies that $\mathcal{B}\subseteq \topT_{X\times Y}$.
    \item Now we must show that for each set $W\in \topT_{X\times Y}$ and for each point $p\in W$ there is some set $V\in \mathcal{B}$ such that $p\in V\subseteq U$. \\
    
    Let $W$ be arbitrary in $\topT_{X\times Y}$, and let $p$ be arbitrary in $W$. Since by definition of the product topology $\topT_{X\times Y}$ is the topology generated by the basis $\mathcal{V} = \{U\times V: U\in \topT_X V\in \topT_Y\}$, it follows by theorem 3.1 that there exists some basic set $B$ of $\mathcal{V}$ such that $p\in B \subseteq W$. By definition of $\mathcal{V}$ there must exist open sets $U$ and $V$ in spaces $X$ and $Y$ respectively such that $B = U \times V$. \\
    Now consider the subbasic sets $M = \inv{\pi_X}(U) = U\times Y$ and $N = \inv{\pi_Y}(V) = X\times V$. Consider the intersection $(U\times Y) \cap (X\times V).$ From foundations, recall that $(U\times Y) \cap (X\times V) = (U\cap X)\times (Y\times V) = U\times V = B$. Hence $B$ is basic in the basis generated by the subbasis $\mathcal{S}$. Hence there exists some element $B$ of $\mathcal{B}$ such that $p\in B\subseteq W$. Since $W$ was arbitrary in $\topT_{X\times Y}$, and since $p$ was arbitrary in $W$, it follows that the second requirement of Theorem 3.1 in the case of $\mathcal{B}$ is met.
\end{itemize}
Having shown that $\mathcal{B}$ satisfies all of the requirements of Theorem 3.1, it follows that $\mathcal{B}$ is a basis for the product topology. Q.E.D. 
\end{document}