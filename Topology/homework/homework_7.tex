\documentclass{article}
\usepackage[utf8]{inputenc}
\newcommand{\ii}{{\bf i}}
\newcommand{\jj}{{\bf j}}
\newcommand{\kk}{{\bf k}}
\newcommand{\id}{{\bf 1}}
\newcommand{\hur}{\frac{\id+\ii+\jj+\kk}{2}}%The "Hurwitz point"
\newcommand{\hurwitz}{\Z\left[\hur,\ii,\jj,\kk\right]}%The set of Hurwitz integers
\usepackage{wrapfig}
\usepackage{calligra}
\usepackage[utf8]{inputenc}
\usepackage[dvips]{graphicx}
\usepackage{a4wide}
\usepackage{amsmath}
\usepackage{euscript}
\usepackage{amssymb}
\usepackage{amsthm}
\usepackage{amsopn}
\usepackage[colorinlistoftodos]{todonotes}
\usepackage{graphicx}
\usepackage[T1]{fontenc}
\newcommand\mybar{\kern1pt\rule[-\dp\strutbox]{.8pt}{\baselineskip}\kern1pt}

\usepackage{ulem}
\usepackage{xcolor}
\newcommand{\cs}[1]{\color{blue}{#1}\normalcolor}

%Matrix commands
\newcommand{\ba}{\begin{array}}
\newcommand{\ea}{\end{array}}
\newcommand{\bmat}{\left[\begin{array}}
\newcommand{\emat}{\end{array}\right]}
\newcommand{\bdet}{\left|\begin{array}}
\newcommand{\edet}{\end{array}\right|}
\newcommand{\inv}[1]{#1^{-1}}

%Environment commands
\newcommand{\be}{\begin{enumerate}}
\newcommand{\ee}{\end{enumerate}}
\newcommand{\bi}{\begin{itemize}}
\newcommand{\ei}{\end{itemize}}
\newcommand{\bt}{\begin{thm}}
\newcommand{\et}{\end{thm}}
\newcommand{\bp}{\begin{proof}}
\newcommand{\ep}{\end{proof}}
\newcommand{\bprop}{\begin{prop}}
\newcommand{\eprop}{\end{prop}}
\newcommand{\bl}{\begin{lemma}}
\newcommand{\el}{\end{lemma}}
\newcommand{\bc}{\begin{cor}}
\newcommand{\ec}{\end{cor}}
\newcommand{\lcm}{\mbox{lcm}}
\newcommand{\defn}{\fbox{definition}}
\newcommand{\prop}{\fbox{proposition}}
\newcommand{\stab}{\mbox{stab}}
\newcommand{\Aut}{\mbox{Aut}}
\newcommand{\orb}{\mbox{orb}}

\newcommand{\norm}{\righttriangle}

\newcommand{\and}{\wedge}
\newcommand{\or}{\vee}



%sets of numbers
\newcommand{\N}{\mathbb{N}}
\newcommand{\Z}{\mathbb{Z}}
\newcommand{\Q}{\mathbb{Q}}
\newcommand{\R}{\mathbb{R}}

\newcommand{\topT}{\mathcal{T}}
\newcommand{\standtop}{\mathcal{T}_{STD}}
\newcommand{\cc}{\mathcal{C}}


\documentclass{article}
\usepackage[utf8]{inputenc}
\newcommand{\ii}{{\bf i}}
\newcommand{\jj}{{\bf j}}
\newcommand{\kk}{{\bf k}}
\newcommand{\id}{{\bf 1}}
\newcommand{\hur}{\frac{\id+\ii+\jj+\kk}{2}}%The "Hurwitz point"
\newcommand{\hurwitz}{\Z\left[\hur,\ii,\jj,\kk\right]}%The set of Hurwitz integers
\usepackage{wrapfig}
\usepackage{calligra}
\usepackage[utf8]{inputenc}
\usepackage[dvips]{graphicx}
\usepackage{a4wide}
\usepackage{amsmath}
\usepackage{euscript}
\usepackage{amssymb}
\usepackage{amsthm}
\usepackage{amsopn}
\usepackage[colorinlistoftodos]{todonotes}
\usepackage{graphicx}
\usepackage[T1]{fontenc}
\newcommand\mybar{\kern1pt\rule[-\dp\strutbox]{.8pt}{\baselineskip}\kern1pt}

\usepackage{ulem}
\usepackage{xcolor}
\newcommand{\cs}[1]{\color{blue}{#1}\normalcolor}

%Matrix commands
\newcommand{\ba}{\begin{array}}
\newcommand{\ea}{\end{array}}
\newcommand{\bmat}{\left[\begin{array}}
\newcommand{\emat}{\end{array}\right]}
\newcommand{\bdet}{\left|\begin{array}}
\newcommand{\edet}{\end{array}\right|}
\newcommand{\inv}[1]{#1^{-1}}

%Environment commands
\newcommand{\be}{\begin{enumerate}}
\newcommand{\ee}{\end{enumerate}}
\newcommand{\bi}{\begin{itemize}}
\newcommand{\ei}{\end{itemize}}
\newcommand{\bt}{\begin{thm}}
\newcommand{\et}{\end{thm}}
\newcommand{\bp}{\begin{proof}}
\newcommand{\ep}{\end{proof}}
\newcommand{\bprop}{\begin{prop}}
\newcommand{\eprop}{\end{prop}}
\newcommand{\bl}{\begin{lemma}}
\newcommand{\el}{\end{lemma}}
\newcommand{\bc}{\begin{cor}}
\newcommand{\ec}{\end{cor}}
\newcommand{\lcm}{\mbox{lcm}}
\newcommand{\defn}{\fbox{definition}}
\newcommand{\prop}{\fbox{proposition}}
\newcommand{\stab}{\mbox{stab}}
\newcommand{\Aut}{\mbox{Aut}}
\newcommand{\orb}{\mbox{orb}}

\newcommand{\norm}{\righttriangle}

\newcommand{\and}{\wedge}
\newcommand{\or}{\vee}



%sets of numbers
\newcommand{\N}{\mathbb{N}}
\newcommand{\Z}{\mathbb{Z}}
\newcommand{\Q}{\mathbb{Q}}
\newcommand{\R}{\mathbb{R}}

\newcommand{\topT}{\mathcal{T}}
\newcommand{\standtop}{\mathcal{T}_{STD}}
\newcommand{\cc}{\mathcal{C}}


\title{Topology}
\author{August bergquist}


\begin{document}

\maketitlea
\fbox{Theorem 4.47} The quotient topology defines a topology.\\

\fbox{Proof} Let $X$ and $Y$ be topological spaces, and let $f$ be a surjective function $f:X\rightarrow Y$. Let $\topT = \{U\in \mathcal{P}(Y): \inv{f}(U) \in \topT_X \}$, where $\topT_X$ denotes topology on $X$. We will need to show that $\topT$ meets all of the requirements as provided by the definition of a topology.
\begin{enumerate}
    \item To show that $Y\in \topT$, note by definition of a function, $\inv{f}(Y) = X$. Otherwise not every point in $X$ would map to some point in $Y$ under $f$, which violates the definition of a function. Furthermore, by definition of a topology, $\inv{f}(Y) = X$ must be open in $\topT_X$, hence $Y\in \topT$.
    \item To show that $\emptyset\in \topT$, consider $\inv{f}(\emptyset)$. Suppose by way of contradiction that some $x\in \inv{f}(\emptyset)$. Then $x$ gets mapped to nothing, contradiction the supposition that $f$ is a function. Hence there is nothing in $\inv{f}(\emptyset)$. In other words, $\inv{f}(\emptyset) = \emptyset$, which is open in $X$ by definition of a topology. Hence $\emptyset\in \topT$.
    \item Now suppose that $U$ and $V$ are elements of $\topT$. Then by definition of $\topT$, $\inv{f}(U)$ and $\inv{f}(V)$ are open in $X$. Furthermore, since by definition of a topology the union of open sets is open, $\inv{f}(U)\cap \inv{f}(V)$ must also be open in $X$. Recall from Foundations that $\inv{f}(U\cap V) = \inv{f}(U)\cap \inv{f}(V)$. Hence $\inv{f}(U\cap V)$ must be open in $X$. Hence by definition of the quotient topology, $U\cap V\in \topT$.
    \item Let $\{U_\alpha\}_{\alpha\in A}$ be an arbitrary collection of sets in $Y$ such that $U_\alpha\in \topT$ for all $\alpha\in A$. Then by construction of $\topT$, $\inv{f}(U_\alpha)$ is open in $X$ for all $\alpha\in A$. Recall from Foundations that the inverse image of an arbitrary union is the union of the inverse images, hence $\inv{f}\big(\bigcup_{\alhpa\in A}U_\alpha\big) = \bigcup_{\alpha\in A}\inv{f}(U_\alpha)$. Since each $\inv{f}(U_\alpha)$ is open in $X$, and since by definition of a topology the arbitrary union of open sets must also be open, we conclude that $\inv{f}\big(\bigcup_{\alhpa\in A}U_\alpha\big)$ is open in $X$ as well. Hence by definition of the quotient topology, $\bigcup_{\alpha\in A} U_\alpha\in \topT$.
\end{enumerate}
Having shown that all four requirements of a topology are met in the case of the quotient topology on $Y$ induced by $f$, we conclude that the quotient topology really is a topology. Q.E.D.

\end{document}