\documentclass{article}
\usepackage[utf8]{inputenc}
\newcommand{\ii}{{\bf i}}
\newcommand{\jj}{{\bf j}}
\newcommand{\kk}{{\bf k}}
\newcommand{\id}{{\bf 1}}
\newcommand{\hur}{\frac{\id+\ii+\jj+\kk}{2}}%The "Hurwitz point"
\newcommand{\hurwitz}{\Z\left[\hur,\ii,\jj,\kk\right]}%The set of Hurwitz integers
\usepackage{wrapfig}
\usepackage{calligra}
\usepackage[utf8]{inputenc}
\usepackage[dvips]{graphicx}
\usepackage{a4wide}
\usepackage{amsmath}
\usepackage{euscript}
\usepackage{amssymb}
\usepackage{amsthm}
\usepackage{amsopn}
\usepackage[colorinlistoftodos]{todonotes}
\usepackage{graphicx}
\usepackage[T1]{fontenc}
\newcommand\mybar{\kern1pt\rule[-\dp\strutbox]{.8pt}{\baselineskip}\kern1pt}

\usepackage{ulem}
\usepackage{xcolor}
\newcommand{\cs}[1]{\color{blue}{#1}\normalcolor}

%Matrix commands
\newcommand{\ba}{\begin{array}}
\newcommand{\ea}{\end{array}}
\newcommand{\bmat}{\left[\begin{array}}
\newcommand{\emat}{\end{array}\right]}
\newcommand{\bdet}{\left|\begin{array}}
\newcommand{\edet}{\end{array}\right|}
\newcommand{\inv}[1]{#1^{-1}}

%Environment commands
\newcommand{\be}{\begin{enumerate}}
\newcommand{\ee}{\end{enumerate}}
\newcommand{\bi}{\begin{itemize}}
\newcommand{\ei}{\end{itemize}}
\newcommand{\bt}{\begin{thm}}
\newcommand{\et}{\end{thm}}
\newcommand{\bp}{\begin{proof}}
\newcommand{\ep}{\end{proof}}
\newcommand{\bprop}{\begin{prop}}
\newcommand{\eprop}{\end{prop}}
\newcommand{\bl}{\begin{lemma}}
\newcommand{\el}{\end{lemma}}
\newcommand{\bc}{\begin{cor}}
\newcommand{\ec}{\end{cor}}
\newcommand{\lcm}{\mbox{lcm}}
\newcommand{\defn}{\fbox{definition}}
\newcommand{\prop}{\fbox{proposition}}
\newcommand{\stab}{\mbox{stab}}
\newcommand{\Aut}{\mbox{Aut}}
\newcommand{\orb}{\mbox{orb}}

\newcommand{\norm}{\righttriangle}

\newcommand{\and}{\wedge}
\newcommand{\or}{\vee}



%sets of numbers
\newcommand{\N}{\mathbb{N}}
\newcommand{\Z}{\mathbb{Z}}
\newcommand{\Q}{\mathbb{Q}}
\newcommand{\R}{\mathbb{R}}

\newcommand{\topT}{\mathcal{T}}
\newcommand{\standtop}{\mathcal{T}_{STD}}
\newcommand{\cc}{\mathcal{C}}


\documentclass{article}
\usepackage[utf8]{inputenc}
\newcommand{\ii}{{\bf i}}
\newcommand{\jj}{{\bf j}}
\newcommand{\kk}{{\bf k}}
\newcommand{\id}{{\bf 1}}
\newcommand{\hur}{\frac{\id+\ii+\jj+\kk}{2}}%The "Hurwitz point"
\newcommand{\hurwitz}{\Z\left[\hur,\ii,\jj,\kk\right]}%The set of Hurwitz integers
\usepackage{wrapfig}
\usepackage{calligra}
\usepackage[utf8]{inputenc}
\usepackage[dvips]{graphicx}
\usepackage{a4wide}
\usepackage{amsmath}
\usepackage{euscript}
\usepackage{amssymb}
\usepackage{amsthm}
\usepackage{amsopn}
\usepackage[colorinlistoftodos]{todonotes}
\usepackage{graphicx}
\usepackage[T1]{fontenc}
\newcommand\mybar{\kern1pt\rule[-\dp\strutbox]{.8pt}{\baselineskip}\kern1pt}

\usepackage{ulem}
\usepackage{xcolor}
\newcommand{\cs}[1]{\color{blue}{#1}\normalcolor}

%Matrix commands
\newcommand{\ba}{\begin{array}}
\newcommand{\ea}{\end{array}}
\newcommand{\bmat}{\left[\begin{array}}
\newcommand{\emat}{\end{array}\right]}
\newcommand{\bdet}{\left|\begin{array}}
\newcommand{\edet}{\end{array}\right|}
\newcommand{\inv}[1]{#1^{-1}}

%Environment commands
\newcommand{\be}{\begin{enumerate}}
\newcommand{\ee}{\end{enumerate}}
\newcommand{\bi}{\begin{itemize}}
\newcommand{\ei}{\end{itemize}}
\newcommand{\bt}{\begin{thm}}
\newcommand{\et}{\end{thm}}
\newcommand{\bp}{\begin{proof}}
\newcommand{\ep}{\end{proof}}
\newcommand{\bprop}{\begin{prop}}
\newcommand{\eprop}{\end{prop}}
\newcommand{\bl}{\begin{lemma}}
\newcommand{\el}{\end{lemma}}
\newcommand{\bc}{\begin{cor}}
\newcommand{\ec}{\end{cor}}
\newcommand{\lcm}{\mbox{lcm}}
\newcommand{\defn}{\fbox{definition}}
\newcommand{\prop}{\fbox{proposition}}
\newcommand{\stab}{\mbox{stab}}
\newcommand{\Aut}{\mbox{Aut}}
\newcommand{\orb}{\mbox{orb}}

\newcommand{\norm}{\righttriangle}

\newcommand{\and}{\wedge}
\newcommand{\or}{\vee}



%sets of numbers
\newcommand{\N}{\mathbb{N}}
\newcommand{\Z}{\mathbb{Z}}
\newcommand{\Q}{\mathbb{Q}}
\newcommand{\R}{\mathbb{R}}

\newcommand{\topT}{\mathcal{T}}
\newcommand{\standtop}{\mathcal{T}_{STD}}
\newcommand{\cc}{\mathcal{C}}


\title{Topology}
\author{August bergquist}


\begin{document}

\maketitle
\fbox{Theorem 3.16} Suppose that $X$ is a set and $\mathcal{S}$ is a collection of subsets $X$. Then $\mathcal{S}$ is a subbasis for some topology on $X$ if and only if every point of $X$ is in some element of $\mathcal{S}$.\\

\fbox{proof} Since this statement is biconditional, this proof will be broken into two parts. Let $\mathcal{B}$ denote the set of all finite intersections of sets in $\mathcal{S}$.
\begin{itemize}
    \item[$\Rightarrow$] Suppose that $\mathcal{S}$ is a subbasis for some topology on $X$. Then by definition, the set of all intersections of members of $\mathcal{S}$, $\mathcal{B}$, forms a basis for a topology. Call this topology $\topT$. Let $p$ be arbitrary in $X$. Since by definition of a topology $X$ must be open in $\topT$, we have an open containing $p$. Since $\mathcal{B}$ is a basis for $\topT$, we know that there must exist some basic set $V$ of $\mathcal{B}$ such that $p\in V\subseteq X$. By construction of $\mathcal{B}$, 
    $V = \bigcap_{i = 0}^nS_i$ for elements $S_0,\dots,S_n\in \mathcal{S}$ and for some $n\in \N^0$. Since $p$ is in this intersection, it follows that for each $S_j$ being intersected over, $p\in S_j$. So there must be some $S\in \mathcal{S}$ such that $p\in S$. Since $p$ is arbitrary in $X$, it follows that all points of $X$ are in some element of $\mathcal{S}$.
    \item[$\Leftarrow$] Suppose that each point in $X$ is in some element of $\mathcal{S}$. We will now use theorem 3.3 to show that $\mathcal{B}$ is the basis for some topology on $X$, which will show that $\mathcal{S}$ forms the subbasis for some topology on $X$.
    \begin{enumerate}
        \item First we will need to show that every element of $X$ is in some member of $\mathcal{B}$. Let $p$ be an arbitrary element of $X$. By our supposition, there exists some $S\in \mathcal{S}$ such that $p\in S$. Consider the intersection of only $S$, which will just be $S$. Hence $S$ is an element of $\mathcal{B}$ which contains $p$. Since $p$ was arbitrary in $X$, it follows that all points of $X$ are in some member of $\mathcal{B}$.
        \item Now let $U$ and $V$ be arbitrary elements of $\mathcal{B}$, and let $p$ be arbitrary in $U\cap V$. Since $U$ and $V$ are elements of $\mathcal{B}$, by construction of $\mathcal{B}$ we know that $U$ and $V$ are both finite intsersections of elements in $\mathcal{S}$, hence their union is as well. This means that $U\cap V$ is itself in $\mathcal{B}$. So there is a basic element in $\mathcal{B}$, namely $U\cap V$, such that $p\in U\cap V\subseteq U\cap V$.. Since $U$ and $V$ are arbitrary in $\mathcal{B}$, and $p$ is arbitrary in $U\cap V$, it follows that the second criteria of Theorem 3.3 is met.
        \end{enumerate}
    Since both the first and second criteria of Theorem 3.3 are satisfied, it follow that $\mathcal{B}$ is the basis for some topology on $X$. Since $\mathcal{B}$ is the set of all finite intersections of members of $\mathcal{S}$, it follows that $\mathcal{S}$ forms the subbasis for some topology on $X$.
\end{itemize}

\newpage


\fbox{Theorem 3.30} Let $(X,\topT)$ be a topological space, and let $Y$ be a subset of $X$. Given a basis $\mathcal{B}$ of $X$, the set $\mathcal{B}_Y = \{B\cap Y : B\in \mathcal{B}\}$ is a basis for the subspace $\topT_Y$.\\
\fbox{proof} We will use Theorem 3.1.
\begin{enumerate}
    \item First, let $x$ be arbitrary in $Y$. Since $Y\subseteq X$, and since $\mathcal{B}$ is a basis for $X$, it follows by Theorem 3.1 that there exists some $B$ such that $x\in B$. Since $x\in B$ and $x\in Y$, it follows by definition of the intersection that $x\in B\cap Y = B_Y$. Since $B\in \mathcal{B}$, it follows by construction of $\mathcal{B}_Y$ that $ B_Y \in \mathcal{B}_Y$. Hence there exists some basic set of $\mathcal{B}_Y$ which contains $x$, namely $B_Y$. Since $x$ is arbitrary in $Y$, the first requirement of Theorem 3.1 is satisfied in the case of $\mathcal{B}_Y$.
    \item Now let $U_Y$ be an arbitrary open set in the subspace induced by $Y$. Let $p$ be an arbitrary point in $U_Y$. By definition of the subspace induced by $Y$ we know that $U_Y = U\cap Y$ for some open $U$ set in the space $X$. By definition of the intersection we know that $p\in U$ and $p\in Y$. Since $p\in U$ which is open in $X$, and $\mathcal{B}$ forms a basis for the space $X$, it follows by Theorem 3.1 that there must exist some basic set $V\in \mathcal{B}$ such that $ p\in V\subseteq U $.\\
    
    Consider the set $V_Y = V\cap Y \in \mathcal{B}_Y$. Since, as we have shown, $p\in Y$ and $p\in V$, it follows by definition of the intersection that $p\in V\cap Y = V_Y$. We now need to show that $V_Y \subseteq U_Y$. To do so, let $a$ be arbitrary in $V_Y$. Then by definition of $V_Y$ and by definition of the intersection it follows that $a\in V$ and $a\in Y$. Since by construction $V\subseteq U$, it follows by definition of a subset that $a\in U$ as well. Since $a\in U$ and $a\in Y$, it follows by definition of the intersection that $a\in U\cap Y = U_Y$. Since $a$ was arbitrary in $V$, it follows that $V_Y\subseteq U_Y$.\\
    
    Having shown that there exists some basic set $V_Y$ of $\mathcal{B}_Y$ such that $p\in V_Y\subseteq U_Y$, and since $U_Y$ is arbitrary in $Y$ and $p$ is arbitrary in $U_Y$, it follows that the second requirement of Theorem 3.1 is met.
\end{enumerate}
Having shown that the requirements of Theorem 3.1 are met in the case of $\mathcal{B}_Y$ and $Y$, it follows that $\mathcal{B}_Y$ forms a basis for the subspace induced by $Y$.

\end{document}

