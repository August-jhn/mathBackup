\documentclass[11pt]{article}

\usepackage{amsmath, amssymb, amsthm, graphicx, fancyhdr, textcomp, enumerate, diagbox, tcolorbox, esvect, tikz, adjustbox}


\usepackage{xcolor}

\graphicspath{{./images/}}


\usepackage{halloweenmath, tikzsymbols}

\newcommand{\R}{\mathbb{R}}
\newcommand{\Z}{\mathbb{Z}}
\newcommand{\C}{\mathbb{C}}
\newcommand{\N}{\mathbb{N}}
\newcommand{\Q}{\mathbb{Q}}
\newcommand{\Arg}{\mbox{Arg}}
\newcommand{\Log}{\mbox{Log}}

\newcommand{\recip}[1]{\frac{1}{#1}}


%geometry/topology
\newcommand{\bndry}{\partial}

\newcommand{\inv}[1]{{#1}^{-1}}

\theoremstyle{definition}

\newtheorem*{definition}{Definition}
\newtheorem{theorem}{Theorem}
\newtheorem{corollary}{Corollary}
\newtheorem{proposition}{Proposition}
\newtheorem{remark}{Remark}
\newtheorem{conjecture}{Conjecture}
\newtheorem{lemma}{Lemma}

\title{Real Analysis}
\author{August Bergquist}

\begin{document}

\maketitle

\begin{proposition}
Given a non-zero real number $a$, $a\in \R^+$ if and only if $ \recip{a}\in \R^-$, and $a\in \R^+$ if and only if $\recip{a}\in \R^-$.
\end{proposition}

\begin{proof}
I showed in Lemma 2 of my last assignment that $a\in \R^+$ implies $\recip{a}\in \R^+$. Since $a$ was arbitrary, the converse is an immediate consequence of Corollary 2, also of my last assignment.\\

We now claim that $\recip{-a} = -\recip{a}$. That is, that the multiplicative inverse of the additive inverse of $a$ is the additive inverse of the reciprocal of $a$. Recall the result (we presented it in class) that for real numbers $x,y$, we have $ -(xy) = (-x)y = x(-y)$. Using this, and the definition of multiplicative inverse, as well as Corollary 2 of the last assignment, we have the following string of equalities:
\[
\begin{array}{c}
(-a)(-\recip{a})\\
= -(a(-\recip{a}))\\
= -(-(a\recip{a}))\\
= a\recip{a} = 1
\end{array}
\]
So $ -\recip{a}$ is indeed the multiplicative inverse for $-a$. Recall (once again I believe we presented on this, though it's an easy consequence of the Order Axiom and the definition of negativity) that $a\in \R^+$ if and only if $-a\in \R^-$. So if $a\in \R^-$, $-a\in \R^+$, hence $-\recip{a} \in \R^+ $. Thus $\recip{a}\in \R^-$. Once again, the symmetry implied by Corollary 2 means that the converse follows immediately from this.
\end{proof}

\begin{lemma}
Suppose $a$ and $b$ are both real numbers such that $a > b$. Let $c \in \R^-$. Then $ca < cb$. 
\end{lemma}

\begin{proof} Let all be instantiated as above. By the definition of the relation $>$, we have $a-b$ is positive. Moreover, by Exersize 1.3.1 we have $-c$ is positive, since $c$ is negative and $-(-c) =c$. Therefore $-c(a-b)$ is positive by the closure part of the order axiom. As follows from exersize 1.2.7, we have $-c(a-b) = -(c(a-b))$ is positive. By distributivity (and technically using the definition of subtraction and lemma exercise 1.2.7, we obtain $ca - cb $ is positive. By definition of the order relation, it follows that $ca > cb $  as desired.
\end{proof}


\begin{proposition}
Given non-zero $a,b\in \R$ of the same sign, we have $a> b$ if and only if $\recip{b} > \recip{a}$.
\end{proposition}
\begin{proof}
{\color{red}First suppose that $a$ and $b$ are positive.} From the closure part of the Order Axiom, we have $ ab \in \R^+$. Hence by the first proposition, $\recip{ab}\in \R^+ $ Moreover, by assumption that $a> b$, we also have $a-b\in \R^+$ from the definition of the relation $>$. So then, once again by closure of the positives, we have $ \frac{a-b}{ab} \in \R^+$. By commutativity of multiplication, distributivity of multiplication over addition, the "socks shoes" property of groups, once again commutativity of multiplication, associativity of multiplication, multiplicative inverse, and identity, all ensured by the Field Axiom, we have the following tedious yet beautiful string of equalities (which we don't actually have to write out, but it's interesting to see all of the intricate cogs of our formal system spinning and whirring as they work to mimick the strange universe of the reals!):
\[
\begin{array}{c}
(a-b)\inv{(ab)}  \\
=\inv{(ab)}(a-b) \\
=\inv{ab}(a)-\inv{ab}b  \\
=\inv{b}\inv{a}(a) - \inv{ab}(b) \\
=\inv{b}\inv{a}(a) - \inv{b}\inv{a}(b) \\
=\inv{b}\inv{a}(a) - \inv{a}\inv{b}(b) \\
=\inv{b}(\inv{a}(a)) - \inv{a}\inv{b}(b) \\
=\inv{b}(1) - \inv{a}\inv{b}(b) \\
=\inv{b}(1) - \inv{a}(\inv{b}(b)) \\
=\inv{b}(1) - \inv{a}(1) \\
=\inv{b} - \inv{a}(1)\\
=\recip{b} - \recip{a}
\end{array}
\]
[[Does book-keeping matter? Is this the least number of steps? How may steps are used in a typical proof, if all of them were to be written out like this? A theorem is really a big-ol' package (maybe a sub-routine is a better word) that carries out all of these steps for us all at once. I bet we use thousands of tiny theorems just to prove a medium sized one]] Anyway, well, we've got $\recip{b} - \recip{a} \in \R^+$ upon substiution. But hey! That means (if means is the right word) that $\recip{b} > \recip{a}$. By Corollary 2 of my last lemma, the argument is symmetric. We don't have to get down in the weeds (which I clearly hate doing) and go through the proof of the converse. After all $a$ and $b$ are just run of the mill positives{\color{red} what I meant by this is that they are arbitrary}.\\

{\color{red} 
In the case where both of them are negative, multiply $a$ and $b$ both by $-1$. As problem 1.2 (which I think we presented, $(-1)a = -a$ and similarly for $b$. Using this, and since $-1$ is negative, by the previous lemma we have $-a < -b $. Moreover, since we have now suppose $a$ and $b$ to both be negative, by exercise 1.3.1 that $-a, -b$ are both positive. In the first part of this proof the $a$ and $b$ were arbitrary. Using the fact that $-(1/a) = (1/-a)$ as proved in the last problem, we apply the first half of the proof to obtain the result.
}

\end{proof}


\begin{proposition}
Zero is basically just the reciprocal of a natural number, in the same way that every real number is pretty much just a rational number (especially if your an engineer). In other words, zero can be approximated arbitrarily well by reciprocals of natural numbers. In other words, for all positive $\epsilon\in \R$, there exists a natural number $N$ such that $\recip{N} < \epsilon$.

\end{proposition}

\begin{proof}
{\color{red}Let $\epsilon$ be real.} By the Field Axiom, $\recip{\epsilon}$ exists within the reals. By the Archimedean Principle, there exists some $N \in \N$ (a suggestive name for an $\N$), such that $N < \epsilon$. {\color{red}By the Lemma that all natural numbers are positive, it follows that $N$ is positive.} By Proposition 1, $\recip{\epsilon}$ is positive, cuz $\epsilon$ is. By Proposition 2, it follows that $\recip{N} < \recip{\recip{\epsilon}}$. By the trusty Corollary 2 of last assignment, $\recip{\recip{\epsilon}} = \epsilon$. By substitution, it follows that $\recip{N} < \epsilon$. So $N$ is the desired construction, and this completes the proof.

\end{proof}

\end{document}
