\documentclass[12pt, letterpaper]{article}

\usepackage{amsmath, amssymb, amsthm, graphicx, fancyhdr, textcomp, enumerate, diagbox, tcolorbox, esvect, tikz, adjustbox, xcolor}


\graphicspath{{./images/}}


\usepackage{halloweenmath, tikzsymbols}

\newcommand{\R}{\mathbb{R}}
\newcommand{\Z}{\mathbb{Z}}
\newcommand{\C}{\mathbb{C}}
\newcommand{\N}{\mathbb{N}}
\newcommand{\Q}{\mathbb{Q}}
\newcommand{\Arg}{\mbox{Arg}}
\newcommand{\Log}{\mbox{Log}}




%geometry/topology
\newcommand{\bndry}{\partial}

\newcommand{\inv}[1]{{#1}^{-1}}

\theoremstyle{definition}

\newtheorem*{definition}{Definition}
\newtheorem{theorem}{Theorem}
\newtheorem{corollary}{Corollary}
\newtheorem{proposition}{Proposition}
\newtheorem{remark}{Remark}
\newtheorem{conjecture}{Conjecture}
\newtheorem{lemma}{Lemma}

\title{Real Analysis}
\author{August Bergquist}

\begin{document}

\maketitle

\begin{lemma}\label{unboundedLemma}
    Let $k > 1$. Then the sequence $ n\mapsto k^n $ has the property that for all real $M>0$, there exists some natural number $N$ such that 
    if $n > N$, then $ k^n > M $. 

\end{lemma}

\begin{proof}
    Recall the inequality from my last homework assignment (which I had proven in my last assignment), which states that if $h > 0$, then $(1 + h)^n > 1 + nh$.
    So for any given $M> 0$, we have by the archimedian principle some $N > \frac{M-1}{h}$, with $ N\in \N $. Now suppose $n> N$. Then $n > \frac{M-1}{h}$, 
    whence $k^n > 1 + nh > M$. This concludes the proof.
\end{proof}

\begin{lemma}\label{convergenceLemma}

    Let $0\le k< 1$ for $k\in \R$. Then the sequence $\sigma(n) = k^n$ converges to $0$.  

\end{lemma}

\begin{proof}
    Now let $\epsilon > 0$. Notice that $ \frac{1}{k} > 1 $, so by the last proposition there exists a natural number $N$ with the 
    property that if $n > N$, then $ \frac{1}{k^n} > \frac{1}{\epsilon} $. So for any $n > N$, we have $|k^n - 0| k^n < \epsilon $. This proves the limit. 
\end{proof}

For the next two theorems, let $U\subset \R$ be an open set, and let $f:U \to \R$ be differentiable on all of $U$, and let $p\in U$ such that $f'$ is continuous
at $p$.

\begin{theorem}
     If $|f'(p)| < 1$, then $ p $ is an attractor.
\end{theorem}

\begin{proof}
    Since $f'$ is continuous at $p$, and since the absolute value function is continuous everywhere, it follows that the composition $|f'|$
    is continuous at $p$. Moreover, since $|f'(p)| < 1$, there exists some $|f'(p)|<k < 1$, and $0 \ge |f'(p)|$ by definition of the absolute value.
    Since $ \frac{k - |f'(p)|}{2} $ is positive, by continuity there exists some $ \delta > 0 $ such that if $x\in U$ such that $|x - p| < \delta$,
    then $\big||f'(x)| - |f'(p)||\big| < \frac{k - |f'(p)x|}{2}$. By openess of $U$, there exists an $\epsilon > 0$ such that if $ |x-p| < \epsilon $, then $x\in U$. 
    Set $\eta = \min\{\epsilon, \delta\}$, and construct the interval $ I = (p - \eta, p + \eta)$.

    First, note that for any $x\in I$, $|x - p| < \eta \le \delta$, whence we have by construction of $\delta$ that
    \[
        |f'(x)| < |f'(p)| - \frac{k - |f'(p)|}{2} = \frac{|f'(p)|}{2} + \frac{k}{2} < \frac{k}{2} + \frac{k}{2} = k.
        \]
    From a previous result which we presented in class, it follows that $f$ is Lipschitze with constant $k$. In other words, 
    $ |f(x) - f(p)| = |f(x) - p| \le k|x-p|$. 
    
    I will use this to show that $ |f^n(x) - p| \le k^n|x-p| $ for all $n$, and that $ f^n(x)\in I $ for all $n$. Let the previous observation
    serve as the base case. Moreover, note that since $k < 1$, $|f(x) - p| \le k|x - p| < |x - p| < \eta \le \delta$, so $ |f(x) - p|\in I $.

    Now suppose that for $n\in \N$, $ |f^n(x) - p|\le k^n|x - p|$, and that $f^n(x)\in I$. Then by Lipschitzeness, we have 
    \[
        |f(f^n(x))- f(p)| = |f^{n+1}(x) - p| \le k|f^n(x) - p| \le k(k^n|x-p|) = k^{n +1}|x-p|.
    \]

    It follows then that $|f^n(x) - p|\le k^n|x - p|$ for all $n\in \N$. 

    If $x = p$, it immediately follows that $f^n(x)$ converges to $p$, for $p$ is a fixed point. Assume then that $x \ne p$, from which it follows that 
    $|x - p| > 0$.

    Now let $\epsilon > 0$. Then $\epsilon/|x-p| > 0$. By Lemma\ref{convergenceLemma}, there must exist some natural number $ N $ such that for 
    $ n> N $, $ k^n < \epsilon/|x - p| $. For such an $n > N$, we also have $ 0 \le |f^n(x) - p| < k^n |x - p|$, whence 
    \[ 
        \left|  |f^n(x) - p|-0| < k^n |x - p| < (\epsilon / |x- p|)|x - p| = \epsilon  \right|.
         \]

    By definition of a limit, the sequence $ |f^n(x) - p| $ converges to zero, so by a previous result which we presented in class (relating distance and convergence), 
    it follows that $ f^n(x) \to p $ as $n \to \infty$.

    Our element $x$ was arbitrary in $I$, so we have shown that there exists an open ball $I$ about $p$ such the iterated sequence of $f$ based at $x\in I$ converge to $p$.
    This is the definition of an attractor. 
    
\end{proof}

A quick remark should be made about the definition of a repellor. I think a repellor should be defined as

\begin{definition}
    Given a functino $\phi : X\to X$ in a metric space, a fixed point $q\in X$ is a \textit{repellor} if and only if there exists an open ball $B_\epsilon(q)$ such that
    for any point $x\in B_\epsilon(q)$, there exists a natrual number $n$ such that $\phi^n(x)\not \in B_\epsilon(q)$.
\end{definition}

If this is what a repellor is, then I can prove the following theorem.

\begin{theorem}
    If $ |f'(p)| > 1 $, then $p$ is a repellor.
\end{theorem}

\begin{proof}
    By basic properties of the reals, there is some $K\in \R$ such that $ |f'(p)| > k > 1 $. 

    By the same argument for continuiuty, $|f'|$ is continuous at $p$. Since $\frac{|f'(p)| - k}{2} > 0$, by continuity of $ |f'| $ there must 
    exist some $ \delta > 0 $ such that if $ x\in U $ such that $ | x - p | < \delta $, then $ | |f'(x)| - |f'(p)| | < \frac{k - |f'(p)|}{2}. $
    By openness of $U$, there exists some $\epsilon > 0$ such that $ x\in U $ whenever $ |x - p| < \epsilon $. Let $\eta = \min\{\epsilon, \delta\}$,
    and construct the interval $ I = (p - \eta, p+ \eta) $.

    Let $x\in I$. Then $|x - p| < \eta \le \delta$, hence by construction of $\delta$ we have $ \left||f'(x)| - |f'(p)| \right| < \frac{|f'(p)| - k}{2} $.
    Hence,
    \[
        k = \frac{k}{2} + \frac{k}{2} < \frac{k}{2} + \frac{|f'(p)|}{2} = |f'(p)| -\frac{|f'(p)| - k}{2} < |f'(x)|.
        \]

    So $k < |f'(x)|$ for any $x\in I$.
 
    Suppose now that there existed some $ x\ne p $ in $I$ such that $ |f(x) - f(p)|\le k|x-p| $. 
    Then by the mean value theorem, there exists some 
    $c \in I$ such that $ \frac{f(x) - f(p)}{x-p} = f'(c) $. Hence 
    \[
        |f'(c)| = \frac{|f(x) - f(p)|}{|x - p|} \le k.
        \]
    But $c\in I$, so this is a contradiction.

    From this we obtain the inequality *
    \[
        |f(x) - f(p)| = |f(x) - p| > k|x-p|,
        \]
    for all $x\in I$.

    Using this inequality, we aim to obtain a new inequality
    \[
        |f^n(x) - p| > k^n |x-p|,
        \]
    under the condition that $ f^m(x)\in I $ for $m < n$. 

    The base case has already been established. Now suppose that $|f^n(x) - p| > k^n|x-p|$ with $n\in \N$ such that for $ m<n $, $f^m(x)\in I$. 
    For $n + 1$, there are two possibilities. Either $f^n(x)$ is not in $ I $ or it is. If it isn't, the claim is satisfied.
    So suppose $ f^n(x) \in I $. In this case, we have by the inequality * and the induction hypothesis that $ |f(f^n(x)) - f(p)| > k|f^n(x) - p| >  k^{n+1}|x-p|$.
    This establishes the claim. (Note we can't just assume that this inequality always holds; it only holds so long as we've ``been in $I$ the whole time.'')

    Now we suppose by way of contradiction that $f^n(x)\in I$ for all $n\in \N$. 


    Now chose any $x\in I$ distinct from $p$. We aim to construct a natural number $N$ such that $f^N(x) \not \in I$. 

    Suppose by way of contradiction that $f^n(x)\in I$ for all natural numbers $n$. By Lemma \ref{unboundedLemma} and the fact that $k > 1$, it follows that 
    there exists a natural number $N$ such that $k^N > \frac{\eta}{|x - p|}$. Since by assumption $f^n(x)\in I$ for all $n < N$ (indeed, for all $n\in \N$),
    \[
        |f^N(x) - p| > k^n|x - p| > |x-p|\frac{\eta}{|x-p|} = \eta,
        \]
    hence $f^N(x)\not\in I$, which is a contradiction.

    So then, it follows that there exists an $N$ such that $f^N(x)\not\in I$, (since supposing otherwise gave us a contradiction).
    But $x\in I$ was arbitrary, and so regardless of base point, we'll eventually leave (though we may come back to) $I$ (which is an open interval about $p$). 
    From this it follows that $ p $ is a repellor.
\end{proof} 

This can be used to determine when a fixed point is a repeollor or an attractor.

\begin{proposition}
    For the family of functions $\R\to \R$ defined $f_k:x\mapsto kx(1-x) = kx - kx^2$, the fixed points $x_k = \frac{k-1}{k}$ are attractors if $1 < k < 3$ and rellors when $ 3 < k $.
\end{proposition}

\begin{proof}
    First, note that $f_k$ is infinitely differentiable, so it's derivative is continuous. Upon taking the derivative, we find $f_k'(x) = k- 2kx$. Evaluating at $ x_k $, we get $ f_k(x_k) = k - 2k\frac{k-1}{k} = k - 2k + 2 = 2 - k $.
    The condition that $ |f'(x_k)| = |2- k| < 1 $ is equivalent to the condition that $k \in (2 - 1, 2 + 1) = (1,3)$, so $x_k$ is an attractor when $k$ is in this range. Moreover, for $k > 3$, we have $ 2 - k < 2 - 3 = -1$,
    from which it follows that $|f'(x_k)| > 1$ for $k > 3$, hence $f_k$ is a repellor when this is the case.
\end{proof}

\end{document}
