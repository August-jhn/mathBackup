\documentclass[12pt]{article}

\parindent0pt
\parskip10pt

\usepackage{amsmath, amssymb, amsthm, graphicx, fancyhdr, textcomp, enumerate, svg}


\graphicspath{{./images/}}

\newcommand{\R}{\mathbb{R}}
\newcommand{\Z}{\mathbb{Z}}
\newcommand{\C}{\mathbb{C}}
\newcommand{\N}{\mathbb{N}}
\newcommand{\Q}{\mathbb{Q}}
\newcommand{\Arg}{\mbox{Arg}}
\newcommand{\Log}{\mbox{Log}}

% Analysis

\newcommand\norm[1]{\lVert#1\rVert} %for norms and meshes
\newcommand\rsum[2]{\mathcal{R} (#1,#2)} %R(f,P) for Reimann sums

%geometry/topology
\newcommand{\bndry}{\partial}

\newcommand{\inv}[1]{{#1}^{-1}}

\theoremstyle{definition}

\newtheorem*{definition}{Definition}
\newtheorem{theorem}{Theorem}
\newtheorem{corollary}{Corollary}
\newtheorem{proposition}{Proposition}
\newtheorem{remark}{Remark}
\newtheorem{conjecture}{Conjecture}
\newtheorem{lemma}{Lemma}

\title{Real Analysis}
\author{August Bergquist}

\begin{document}

\section{Problem 12.5.5: an interesting function} 

Consider the function $f: \R\to \R$ defined 
\[
    f(x) = 
    \begin{cases}
        0 \mbox{ if $x$ is irrational }\\
        1/q \mbox{ if $x$ is rational, where $q$ is the denominator in lowest terms. }
    \end{cases}
    \]

Its fairly easy to see that this function is discontinuous for all rational numbers. Recall one of the equivalent definitions of continuity, which states that a function $g:X\to Y$ is continuous at some $a\in X$ iff for all $\epsilon > 0$, there exists a $\delta > 0$ such that whenever $x\in X$ with $d(x,a) < \delta$, $d(g(x), g(a)) < \epsilon$. Negating this, we must show that there is some $\epsilon > 0$ such that whenever $ \delta > 0 $, there is some $x\in X$ with $d(x,a) < \delta$ and $ d(g(x), g(a)) \ge \epsilon $. 

\begin{proposition}
    The function $f$, defined above, is discontinuous at all $s\in \Q$.
\end{proposition}

\begin{proof}
Back to our specific function, let $s\in \Q$. Then, by definition of a rational number $s = p/q$ (in lowest terms), with $p\in \Z$ and $q\in \N$. Let $\delta > 0$. Consider the positive real number $1/q$. Using a basic property of the reals (which we proved!), there must exist some irrational $r $ with $ s < r < s + \delta$. From this it follows that $ |r - s| < \delta $. Moreover, by definition of $f$, we have $f(s) = 1/q$, and $ f(r) = 0 $. So $|f(s) - f(r)| = |1/q - 0| = 1/q \ge 1/q$. But $\delta$ was arbitrary greater than zero, so for all $\delta$, there exists some $ r\in \R $ with $ |r -s| < \delta $ and $ |f(r) - f(s) \ge \epsilon $. Hence $f$ is discontinuous at $s$!
\end{proof}


Now we'll take it's integral. Unfortunately (or fortunately for coolness's sake), we can't take an antiderivative, for clearly none exists (it isn't continuous on any interval, since every interval contains a rational). Along the way, we'll prove the function is continuous at every irrational number.

\begin{lemma}
    Let $t\in \R$. Let $r > 0$, and let $ n\in \N $. Construct the set $S_n = \{s\in [t - r, t + r]: f(s) = 1/n\}$. Then $S_n$ is finite.
\end{lemma}

\begin{proof}
    Let $s\in S_n$. Then $ f(s) = 1/n $. This isn't zero, hence $s$ is rational, and by definition of $f$, $s = p/n$ for some $p\in \Z$, where the fraction $p/n$ is in reduced terms. But since $s\in [t - r, t + r]$, it must be the case that $ t - r\le s=p/n\le t + r$, and since $n $ is positive, this is equivalent to 
    \[
        p\in [n(t - r) , n(t + r)].
        \]
    But since $p$ is an integer, only finitely many such $p$ can fit within this interval. So only finitely many elements could be in $S_n$, for each would correspond to a distinct integer $p\in [n(t - r) , n(t + r)]$.
\end{proof}

This allows us to establish an important property on how "big" the function can be, and how often it can be that big. 

\begin{lemma}
    Let $K\subset \R$ be bounded. Then for all $\epsilon > 0$, the set $ A = \{s\in K: f(s) > s\} $ is finite.
\end{lemma}

\begin{proof}
    First, note that since $K$ is bounded, it follows that it lives inside of some open interval of some radius, and hence inside the corresponding closed interval. Let that closed interval be $ [t - r, t + r] $ for some $t\in r$, and for some $r> 0$. Note that the set $ S_\epsilon = \{s\in [t - r, t + r]: f(s) < 1/n\} $

    By the archimedian principle, there must exist some natural number $N$ such that $1/N < \esilon$. Note that if $  $
\end{proof}




\end{document}
