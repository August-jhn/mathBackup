\documentclass[12pt]{article}

\parindent0pt
\parskip10pt

\usepackage{amsmath, amssymb, amsthm, graphicx, fancyhdr, textcomp, enumerate}


\graphicspath{{./images/}}

\newcommand{\R}{\mathbb{R}}
\newcommand{\Z}{\mathbb{Z}}
\newcommand{\C}{\mathbb{C}}
\newcommand{\N}{\mathbb{N}}
\newcommand{\Q}{\mathbb{Q}}
\newcommand{\Arg}{\mbox{Arg}}
\newcommand{\Log}{\mbox{Log}}

% Analysis

\newcommand\norm[1]{\lVert#1\rVert} %for norms and meshes
\newcommand\rsum[2]{\mathcal{R} (#1,#2)} %R(f,P) for Reimann sums

%geometry/topology
\newcommand{\bndry}{\partial}

\newcommand{\inv}[1]{{#1}^{-1}}

\theoremstyle{definition}

\newtheorem*{definition}{Definition}
\newtheorem{theorem}{Theorem}
\newtheorem{corollary}{Corollary}
\newtheorem{proposition}{Proposition}
\newtheorem{remark}{Remark}
\newtheorem{conjecture}{Conjecture}
\newtheorem{lemma}{Lemma}

\title{Real Analysis}
\author{August Bergquist}

\begin{document}

\maketitle

\section{The Uniformly Convergent Limit of Continuous Functions is Continuous}

\begin{proposition} Let $X$ and $Y$ be metric spaces, and let $f_n:X\to Y$ be a uniformly convergent sequence of continuous functions, which converge 
to some function $ f:X\to Y $. Then $f$ is continuous.
\end{proposition}

\begin{proof} 

Let everything be instantiated as above. Suppose for the sake of contradiction that $f$ is not continuous. Then there exists some point of discontinuity 
$a\in X$. Hence there exists some $\epsilon > 0$ such that for all $\delta > 0$, there exists some $ x\in X $ with $ d_X(x,a) <\delta$ and
$\epsilon \le d_Y(f(x), f(a))$. 

By uniform convergence of $f_n$, there exists natural number $N$ such that for all $n> N$, and for all $x\in X$, $d_Y(f(x), f_n(x)) < \epsilon/3$. In 
particular, $ d_Y(f_n(a),f(a))\le \epsilon/3 $.

Moreover, by assumption that each $f_n$ is continuous, it is continuous in particular at $a$. Hence there exists some $\delta > 0$ such that for any $ 
x\in X$ with $d_X(x,a) <\delta$, we must have $ d_Y(f_n(x), f_n(a)) < \epsilon/3 $.

But by construction of $\epsilon$, there exists some $ b\in X $ such that $ d_X(b,a) < \delta $ and $ <\epsilon \le d_Y(f(a),f(b)) $. But $ d_Y(f(a), 
f_n(a)) < \epsilon/3$, $ d_Y(f_n(a), f_n(x)) < \epsilon/3 $ and $ d_Y(f_n(x), f(x)) < \epsilon/3 $, Hence by the triangle inequality we have 
\begin{equation}
    \begin{split}
        \epsilon \le d_Y(f(a), f_n(a)) \\
        \le d_Y(f(a), f_n(a)) + d_Y(f_n(a), f_n(x)) + d_Y(f_n(x), f(x))  \\
        < \epsilon/3 + \epsilon/3 + \epsilon/3 =\epsilon.
    \end{split}
\end{equation}
 Hence $\epsilon < \epsilon$, which is a contradiction. \end{proof}

\section{Uniform Convergence and Arithmetic}

\begin{lemma}
 Let $X$ be a metric space, and let $f_n:X\to \R$ be a uniformly convergent sequence of functions, which converge to $f:X\to \R$. Suppose 
$f$ is bounded by $M\in \R$, such that for all $x\in X$, 
\[
	|f(x)| \le M.
\]

Then for all $\epsilon > 0 $ there exists a natural number $N\in \N$ such that for all $n> N$, and for all $x\in X$, 
\[
	|f_n(x)| \le M + \epsilon.
\]

\end{lemma}

\begin{proof} Let everything be instantiated as above. Let $\epsilon > 0$. By uniform convergence of $(f_n)$, we have some natural number $N$ such 
that for all $n > N$, $ |f(x) - f_n(x)| < \epsilon $. Take any $n> N$. Then by the Triangle Inequality, we have
\begin{equation}
	\begin{split}
		|f_n(x)| = |f_n(x) - 0| \\
		|f_n(x) - f(x)| + |f(x) - 0| = |f_n(x) - f(x)| + |f(x)|.
	\end{split}
\end{equation}
But $ x\in X $, so $ |f(x)| \le M $. Moreover, since $n> N$, by construction of $N$ we have $ |f_n(x) - f(x) <\epsilon$, hence
\[
|f(x)| < M + \epsilon.
\]

 \end{proof}

For what follows, let $X$ be a metric space, and let $f_n$ and $g_n$ be sequences of functions $ X\to \R $, which uniformly converge to functions 
$f,g:X\to \R$ respectively.

\begin{proposition}
The sequence of pointwise defined sums, $ f_n+g_n $, converges uniformly to the pointwise sum $f+g$.
\end{proposition}

\begin{proof}
For let $\epsilon > 0$. Then $\epsilon /2 > 0$. By uniform convergence of $ f_n $ to $f$, there exists some natural number $N_1$ such that for all 
$n>N_1$, and for all $x\in X$, $ |f_n(x) - f(x) | < \epsilon/2 $. Similarly, there exists some natural number $N_2$ such that for all $n>N_2$, and for 
all $x\in X $, $| g_n(x) + g(x) | < \epsilon /2$. 

Consider $ N = \max(N_1, N_2)$. Let $n> N$, and let $x\in X$. Since $n> N$, $n> N_1, N_2$, hence $ |f_n(x) - f(x)| < \epsilon/2 $ and $ |g_n(x) - 
(x)|<\epsilon /2 $. From this it follows that 
\begin{equation}
	\begin{split}
		|(f_n + g_n)(x) - (f+g)(x)| = |f_n(x) + g_n(x) - (f(x) + g(x))| \\
		= |(f_n(x) - f(x) + (g_n(x) - g(x))| \\
		\le |f_n(x) - f(x)| + |g_n(x) - g(x)| < \epsilon/2 + \epsilon/2 = \epsilon.
	\end{split}
\end{equation}
This concludes the proof.
\end{proof}

\begin{proposition}
If, in addition, $f$ and $g$ are both bounded, then the pointwise product $f_ng_n$ converges uniformly to the pointwise product $fg$. 
\end{proposition}

\begin{proof}
Let $\epsilon> 0$. Since $f$ and $g$ are bounded, there exist real numbers $F, G> 0$ such that for all $ x\in X $, $ |f(x)| \le F $ and $ |g(x)|\le G $. 
Hence 
\[
	\eta = \frac{\epsilon}{F + G} > 0.
\]

By the above lemma, it follows that there exists some natural $N_1$ such that for all $n > N_1$, and for all $x\in X$, $ |f_n(x)|\le F + 
\eta$. By the same lemma, there exists some natural number $ N_2 $ such that for all $n> N_2$, and for all $x\in X$, $ |g_n(x)| < G + \eta $. 
Moreover, by uniform convergence of $f_n$ to $f$, there exists some natural number $N_3$ such that for all $n>N_3$, and for all $x\in X$, 
$| f_n(x) - f(x) | < \eta$. By uniform convergence of $g_n$ to $g$, there exists some natural number $N_4$ such that for all $ n> N_4 $, and 
for all $x\in X$, $ |g_n(x) -g(x)| < \eta $. Let $ N = \max\{N_1, N_2, N_3, N_4\} $, and suppose $n> N$ and $x\in X$. Then 
\begin{equation}
	\begin{split}
		|(f_ng_n)(x) - fg(x)| = |f_n(x)g_n(x) - f(x)g(x)| \\
		= |f_n(x)g_n(x) - f_n(x)g(x) + f_n(x)g(x) - f(x)g(x)|\\
		\le |f_n(x)g_n(x) - f_n(x)g(x)| + |f(x)g(x) - f_n(x)g(x)|\\
		= |f_n(x)||g_n(x) - g(x)| + |g(x)||f(x) - f_n(x)| \\
		< (F + \eta)\eta + G\eta = (\eta + F + G)\eta < (F+G)\eta\\
		= (F+G)\frac{\epsilon}{F+G} = \epsilon.
	\end{split}
\end{equation}

Generalizing and stuff, by definition of uniform convergence, $f_ng_n$ converges uniformly to $fg$ as desired.

\end{proof}

\end{document}
