\documentclass[11pt]{article}

\usepackage{amsmath, amssymb, amsthm, graphicx, fancyhdr, textcomp, enumerate, diagbox, tcolorbox, esvect, tikz, adjustbox}


\graphicspath{{./images/}}


\usepackage{halloweenmath, tikzsymbols}

\newcommand{\R}{\mathbb{R}}
\newcommand{\Z}{\mathbb{Z}}
\newcommand{\C}{\mathbb{C}}
\newcommand{\N}{\mathbb{N}}
\newcommand{\Q}{\mathbb{Q}}
\newcommand{\Arg}{\mbox{Arg}}
\newcommand{\Log}{\mbox{Log}}


%geometry/topology
\newcommand{\bndry}{\partial}

\newcommand{\inv}[1]{{#1}^{-1}}

\theoremstyle{definition}

\newtheorem*{definition}{Definition}
\newtheorem{theorem}{Theorem}
\newtheorem{corollary}{Corollary}
\newtheorem{proposition}{Proposition}
\newtheorem{remark}{Remark}
\newtheorem{conjecture}{Conjecture}
\newtheorem{lemma}{Lemma}
\newtheorem{lameobservation}{Lame Observation}

\title{Real Analysis}
\author{August Bergquist}

\begin{document}

\maketitle

\section{9.4.6}

In all that follows, let $K$ be a "D"-domain. Recall that an antiderivative of a function $f:K\to \R$ is a function $F:K\to \R$ (which is then differentiable on $K$), such that $F'(x) = f(x)$ for all $x\in K$.

\begin{proposition}
Suppose that $f$ is a function $K\to \R$, where $K$ is a D-domain and $f$ differentiable on $K$, such that there exists some constant $a\in \R$ such that $f'(x) = a$ for all $x\in K$. Then there exists some constant $b\in \R$ such that $f(x) = ax + b$ for all $x\in K$.

\end{proposition}

\begin{proof}
First, we have already shown that antiderivatives differ by a constant. Therefore, it suffices to show that the function defined $g(x) = ax $ for all $x\in K$ is an antiderivative of the constant function $a$. This is obvious, for when the difference quotient is taken, we find that the difference quotient is constant at $a$. \\

Since $g(x)$ is an antiderivative for the constant function $a$, and since by definition of an antiderivative, so is $f$, it follows $f$ and $g$ differ by some constant $b\in \R$. In other words, $f(x) = g(x) + b = ax + b$ for all $x\in K$, which was what we wanted to show.
\end{proof}

\section{Strict monotonicity and derivatives}

\begin{proposition}
Let $f:K\to \R$ be differentiable on $[a,b]\subset K$ with $a < b$, such that $f'(x) < 0$ for all $ x\in [a,b]$. Then $f$ is strictly decreasing on $[a,b]$. 
\end{proposition}

\begin{proof}
Since $f'(x) < 0$ on all $[a,b]$, it follows by Theorem 9.5.2 of the textbook that $f$ is decreasing. The only issue we could run into is that $x < y$ in $[a,b]$ while $f(x) = f(y)$ in $\R$. This possibility is ruled out by Theorem 9.5.3, for $f'(x) < 0$ implies $f'(x)\ne 0$, for all $x\in (a,b)$. Hence $f$ is one-to-one.
\end{proof}

The converse is false. To see this, consider the map 

$[-1,1]\to \R$ defined $f(x) = -x^3$ for all $x\in [-1,1]$. This function is strictly decreasing, since $x^2>0$ when $x>0$, hence if $x, y\in [-1,1]$ if we have $x< y$, then $-x^3 = -x(x^2) < -y(y^2)$. Moreover, it's derivative is zero (which is obvious if we assume the power rule, and even if we don't).

To make sure it actually does have zero derivative at zero, notice for any $x$ and $y$, $y^3 - x^3 = (y - x)(y^2 + xy + x^2) $, hence the difference quotient is 
$(y^3 - x^3)/(y-x) = (y^2 + xy + x^2)$. Fixing $x = 0$, the difference quotient becomes $y^2$, hence the derivative is $\lim_{y\to 0}y^2$, which is $0$ since $y^2$ is continuous and $0^2 = 0$. 
\end{document}