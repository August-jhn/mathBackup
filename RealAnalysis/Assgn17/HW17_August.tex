\documentclass[12pt]{article}

\parindent0pt
\parskip10pt

\usepackage{amsmath, amssymb, amsthm, graphicx, fancyhdr, textcomp, enumerate, svg}


\graphicspath{{./images/}}

\newcommand{\R}{\mathbb{R}}
\newcommand{\Z}{\mathbb{Z}}
\newcommand{\C}{\mathbb{C}}
\newcommand{\N}{\mathbb{N}}
\newcommand{\Q}{\mathbb{Q}}
\newcommand{\Arg}{\mbox{Arg}}
\newcommand{\Log}{\mbox{Log}}

% Analysis

\newcommand\norm[1]{\lVert#1\rVert} %for norms and meshes
\newcommand\rsum[2]{\mathcal{R} (#1,#2)} %R(f,P) for Reimann sums

%geometry/topology
\newcommand{\bndry}{\partial}

\newcommand{\inv}[1]{{#1}^{-1}}

\theoremstyle{definition}

\newtheorem*{definition}{Definition}
\newtheorem{theorem}{Theorem}
\newtheorem{corollary}{Corollary}
\newtheorem{proposition}{Proposition}
\newtheorem{remark}{Remark}
\newtheorem{conjecture}{Conjecture}
\newtheorem{lemma}{Lemma}

\title{Real Analysis}
\author{August Bergquist}

\begin{document}

    \maketitle

    \section{Converse Part of the Cauchy Criterion Theorem}

    \begin{theorem}
        Suppose $f:[a,b]\to \R$ is Reimann integrable. Then it satisfies the Cauchy Criterion.
    \end{theorem}

    \begin{proof}
        Let $\epsilon > 0$. Then $\epsilon/2 > 0$. By integrability, it follows that there exists some $ \delta > 0$ such that whenever $\mathcal{R}(f,P)$ is a Reimann sum for a partition $P$ of mesh less than $\delta$, we have 
        \[
        \left|\mathcal{R}(f,P) - \int_a^bf\right| < \epsilon/2.     
        \]
        Now consider any such partition $P$ with $\norm{P} < \delta$, and consider any two Reimann sums $\mathcal{R}_1(f,P)$ and $\mathcal{R}_2(f,P)$. By construction of $\delta$ and $P$, it follows that 
        \[
        \left|\mathcal{R}_{1,2}(f,P) - \int_a^bf\right| < \delta.    
        \]
        Then 
        \begin{equation}
            \begin{split}
            \left|\mathcal{R}_1(f,P)-\mathcal{R}_2(f,P)\right|  \\
            = \left|\mathcal{R}_1(f,P)- \int_a^bf + (I - \mathcal{R}_2(f,P))\right| 
            \le \left|\mathcal{R}_1(f,P)-\int_a^bf\right| + \left|\mathcal{R}_2(f,P)-\int_a^bf\right| < \epsilon/2 + \epsilon/2 = \epsilon.
            \end{split}
        \end{equation}

        This proves that $f$ on $[a,b]$ satisfies the Cauchy Criterion.
        
    \end{proof}

    \section{Upper and Lower Sum Criterion}

    \begin{proposition}
        Let $f:[a,b]\to \R $ be a bounded function (boundedness is needed to make sense of upper and lower sums). Then for all $\epsilon > 0$, there exists some $\delta > 0$ such that whenever $ P $ is a partition of $[a,b]$ with $\norm{P} < \delta$, the inequality
        \[
        \left| \mathcal{U}(f,P) - \mathcal{L}(f,P) \right| < \epsilon    
        \] 
        holds, if and only if $f$ is Reimann integrable on $[a,b]$.
    \end{proposition}

    \begin{proof}
        First suppose that $f$ is Reimann integrable. Let $\epsilon > 0$. Then $\epsilon /2 > 0$. By the converse to the Cauchy Criterion Theorem, there exists some $\delta$ such that for all $P$ with $\norm{P} < \delta$, and for any Reimann sums $\mathcal{R}_1(f,P)$ and $\mathcal{R}_2(f,P)$, we have 
      \[
        \left| \R_2{U}(f,P) - \R_1{L}(f,P) \right| < \epsilon/2.
        \]   
        Chose any such $P$. Then the above inequality holds for any Reimann sums of the above form. By Lemma 11.4.7 of the textbook, we have
        \[
            \left| \mathcal{U}(f,P) - \mathcal{L}(f,P) \right| \le \epsilon/2 < \epsilon.
            \]

            The partition $P$ was arbitrary with mesh size less than $\delta$, hence for any partition, the above inequality would hold. This proves the converse.

        Now suppose that for all $\epsilon > 0$, there exists some $\delta > 0$ such that whenever $ P $ is a partition of $[a,b]$ with $\norm{P} < \delta$, the inequality
        \[
        \left| \mathcal{U}(f,P) - \mathcal{L}(f,P) \right| = \mathcal{U}(f,P) - \mathcal{L}(f,P) < \epsilon.
        \] 

        For $\epsilon$ fixed, we have one such $\delta.$ Suppose we have some parition $P$ with mesh size less that $\delta$. Let $\mathcal{R}_1(f,P)$ and $ \mathcal{R}_2(f,P)$ be Reimann sums for this partition. Since the inequality with the upper and lower sums hold for this particular $\delta$, and this particular $\epsilon$, and this particular partition, we have $\mathcal{L}(f,P)\le \mathcal{R}_1(f,P), \mathcal{R}_2(f,P)\le \mathcal{U}(f,P) $. Without loss of generality, suppose that $\mathcal{R}_1(f,P) \ge \mathcal{R}_2(f,P)$, whence $|\mathcal{R}_1(f,P) - \mathcal{R}_2(f,P)| = \mathcal{R}_1(f,P) - \mathcal{R}_2(f,P)$. Since $\mathcal{U}(f,P) \ge \mathcal{R}_1(f,P)$ and $ \mathcal{L} \le \mathcal{R}_2(f,P) $, we have 
        \[
        |\left|\mathcal{R}_1(f,P) - \mathcal{R}_2(f,P)\right| = \mathcal{R}_1(f,P) - \mathcal{R}_2(f,P) \le \mathcal{U}(f,P) - \mathcal{L}(f,P) < \epsilon.
        \]
        Since $\epsilon$ was arbitrary, and $P$ was as well, and so were the Reimann sums, it follows that $f$ satisfies the Cauchy Criterion on $[a,b]$. From this and the Caucy Criterion Theorem, it follows that $f$ is Riemann integrable on $[a,b]$.
    \end{proof}

    \section{A Refinement, and Reimann sum, that isn't better}

    Consider the function $f(x)$ which is defined as $-1$ if $x>0$ and $1$ otherwise. This is Reimann integrable on $[-1,1]$, and it's integral is zero. 

\end{document}
